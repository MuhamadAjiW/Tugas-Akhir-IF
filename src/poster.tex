%%%%%%%%%%%%%%%%%%%%%%%%%%%%%%%%%%%%%%%%%
% a0poster Landscape Poster - REVAMPED STYLE
% Original Template from: http://www.LaTeXTemplates.com
%%%%%%%%%%%%%%%%%%%%%%%%%%%%%%%%%%%%%%%%%

%----------------------------------------------------------------------------------------
%   PACKAGES AND OTHER DOCUMENT CONFIGURATIONS
%----------------------------------------------------------------------------------------

\documentclass[a2,portrait]{config/poster/a0poster}
\usepackage{config/poster/a0size}

% Core Packages
\usepackage{multicol} % For multiple columns
\usepackage[svgnames]{xcolor} % For custom colors
\usepackage{times} % Use the times font
\usepackage{graphicx} % For including images
\usepackage{booktabs} % For professional tables
\usepackage[font=small,labelfont=bf]{caption} % For figure/table captions
\captionsetup{font=footnotesize,aboveskip=1pt,belowskip=0pt}
\usepackage{amsfonts, amsmath, amsthm, amssymb} % Math packages
\usepackage[style=ieee]{biblatex} % IEEE style citations
\usepackage{tikz} % For creating diagrams
\usepackage{pgfplots} % For creating plots
\pgfplotsset{compat=1.18}
\usepackage{enumitem} % compact lists

% Styling Packages
\usepackage[most]{tcolorbox} % For creating colored boxes (headers)

\hyphenpenalty=100000
\tolerance=10000

%----------------------------------------------------------------------------------------
%   DOCUMENT CONFIGURATION
%----------------------------------------------------------------------------------------

% Layout settings
\columnsep=4pt % Smaller space between columns to save width

% Reduce vertical spacing for captions so figures use less vertical real-estate
\setlength{\abovecaptionskip}{1pt}
\setlength{\belowcaptionskip}{0pt}
\setlength{\parskip}{0pt}

% Tighter list spacing globally
\setlist[itemize]{leftmargin=*, itemsep=0pt, topsep=0pt, parsep=0pt, partopsep=0pt}

% Set graphic path
\graphicspath{{./}{resources/chapter-4/}} % Location of the graphics files

% Bibliography resources
\addbibresource{config/paper/IEEEabrv.bib}
\addbibresource{references.bib}

% Custom Colors and Commands
\definecolor{ITBblue}{rgb}{0.0, 0.21, 0.42} % ITB blue branding
\definecolor{ECcolor}{rgb}{0.8, 0.2, 0.2} % Erasure coding color
\definecolor{REPcolor}{rgb}{0.2, 0.6, 0.2} % Replication color

% Command for creating a styled section header
\newcommand{\postersection}[1]{%
	\begin{tcolorbox}[
		colback=ITBblue,
		colframe=ITBblue,
		fonttitle=\bfseries,
		coltext=white,
		sharp corners,
		boxrule=0pt,
		top=0pt,
		bottom=0pt,
		halign=center
	]
	\normalsize #1
	\end{tcolorbox}%
}

\begin{document}

%----------------------------------------------------------------------------------------
%   POSTER HEADER 
%----------------------------------------------------------------------------------------

\hspace{-0.5cm} 
\begin{minipage}[c]{\linewidth}
\vspace*{0.5cm}
\Huge \textbf{Analisis Kinerja Erasure Coding vs Replikasi \newline pada Key-Value Store Terdistribusi} \\
\newline
\noindent \small \textit{Erasure Coding Performance Analysis Against Replication on Distributed Key-Value Store Database} \\
\noindent \small \textbf{Muhamad Aji Wibisono (13521095)} \\
\noindent \small Program Studi Teknik Informatika, Institut Teknologi Bandung \\
\end{minipage}
\vspace*{-0.75cm}
\noindent

%----------------------------------------------------------------------------------------
%   POSTER BODY
%----------------------------------------------------------------------------------------

\begin{multicols}{2} % Use 2 columns for the body
\normalsize % slightly larger body text while keeping one page

% One-column image helper: image will occupy a single multicols column (width=\columnwidth)
\newcommand{\onecolimg}[2]{%
	\begin{center}
		\parbox{\columnwidth}{%
			\centering
			\includegraphics[width=\linewidth,height=0.18\textheight,keepaspectratio]{#1}
			\captionof{figure}{#2}
		}
	\end{center}
}

%----------------------------------------------------------------------------------------
%   RINGKASAN
%----------------------------------------------------------------------------------------

\postersection{Ringkasan}
\noindent Penelitian ini menganalisis kinerja \textit{erasure coding} dibandingkan replikasi pada \textit{distributed key-value store}, berfokus pada \textit{response time} operasi \textit{write} dan \textit{read}. Motivasi utamanya adalah kebutuhan efisiensi penyimpanan dan ketahanan data dengan tetap menjaga kinerja. Hasil utama: terdapat kondisi ambang (threshold) di mana \textit{erasure coding} mengungguli replikasi pada operasi \textit{write}; sedangkan pada operasi \textit{read}, replikasi selalu lebih cepat karena biaya rekonstruksi pada \textit{erasure coding}.
\newline
\vspace*{-0.5cm}

\noindent \textbf{Kata kunci:} Erasure coding, replikasi, key-value store, kinerja, \textit{distributed systems}

%----------------------------------------------------------------------------------------
%   PERTANYAAN & TUJUAN
%----------------------------------------------------------------------------------------

\postersection{Pertanyaan Riset dan Tujuan}
\begin{itemize}
	\item Dalam kondisi apa \textit{erasure coding} memiliki \textit{response time} lebih rendah daripada replikasi?
\end{itemize}
\begin{itemize}
	\item Menilai kelayakan kinerja \textit{erasure coding} pada \textit{key-value store} terdistribusi dan menemukan ambang kinerja tersebut.
\end{itemize}

%----------------------------------------------------------------------------------------
%   METODE
%----------------------------------------------------------------------------------------

\postersection{Metode dan Sistem}
\begin{itemize}
	\item Rust + RocksDB (\textit{persistent store}), OmniPaxos (konsensus), Reed–Solomon (\textit{erasure coding}).
	\item Antarmuka klien sederhana; \textit{in-memory store} hanya untuk optimasi (diabaikan saat evaluasi \textit{read}).
\end{itemize}
\begin{itemize}
	\item Variasi \textit{bandwidth}: 1 Mbps; 10–70 Mbps (rata-rata Indonesia ±40 Mbps); 10 Gbps.
	\item Ukuran \textit{payload}: 1 KB; 200–1000 KB.
	\item Beban kerja: \textit{write} dan \textit{read}. Pengukuran: \textit{response time} end-to-end.
\end{itemize}

%----------------------------------------------------------------------------------------
%   HASIL WRITE
%----------------------------------------------------------------------------------------

\postersection{Hasil: Operasi Write}
\begin{itemize}
	\item \textbf{Internet cepat, payload kecil} (10 Gbps, \textasciitilde1 KB): replikasi lebih cepat; biaya komputasi \textit{encoding} mendominasi.
	\item \textbf{Internet lambat, payload besar} (\textless{}\,10 Mbps; 200–1000 KB): \textit{erasure coding} lebih cepat; pengurangan data kirim mengalahkan biaya \textit{encoding}.
	\item \textbf{Internet menengah, payload besar} (10–70 Mbps): terdapat \textbf{ambang} peralihan; bergantung kombinasi \textit{bandwidth} dan ukuran data.
\end{itemize}

%----------------------------------------------------------------------------------------
%   AMBANG KINERJA
%----------------------------------------------------------------------------------------

\postersection{Analisis Ambang Kinerja}
\begin{itemize}
	\item Pemodelan batas menggunakan \textit{ridge regression} dari hasil \textit{benchmark} (25 titik data) untuk memetakan area unggul EC vs replikasi.
	\item Kurva batas menunjukkan area kombinasi \textit{bandwidth}-ukuran data di mana EC unggul pada \textit{write}. Lihat figur batas di akhir poster.
	\item Model valid pada lingkungan uji; faktor lain (CPU/disk) tidak dimodelkan eksplisit.
\end{itemize}

%----------------------------------------------------------------------------------------
%   HASIL READ
%----------------------------------------------------------------------------------------

\postersection{Hasil: Operasi Read}
\begin{itemize}
	\item Replikasi selalu lebih cepat; EC membutuhkan rekonstruksi data (mengumpulkan \textit{shard} + \textit{decode}).
	\item Tidak ditemukan ambang di mana EC mengungguli replikasi.
	\item Dengan \textit{in-memory store}, beban \textit{decode} berkurang proporsional \textit{hit rate}; diabaikan pada evaluasi inti.
\end{itemize}

%----------------------------------------------------------------------------------------
%   RINGKASAN TEMUAN
%----------------------------------------------------------------------------------------

\postersection{Ringkasan Temuan}
\begin{itemize}
	\item Trade-off utama: \textbf{Efisiensi penyimpanan} (EC) vs \textbf{kesederhanaan dan \textit{latency} baca} (replikasi).
	\item EC unggul pada \textit{write} saat data besar dan jaringan terbatas; replikasi unggul pada data kecil dan/atau jaringan cepat.
	\item Di \textit{key-value store} praktis (objek kecil, \textit{DC} ber-\textit{bandwidth} tinggi), replikasi cenderung pilihan default; EC berguna untuk skenario khusus/arsip/edge.
\end{itemize}

%----------------------------------------------------------------------------------------
%   KONDISI JARINGAN
%----------------------------------------------------------------------------------------

\postersection{Implikasi Kondisi Jaringan}
\begin{itemize}
	\item Jaringan lambat mendukung EC (data kirim lebih sedikit); jaringan cepat mendukung replikasi (overhead komputasi EC tidak terbayar).
	\item Perpindahan batas dipengaruhi ukuran data, \textit{bandwidth}, dan biaya komputasi.
\end{itemize}

%----------------------------------------------------------------------------------------
%   KESIMPULAN
%----------------------------------------------------------------------------------------

\postersection{Kesimpulan dan Rekomendasi}
			\textbf{Kesimpulan:}
\begin{itemize}
	\item EC unggul pada \textit{write} untuk data besar di jaringan terbatas; replikasi unggul pada \textit{read} dan pada data kecil/jaringan cepat.
	\item Tidak ada skenario \textit{read} di mana EC mengalahkan replikasi tanpa bantuan \textit{cache}.
\end{itemize}
\begin{itemize}
	\item \textbf{Pilih EC}: arsip/\textit{edge}/link terbatas, beban \textit{write} dominan, objek besar, hemat penyimpanan penting.
	\item \textbf{Pilih replikasi}: \textit{DC} ber-\textit{bandwidth} tinggi, objek kecil, \textit{read}-heavy, latensi ketat.
	\item Hybrid: replikasi untuk data kecil/sering dibaca; EC untuk data besar/arsip.
\end{itemize}

%----------------------------------------------------------------------------------------
%   KETERBATASAN
%----------------------------------------------------------------------------------------

\postersection{Keterbatasan dan Pekerjaan Lanjutan}
\begin{itemize}
	\item Keanggotaan kluster statis; lingkungan uji lokal dengan \textit{traffic shaping}.
	\item Parameter EC tetap; tidak mengevaluasi pemulihan kegagalan/rekonfigurasi.
\end{itemize}
\begin{itemize}
	\item Penalaan/parallelisasi EC, model hibrida EC+replikasi, uji lintas \textit{datacenter}, dan skenario nyata berjarak.
\end{itemize}

%----------------------------------------------------------------------------------------
%   UCAPAN TERIMA KASIH
%----------------------------------------------------------------------------------------

\postersection{Ucapan Terima Kasih}
\begin{itemize}
	\item Pembimbing: Achmad Imam Kistijantoro, S.T., M.Sc., Ph.D.
	\item Fakultas Teknik Informatika ITB, Distributed Systems Research Group.
	\item Komunitas sumber terbuka (Rust, RocksDB, OmniPaxos).
\end{itemize}

\vspace{0.1cm}
			\textbf{Repo Implementasi:}
			
			\url{github.com/MuhamadAjiW/DistKV-Erasure-Coding}

% Place the two write-related single-column figures at the end of the poster
% so they render after all the textual content and do not overlap.
\vspace{0.02cm}
\onecolimg{write_bigload_avgnet_heatmap.png}{Heatmap \textit{Write}: Internet Menengah, Payload Besar}
\vspace{0.02cm}
\onecolimg{write_bigload_avgnet_boundary.png}{Batas Kinerja EC vs Replikasi (Model Regresi)}
\end{multicols}
\end{document}