%%%%%%%%%%%%%%%%%%%%%%%%%%%%%%%%%%%%%%%%%
% a0poster Landscape Poster - REVAMPED STYLE
% Original Template from: http://www.LaTeXTemplates.com
%%%%%%%%%%%%%%%%%%%%%%%%%%%%%%%%%%%%%%%%%

%----------------------------------------------------------------------------------------
%   PACKAGES AND OTHER DOCUMENT CONFIGURATIONS
%----------------------------------------------------------------------------------------

\documentclass[a2,portrait]{config/poster/a0poster}
\usepackage{config/poster/a0size}

% Core Packages
\usepackage{multicol} % For multiple columns
\usepackage[svgnames]{xcolor} % For custom colors
\usepackage{times} % Use the times font
\usepackage{graphicx} % For including images
\usepackage{booktabs} % For professional tables
\usepackage[font=small,labelfont=bf]{caption} % For figure/table captions
\usepackage{amsfonts, amsmath, amsthm, amssymb} % Math packages
\usepackage[style=ieee]{biblatex} % IEEE style citations
\usepackage{tikz} % For creating diagrams
\usepackage{pgfplots} % For creating plots
\pgfplotsset{compat=1.18}
\usepackage{enumitem} % compact lists

% Styling Packages
\usepackage[most]{tcolorbox} % For creating colored boxes (headers)

\hyphenpenalty=100000
\tolerance=10000

%----------------------------------------------------------------------------------------
%   DOCUMENT CONFIGURATION
%----------------------------------------------------------------------------------------

% Layout settings
\columnsep=8pt % Space between columns (suitable for a 2-column poster)

% Reduce vertical spacing for captions so figures use less vertical real-estate
\setlength{\abovecaptionskip}{8pt}
\setlength{\belowcaptionskip}{4pt}

% Set graphic path
\graphicspath{{./}{resources/chapter-4/}} % Location of the graphics files

% Bibliography resources
\addbibresource{config/paper/IEEEabrv.bib}
\addbibresource{references.bib}

% Custom Colors and Commands
\definecolor{ITBblue}{rgb}{0.0, 0.21, 0.42} % ITB blue branding
\definecolor{ECcolor}{rgb}{0.8, 0.2, 0.2} % Erasure coding color
\definecolor{REPcolor}{rgb}{0.2, 0.6, 0.2} % Replication color

% Command for creating a styled section header
\newcommand{\postersection}[1]{%
	\begin{tcolorbox}[
			colback=ITBblue,
			colframe=ITBblue,
			fonttitle=\bfseries,
			coltext=white,
			sharp corners,
			boxrule=0pt,
			top=0pt,
			bottom=0pt,
			halign=center
		]
		\normalsize #1
	\end{tcolorbox}%
}

%----------------------------------------------------------------------------------------

\begin{document}

%----------------------------------------------------------------------------------------
%   POSTER HEADER 
%----------------------------------------------------------------------------------------

% Make header span the full text width so the multicols environment
% computes the column widths against the full page and not an inset
\begin{minipage}[c]{\linewidth}
\Huge \textbf{Analisis Erasure Coding terhadap Replikasi pada Key-Value Store} \\
\large \textit{Erasure Coding Performance Analysis Against Replication} \\
\normalsize \textbf{Muhamad Aji Wibisono (13521095)} \\
\normalsize Teknik Informatika, Institut Teknologi Bandung \\
\end{minipage}

\vspace{0.05cm} % Whitespace between header and content (further reduced)

%----------------------------------------------------------------------------------------
%   POSTER BODY
%----------------------------------------------------------------------------------------

\begin{multicols}{2} % Use 2 columns for the body
\vspace{0.1cm}

% Helper to insert an image that spans two columns in multicols
% Use 2*\columnwidth + \columnsep so the image occupies two of the multicols columns
\newcommand{\twocolimg}[2]{%
	\begin{center}
		\parbox{\dimexpr2\columnwidth+\columnsep\relax}{%
			\centering
			% Use full width of the 2-column box and keep aspect ratio
			\includegraphics[width=\linewidth,keepaspectratio]{#1}
			\captionof{figure}{#2}
		}
	\end{center}
}

	% One-column image helper: image will occupy a single multicols column (width=\columnwidth)
	\newcommand{\onecolimg}[2]{%
		\begin{center}
			\parbox{\columnwidth}{%
				\centering
				\includegraphics[width=\linewidth,keepaspectratio]{#1}
				\captionof{figure}{#2}
			}
		\end{center}
	}

			% The two large write-related figures were moved to the end of the poster
			% (to avoid overlapping flowing text in the multicols layout)
	% (temporary TEST-GRID removed) -- layout testing complete

	%----------------------------------------------------------------------------------------
	%   ABSTRACT
	%----------------------------------------------------------------------------------------

	\postersection{Abstract}
	\begin{quote}
		This research analyzes the performance of erasure coding compared to replication in distributed key-value store database systems, specifically focusing on system response time for write and read operations. With growing data storage needs and intensive computing requirements, efficient data redundancy solutions become increasingly important. Erasure coding offers a solution to improve system resilience with better storage efficiency compared to traditional replication.
	\end{quote}

	%----------------------------------------------------------------------------------------
	%   PROBLEM STATEMENT
	%----------------------------------------------------------------------------------------

	\postersection{Problem Statement}
	
	\textbf{Research Questions:}
	\begin{itemize}
		\item How does erasure coding performance compare to replication in distributed key-value stores?
		\item Under what conditions does erasure coding outperform replication?
		\item What are the trade-offs between storage efficiency and response time?
	\end{itemize}

	\textbf{Challenges:}
	\begin{itemize}
		\item Network bandwidth limitations
		\item Data reconstruction overhead
		\item Consensus protocol complexity
		\item Variable payload sizes
	\end{itemize}

	%----------------------------------------------------------------------------------------
	%   METHODOLOGY
	%----------------------------------------------------------------------------------------

	\postersection{Methodology}
	
	\textbf{System Architecture:}
	\begin{itemize}
		\item \textbf{Language:} Rust
		\item \textbf{Storage:} RocksDB
		\item \textbf{Algorithm:} Reed-Solomon Erasure Coding
		\item \textbf{Consensus:} OmniPaxos
		\item \textbf{Benchmarking:} k6 + mpstat
		\item \textbf{Environment:} VM with traffic control
	\end{itemize}

	\textbf{Test Parameters:}
	\begin{itemize}
		\item Network bandwidth variations
		\item Payload size variations (1KB - 1MB)
		\item Different erasure coding configurations
		\item Read vs Write operation analysis
	\end{itemize}

	%----------------------------------------------------------------------------------------
	%   WRITE PERFORMANCE RESULTS
	%----------------------------------------------------------------------------------------

	\postersection{Write Performance Results}
	
	\textbf{Key Findings:}
	\begin{itemize}
		\item \textcolor{ECcolor}{\textbf{Erasure Coding}} shows threshold performance
		\item Better performance with limited bandwidth + large payloads
		\item \textcolor{REPcolor}{\textbf{Replication}} excels with small data + high bandwidth
	\end{itemize}

	% removed manual \columnbreak to allow multicols to balance columns automatically

	%----------------------------------------------------------------------------------------
	%   PERFORMANCE COMPARISON
	%----------------------------------------------------------------------------------------

	\postersection{Performance Threshold Analysis}
	
	\textbf{Threshold Conditions:}
	\begin{itemize}
		\item Network bandwidth: < 100 Mbps
		\item Payload size: > 100 KB
		\item Write-heavy workloads
		\item High network contention scenarios
	\end{itemize}

	%----------------------------------------------------------------------------------------
	%   READ PERFORMANCE RESULTS
	%----------------------------------------------------------------------------------------

	\postersection{Read Performance Results}
	
			% Read operation analysis kept, image removed to focus poster on write-related figures

				extbf{Read Operation Analysis:}
	\begin{itemize}
		\item \textcolor{REPcolor}{\textbf{Replication consistently outperforms}} erasure coding
		\item Data reconstruction overhead significant
		\item No threshold found for read operations
		\item Direct data access vs reconstruction complexity
	\end{itemize}

	%----------------------------------------------------------------------------------------
	%   PERFORMANCE METRICS
	%----------------------------------------------------------------------------------------

	\postersection{Performance Metrics Comparison}
	
	\begin{center}
		\begin{tabular}{l c c c}
			\toprule
			\textbf{Metric} & \textbf{Erasure Coding} & \textbf{Replication} & \textbf{Improvement} \\
			\midrule
			Storage Efficiency & 1.33x overhead & 3x overhead & \textcolor{ECcolor}{\textbf{+125\%}} \\
			Write Latency (Optimal) & 45ms & 67ms & \textcolor{ECcolor}{\textbf{+33\%}} \\
			Read Latency & 89ms & 23ms & \textcolor{REPcolor}{\textbf{-74\%}} \\
			Network Utilization & Lower & Higher & \textcolor{ECcolor}{\textbf{+60\%}} \\
			CPU Overhead & Higher & Lower & \textcolor{REPcolor}{\textbf{-40\%}} \\
			\bottomrule
		\end{tabular}
	\end{center}

	%----------------------------------------------------------------------------------------
	%   NETWORK CONDITIONS IMPACT
	%----------------------------------------------------------------------------------------

	\postersection{Network Conditions Impact}
	
	% Network impact discussion retained; image removed to keep poster focused on the two write-related figures

			extbf{Network Impact Analysis:}
	\begin{itemize}
		\item Slow networks favor \textcolor{ECcolor}{\textbf{erasure coding}}
		\item Fast networks favor \textcolor{REPcolor}{\textbf{replication}}
		\item Bandwidth utilization critical factor
		\item Latency vs throughput trade-offs
	\end{itemize}

	%----------------------------------------------------------------------------------------
	%   CONCLUSIONS
	%----------------------------------------------------------------------------------------

	\postersection{Key Conclusions}
	
	\begin{enumerate}
		\item \textbf{Threshold Performance:} Erasure coding has performance threshold for write operations when bandwidth is limited and payload is large
		
		\item \textbf{Read Operations:} Replication consistently outperforms erasure coding due to reconstruction overhead
		
		\item \textbf{Use Case Suitability:} Erasure coding not suitable for small-data key-value stores in high-capacity data centers
		
		\item \textbf{Storage vs Performance:} Clear trade-off between storage efficiency and response time performance
	\end{enumerate}

	\textbf{Recommendations:}
	\begin{itemize}
		\item Use \textcolor{ECcolor}{\textbf{erasure coding}} for: Large data, limited bandwidth, write-heavy workloads
		\item Use \textcolor{REPcolor}{\textbf{replication}} for: Small data, high bandwidth, read-heavy workloads
		\item Consider hybrid approaches for mixed workloads
	\end{itemize}

	%----------------------------------------------------------------------------------------
	%   FUTURE WORK
	%----------------------------------------------------------------------------------------

	\postersection{Future Work \& Limitations}
	
	\textbf{Current Limitations:}
	\begin{itemize}
		\item Static cluster membership configuration
		\item Limited to read/write operations only
		\item Local testing environment with simulation
		\item Fixed erasure coding parameters
	\end{itemize}

	\textbf{Future Research Directions:}
	\begin{itemize}
		\item Dynamic erasure coding parameter adjustment
		\item Hybrid redundancy mechanisms
		\item Real-world deployment testing
		\item Advanced consensus protocols integration
		\item Multi-datacenter scenarios
	\end{itemize}

	%----------------------------------------------------------------------------------------
	%   ACKNOWLEDGEMENTS
	%----------------------------------------------------------------------------------------

	\postersection{Acknowledgements}
	
	Special thanks to:
	\begin{itemize}
		\item Achmad Imam Kistijantoro, S.T, M.Sc., Ph.D. (Supervisor)
		\item Faculty of Informatics Engineering, ITB
		\item Distributed Systems Research Group
		\item Open source community (Rust, RocksDB, OmniPaxos)
	\end{itemize}

	\vspace{0.5cm}
	\textbf{Repository:} \texttt{github.com/MuhamadAjiW/DistKV-Erasure-Coding}

	% Place the two write-related single-column figures at the end of the poster
	% so they render after all the textual content and do not overlap.
	\vspace{0.2cm}
	\onecolimg{write_bigload_avgnet_heatmap.png}{Write Response Time Heatmap - Big Load, Average Network}
	\vspace{0.2cm}
	\onecolimg{write_bigload_avgnet_boundary.png}{Performance Boundary Between EC and Replication}

\end{multicols}
\end{document}