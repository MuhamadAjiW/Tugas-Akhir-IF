\chapter{Pemetaan Pengujian}
\label{appendix:pemetaan-pengujian}

Lampiran ini menyediakan tabel yang memetakan kebutuhan fungsional dan non-fungsional ke pengujian yang dilakukan. Tujuannya adalah memudahkan pemetaan antara kebutuhan sistem dan pengujian yang dilakukan serta memastikan bahwa semua kebutuhan telah diuji.

\begin{longtable}{|p{2cm}|p{2cm}|p{3cm}|p{3cm}|p{2cm}|}
\caption{Pemetaan Pengujian}
\label{tab:pemetaan-pengujian} \\
\hline
\rowcolor{black!10} ID Kebutuhan & ID Pengujian & Skenario & Ekspektasi & Realita \\ \hline
\endfirsthead

\caption[]{Pemetaan Pengujian (lanjutan)} \\
\hline
\rowcolor{black!10} ID Kebutuhan & ID Pengujian & Skenario & Ekspektasi & Realita \\ \hline
\endhead

F-1 & P-1 & Pengujian operasi \textit{write} pada sistem & Sistem dapat menyimpan \textit{key-value pair} menggunakan operasi \textit{write} & Sesuai \\ \hline
F-1 & P-2 & Pengujian operasi \textit{read} pada sistem & Nilai \textit{value} dapat didapatkan menggunakan operasi \textit{read} & Sesuai \\ \hline
F-2 & P-3 & Pengujian penyimpanan data \textit{persistent} & Sistem dapat menyimpan data secara \textit{persistent} & Sesuai \\ \hline
F-3 & P-4 & Pengujian pencatatan waktu transaksi pada operasi \textit{write} & Sistem mencatat waktu transaksi dari \textit{request} masuk hingga operasi selesai pada operasi \textit{write} & Sesuai \\ \hline
F-3 & P-5 & Pengujian pencatatan waktu transaksi pada operasi \textit{read} & Sistem mencatat waktu transaksi dari \textit{request} masuk hingga operasi selesai pada operasi \textit{read} & Sesuai \\ \hline
F-4 & P-6 & Pengujian encoding data menggunakan erasure coding & Sistem dapat melakukan \textit{erasure coding} pada data yang disimpan & Sesuai \\ \hline
F-5 & P-7 & Pengujian rekonstruksi data & Sistem dapat merekonstruksi data dari data yang disimpan menggunakan \textit{erasure coding} & Sesuai \\ \hline
F-6 & P-8 & Pengujian distribusi data ke \textit{node} lain & Sistem dapat mendistribusikan data atau sebagian data ke \textit{node} lain untuk keperluan ketahanan data & Sesuai \\ \hline
F-7 & P-9 & Pengujian konfigurasi sistem untuk replikasi & Sistem dapat dikonfigurasi untuk menggunakan replikasi tanpa mengganti konfigurasi lainnya & Sesuai \\ \hline
F-8 & P-10 & Pengujian konfigurasi sistem untuk tingkat ketahanan & Sistem dapat dikonfigurasi untuk mencapai tingkat ketahanan tertentu tanpa mengganti konfigurasi lainnya & Sesuai \\ \hline
F-9 & P-11 & Pengujian request dengan ukuran data bervariasi & Sistem dapat mensimulasikan \textit{request} dengan ukuran data yang bervariasi & Sesuai \\ \hline
F-10 & P-12 & Pengujian pengumpulan data \textit{benchmark} & Sistem dapat menjalankan \textit{request} secara berulang kali dan bervariasi secara otomatis untuk pengumpulan data & Sesuai \\ \hline
NF-1 & P-13 & Pengujian konsistensi data & Sistem menyediakan \textit{consistency} yang tinggi dengan \textit{request} ke \textit{node} manapun menghasilkan hasil yang sama & Tidak dapat diuji dengan pasti \\ \hline
NF-2 & P-14 & Pengujian ketersediaan sistem & Sistem memiliki \textit{availability} yang tinggi dengan dapat tetap tersedia walaupun beberapa \textit{node} ada dalam kondisi gagal & Tidak dapat diuji dengan pasti \\ \hline
NF-3 & P-15 & Pengujian efisiensi penyimpanan & Sistem menggunakan penyimpanan minimal untuk skalabilitas dan efisiensi biaya & Tidak dapat diuji dengan pasti \\ \hline
NF-4 & P-16 & Pengujian waktu respons & Sistem menyediakan \textit{response time} rendah untuk operasi \textit{read} dan \textit{write} & Tidak dapat diuji dengan pasti \\ \hline
\end{longtable}