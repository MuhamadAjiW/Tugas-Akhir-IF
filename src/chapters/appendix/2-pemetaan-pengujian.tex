\chapter{Pemetaan Pengujian}
\label{appendix:pemetaan-pengujian}

Lampiran ini menyediakan tabel yang memetakan kebutuhan fungsional dan non-fungsional ke pengujian yang dilakukan. Tujuannya adalah memudahkan pemetaan antara kebutuhan sistem dan pengujian yang dilakukan serta memastikan bahwa semua kebutuhan telah diuji.

\begin{longtable}{|l|l|p{3cm}|p{3cm}|l|}
\caption{Pemetaan Pengujian}
\label{tab:pemetaan-pengujian} \\
\hline
\rowcolor{black!10} ID Kebutuhan & ID Pengujian & Skenario & Ekspektasi & Realita \\ \hline
\endfirsthead

\caption[]{Pemetaan Pengujian (lanjutan)} \\
\hline
\rowcolor{black!10} ID Kebutuhan & ID Pengujian & Skenario & Ekspektasi & Realita \\ \hline
\endhead

F-1 & P-1 & Pengujian operasi \textit{write} pada sistem & Sistem dapat menyimpan \textit{key-value pair} menggunakan operasi \textit{write} & Sesuai \\ \hline
F-1 & P-2 & Pengujian operasi \textit{read} pada sistem & Nilai \textit{value} dapat didapatkan menggunakan operasi \textit{read} & Sesuai \\ \hline
F-2 & P-3 & Pengujian penyimpanan data \textit{persistent} & Sistem dapat menyimpan data secara \textit{persistent} & Sesuai \\ \hline
\end{longtable}
