\chapter{Perbandingan Distribusi Kubernetes}

\bgroup
\begin{table}[ht]
  \def\arraystretch{1.5}
  \caption{Perbandingan Tiga Distribusi Kubenernetes pada Lingkungan IoT}
  \label{tab:perbandingan-distribusi-kubernetes}
  \centering
  \begin{tabular}{|p{1.7cm}|p{1.7cm}|p{2.3cm}|p{2cm}|p{1.5cm}|p{2cm}|}
    \hline
    \centering{Nama} & \centering{Dukungan arsitektur} & \centering{Instalasi dan Penggunaan}                                                                                                                           & \centering{Komputasi \textit{resource}}                  & Ukuran Cluster                                      & Fitur                            \\
    \hline
    K8s              & aarch64, armv7l, x86-64, macOS  & Sangat mudah, dapat diinstal dengan satu line command, dokumentasi lengkap                                                                                     & Penggunaan \textit{resource} yang minimal                & Kecil hingga menengah                               & Tidak seluruh fitur k8s tersedia \\
    \hline
    Microk8s         & aarch64, x86-64, armv7l         & Instalasi cukup sulit untuk dilakukan, kustomisasi cukup sulit untuk dilakukan, dokumentasi cukup sulit untuk dimeengerti                                      & Memerlukan \textit{resource} yang cukup besar            & Hanya untuk \textit{cluster} lokal                  & Tidak seluruh fitur k8s tersedia \\
    \hline
    KubeEdge         & aarch64, x86-64, armv7l         & Instalasi cukup kompleks serta perlu pemahaman yang lebih untuk menggunakannya. \textit{kustomisasi} yang cukup sulit, namun memiliki dokumentasi yang lengkap & Memerlukan sumber daya yang cukup besar dibandingkan K8s & Cocok untuk lingkungan distribusi serta skala besar & Lengkap                          \\
    \hline
  \end{tabular}
\end{table}
\egroup