\chapter{Perbandingan Rancangan Implementasi}
\label{appendix:tools}

Sebelum memilih kakas Memcached sebagai \textit{in-memory key-value store} dan RocksDB sebagai \textit{persistent database} dalam satu \textit{node}, terdapat beberapa pertimbangan dalam pemilihan arsitektur serta kakas dalam rancangan implementasi

\section{Perbandingan Pengembangan Sistem}

Terdapat beberapa solusi yang dipertimbangkan dalam pembangunan sistem, yaitu membangun sistem dengan 

\subsection{Mengembangkan sistem secara modular dengan in-memory key-value store dan persistent storage terpisah}
Pendekatan ini mengkombinasikan antara sistem \textit{in-memory key-value store} dengan menggunakan \textit{database} terpisah sebagai tempat penyimpanan data \textit{persistence}. Replikasi dan \textit{erasure coding} dapat dilakukan pada data persisten yang disimpan pada \textit{database} tersebut. Kelebihan dari pendekatan ini adalah kemudahan dari implementasi dibandingkan pengembangan sendiri ataupun modifikasi \textit{database} yang sudah lebih kompleks. Namun, kekurangannya adalah \textit{overhead} penggunaan memori dari menjalankan \textit{in-memory key-value store} dan \textit{database} terpisah. Dari ketiga alternatif solusi, kemudahan implementasi dan kekurangan yang dapat ditoleransi menjadikan alternatif ini solusi yang dipilih dalam penelitian ini.

\subsection{Mengembangkan sistem mock database}
Pendekatan ini membangun semua fitur-fitur yang dibutuhkan secara mandiri. Kelebihan dari pendekatan ini kebebasan yang tinggi sehingga sistem dapat disesuaikan untuk eksperimen. Namun, pendekatan ini sulit mencerminkan kondisi nyata. Selain itu, kinerja sistem yang dikembangkan juga bergantung erat pada waktu serta kemampuan pengembangan peneliti.

\subsection{Mengembangkan sistem yang sudah memiliki fitur in memory database dan persistent storage}
Pengembangan ini membangun mengganti fitur yang dibutuhkan untuk penelitian di atas \textit{database} lebih kompleks dari Memcached. Pendekatan ini memiliki kelebihan kedekatan kondisi eksperimen dengan dunia nyata. Namun, pendekatan ini sulit dilakukan karena membutuhkan waktu yang lama untuk memahami dan mengubah \textit{source-code database} yang kompleks.

\section{Perbandingan Kakas}
Setelah menentukan solusi pengembangan sistem secara modular, terdapat beberapa kakas yang dapat dipilih untuk solusi tersebut. Disebabkan pemilihan solusi modular, kakas yang dipertimbangkan untuk digunakan meliputi kakas \textit{in-memory key-value store} dan kakas untuk \textit{persistent database}.

\subsection{In-memory key-value store}
Memcached dipilih sebagai \textit{in-memory key-value store}. Selain Memcached, kakas yang dipertimbangkan untuk digunakan sebagai \textit{in-memory key-value store} adalah seperti yang ada pada tabel \ref{tab:tools-kv} \parencite{redis_docs} \parencite{dragonflydb_docs} \parencite{memcached_documentation}.

\begin{table}[h]
    \centering
    \caption{Perbandingan kakas in-memory key-value store}
    \resizebox{\textwidth}{!}{
        \begin{tabular}{|l|p{5cm}|p{5cm}|p{3cm}|}
            \hline
            \rowcolor{black!10} Kakas & Kelebihan & Kekurangan & Notes \\ \hline
            Redis & +Support struktur data kompleks \newline +Support transaksi kompleks & -Single Threaded \newline -Kompleks & Ada support untuk replikasi \\ \hline
            DragonflyDB & +Multithreaded \newline +Support struktur data kompleks \newline +Support transaksi kompleks & -Lebih kompleks dari Memcached, lebih simpel dari Redis \newline -Relatif baru, komunitas terbatas dibanding kedua alternatif & Rilis 2023, Ada support untuk replikasi\\ \hline
            Memcached & +Simpel \newline +Multithreaded & -Struktur data terbatas \newline -Transaksi terbatas & Sangat minimalis \\ \hline
        \end{tabular}
    }
    \label{tab:tools-kv}
\end{table}

Dari kakas-kakas tersebut, seperti yang sudah dijelaskan pada bagian \ref{chapter:rencana-pelaksanaan}, dipilih Memcached sebagai kakas untuk eksperimen ini. Alasan utama pemilihan Memcached adalah sifatnya yang paling simpel, kinerja \textit{multithreaded}-nya, serta tidak dibutuhkannya fitur-fitur yang dimiliki Redis dan Dragonfly.

\subsection{Persistent database}
RocksDB dipilih sebagai \textit{persistent storage} karena berfokus pada kinerja \textit{write} yang tinggi. Penggunaan RocksDB juga dilakukan seperti \textit{library} sehingga memudahkan pengembangan sistem yang modular. Selain RocksDB, kakas yang dipertimbangkan untuk digunakan sebagai \textit{persistent database} adalah seperti yang ada pada tabel \ref{tab:tools-db} \parencite{cassandra_docs} \parencite{scylladb_docs} \parencite{rocksdb_docs} \parencite{leveldb_docs} \parencite{mongodb_docs}.

\begin{table}[h]
    \centering
    \caption{Perbandingan kakas persistent database}
    \resizebox{\textwidth}{!}{
        \begin{tabular}{|l|p{5cm}|p{5cm}|p{3cm}|}
            \hline
            \rowcolor{black!10} Kakas & Kelebihan & Kekurangan & Notes \\ \hline
            Cassandra & +Horizontal Scalability baik \newline +Sudah ada referensi implementasi erasure coding pada paper lain & -Eventual Consistency \newline -Latensi tinggi untuk operasi write \newline -Sangat Kompleks & Utamanya Wide-Column Store\\ \hline
            ScyllaDB & +Lebih efisien dan performant dibandingkan Cassandra \newline +Memiliki pilihan strong consistency & -Lebih kompleks dibandingkan Cassandra \newline -Ekosistem terbatas & Mirip dan kompatibel dengan Cassandra. \\ \hline
            RocksDB & +Kinerja tinggi untuk operasi write \newline +Recovery menggunakan write ahead log & -Single Node \newline -Keterbatasan pada concurrent writes & Pengembangan LevelDB\\ \hline
            LevelDB & +Simpel \newline +Penggunaan memori kecil \newline +Kinerja tinggi untuk operasi write & -Fitur terbatas seperti compression dan data terbatas pada string \newline -Operasi write skala besar masih lebih lambat dibandingkan RocksDB \newline -Keterbatasan pada concurrent writes & Dasar dari RocksDB\\ \hline
            MongoDB & +Support query kompleks \newline +Penggunaan fleksibel & -Operasi kurang optimal \newline -Konsumsi memori besar \newline -Storage overhead besar & Utamanya Dokumen database\\ \hline
        \end{tabular}
    }
    \label{tab:tools-db}
\end{table}

Dari kakas-kakas tersebut, seperti yang sudah dijelaskan pada bagian \ref{chapter:rencana-pelaksanaan}, dipilih RocksDB sebagai kakas untuk eksperimen ini. Selain RocksDB, LevelDB juga merupakan kandidat yang kuat untuk eksperimen ini karena sifatnya yang simpel. Namun, pada akhirnya, RocksDB dipilih karena pengembangannya yang masih aktif, kinerja \textit{write} yang lebih tinggi, serta fitur-fitur tambahan seperti \textit{compression} dan penyimpanan data dalam \textit{byte}.