\section{Latar Belakang}
\label{sec:latar-belakang}

Seiring berkembangnya teknologi komputasi, penyimpanan yang digunakan untuk data komputer terus berkembang hingga nilai yang tidak terbayangkan. Kapasitas tiap perangkat penyimpanan berkembang dalam laju 175 persen per tahunnya \parencite{kelechava2016storage}. Pengembangan penggunaan data dan penyimpanan ini didorong dengan munculnya teknologi \textit{cloud computing}, kebutuhan data untuk pelatihan \textit{artificial intelligence} berbasis \textit{machine learning}, dan peningkatan kualitas data secara umum pada \textit{file} dan teknologi yang digunakan saat ini.

% Salah satu teknologi yang semakin banyak digunakan untuk pengelolaan data berskala besar adalah \textit{key-value database}. Jenis \textit{database} ini menawarkan arsitektur yang sederhana namun sangat efektif untuk menyimpan data. \textit{Key-value database} memungkinkan akses data yang sangat cepat dan skala yang besar sehingga cocok digunakan dalam sistem yang memerlukan kinerja tinggi dan \textit{response time} rendah \parencite{chen2016towards}.

Dengan semakin intensifnya sistem-sistem yang muncul digunakan, diperlukan juga ketahanan tinggi agar sistem-sistem tersebut tetap operasional, terutama dalam menghadapi kemungkinan kegagalan komponen dan kehilangan data \parencite{weatherspoon2002erasure}. Dalam hal ini, solusi redundansi data menjadi sangat penting. Secara tradisional, teknik seperti replikasi digunakan untuk menduplikasi data ke beberapa server atau lokasi. Pendekatan ini memakan kapasitas penyimpanan yang besar dan meningkatkan biaya operasional secara signifikan.

\textit{Erasure coding} menawarkan solusi untuk meningkatkan ketahanan dari sistem dengan penambahan penyimpanan yang lebih efisien. Dengan penerapan \textit{erasure coding}, biaya penyimpanan dengan tetap menjaga integritas dan ketahanan data, terutama dalam lingkungan sistem yang terdistribusi menggunakan beberapa perangkat sekaligus \parencite{balaji2018erasure}. Namun, \textit{erasure coding} membutuhkan sumber daya komputasi dalam penerapannya. Hal ini menyebabkan tingginya latensi dan \textit{response time} dari layanan yang dibuat. Padahal, banyak layanan aplikasi yang menjadikan latensi rendah sebagai syarat dalam operasinya \parencite{dean2013tail}.

Menurunnya tingkat penyimpanan yang digunakan untuk mencapai ketahanan yang sama dibandingkan replikasi akan menurunkan ukuran data yang dikirim ke \textit{node} lain. Dengan demikian, \textit{erasure coding} berpotensi memiliki \textit{response time} yang lebih rendah pada operasi \textit{write} yang dilakukan. Masalah muncul ketika membutuhkan operasi \textit{read} yang bergantung erat ke implementasi sistem \textit{erasure coding} yang digunakan. Pada \textit{erasure coding} yang menyebarkan masing-masing entri data pada \textit{node} berbeda, operasi \textit{read} akan membutuhkan agregasi data dari semua \textit{node} yang dibutuhkan sehingga menambah \textit{response time}. Akan tetapi, agregasi yang dilakukan juga dapat dilakukan untuk mendapatkan \textit{majority read} dari kuorum sistem terdistribusi untuk konsistensi yang tinggi.

Dinamika yang dimiliki \textit{erasure coding} pada sistem \textit{database} terdistribusi berpotensi menghasilkan suatu sistem yang berfokus untuk write dan memiliki konsistensi tinggi. Penelitian ini berfokus untuk melakukan eksplorasi dari dinamika tersebut untuk pemanfaatannya pada sistem penyimpanan terdistribusi. Dalam penelitian ini, akan dirancang dan dianalisis sistem \textit{database} terdistribusi yang memanfaatkan \textit{erasure coding}. Evaluasi ini diharapkan dapat menghadirkan \textit{use case} baru untuk penerapan \textit{erasure coding} dalam konteks penyimpanan data yang lebih efisien. Selain itu, dari \textit{use case} yang dihasilkan, diharapkan dapat dikembangkan sebuah sistem penyimpanan efisien yang sesuai.

% mengevaluasi peningkatan \textit{response time} yang disebabkan oleh proses komputasi \textit{erasure coding} dalam sistem terdistribusi

% \textit{hot storage}, di mana data sering diakses dan membutuhkan waktu respons yang cepat
