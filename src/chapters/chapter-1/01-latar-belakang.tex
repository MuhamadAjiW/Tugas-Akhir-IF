\section{Latar Belakang}
\label{sec:latar-belakang}

Seiring berkembangnya teknologi komputasi, penyimpanan untuk data komputer terus berkembang hingga nilai yang tinggi. Pengembangan penggunaan data dan penyimpanan ini didorong dengan munculnya teknologi \textit{cloud computing}, kebutuhan data untuk pelatihan \textit{artificial intelligence} berbasis \textit{machine learning}, dan peningkatan kualitas data secara umum pada \textit{file}.

Bersamaan dengan hal tersebut, semakin intensif juga penggunaan sistem-sistem yang digunakan. Penggunaan intensif ini menyebabkan kebutuhan untuk ketahanan tinggi agar sistem-sistem tersebut dapat tetap beroperasi, terutama dalam menghadapi kemungkinan kegagalan komponen dan kehilangan data \parencite{weatherspoon2002erasure}. Dalam hal ini, solusi redundansi data menjadi sangat penting. Secara tradisional, teknik replikasi digunakan untuk menduplikasi data ke beberapa \textit{node}. Namun, dengan berkembangnya kebutuhan untuk data dalam operasional, pendekatan ini memakan kapasitas penyimpanan yang semakin besar dan meningkatkan biaya operasional secara signifikan.

Solusi lain untuk menghadapi kegagalan ini adalah \textit{erasure coding}. Dengan penerapan \textit{erasure coding}, kebutuhan akan penyimpanan data dapat dikurangi dengan tetap menjaga integritas dan ketahanan data, terutama dalam lingkungan sistem yang terdistribusi menggunakan beberapa perangkat sekaligus \parencite{balaji2018erasure}. Namun, \textit{erasure coding} membutuhkan sumber daya komputasi yang lebih tinggi dibandingkan replikasi dalam penerapannya. Hal ini menyebabkan tingginya latensi dan \textit{response time} dari layanan yang dibuat. Padahal, banyak layanan aplikasi yang menjadikan latensi rendah sebagai syarat dalam operasinya \parencite{dean2013tail}.

Akan tetapi, di samping penambahan komputasi, penerapan \textit{erasure coding} pada sebuah sistem mengurangi ukuran data keseluruhan untuk menyediakan integritas dan ketahanan data. Pengurangan ukuran data dapat menyebabkan turunnya juga ukuran data yang dikirim ke \textit{node} lain. Dengan demikian, \textit{erasure coding} berpotensi memiliki kondisi ketika ukuran data cukup besar dan jaringan cukup lambat hingga \textit{response time} lebih rendah pada operasi tertentu dibandingkan melakukan operasi yang sama pada sistem \textit{replikasi}. Kondisi ini dapat terjadi khususnya pada operasi \textit{write}. Operasi \textit{read} tidak ideal karena memerlukan rekonstruksi data yang memerlukan pengiriman data dari node lainnya sedangkan replikasi dapat langsung mengembalikan data yang tersimpan.

Dinamika \textit{erasure coding} pada sistem \textit{database} terdistribusi berpotensi memberikan kinerja yang berbeda dibandingkan sistem replikasi tradisional. Penelitian ini berfokus untuk melakukan eksplorasi dari dinamika tersebut untuk pemanfaatannya pada sistem distributed \textit{key-value store database}. Dalam penelitian ini, akan dirancang dan dianalisis \textit{response time} sistem \textit{database} terdistribusi yang memanfaatkan \textit{erasure coding} berdasarkan variabel berupa tingkat ukuran data dan kecepatan jaringan. Analisis tersebut diharapkan dapat menemukan kondisi ketika response time operasi pada sistem berbasis \textit{erasure coding} lebih cepat dibandingkan sistem yang menggunakan replikasi. Selain itu, dari kondisi yang ditemukan, diharapkan dapat dipelajari dan dimanfaatkan ke dalam sebuah implementasi sistem \textit{database} terdistribusi.
