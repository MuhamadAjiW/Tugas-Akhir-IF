\section{Latar Belakang}
\label{sec:latar-belakang}

Seiring berkembangnya teknologi komputasi, penyimpanan untuk data komputer terus berkembang hingga nilai yang tinggi. Pengembangan penggunaan data dan penyimpanan ini didorong dengan munculnya teknologi \textit{cloud computing}, kebutuhan data untuk pelatihan \textit{artificial intelligence} berbasis \textit{machine learning}, dan peningkatan kualitas data secara umum pada \textit{file}.

Bersamaan dengan hal tersebut, semakin intensif juga penggunaan sistem-sistem yang digunakan. Penggunaan intensif ini menyebabkan kebutuhan untuk ketahanan tinggi agar sistem-sistem tersebut dapat tetap beroperasi, terutama dalam menghadapi kemungkinan kegagalan komponen dan kehilangan data \parencite{weatherspoon2002erasure}. Dalam hal ini, solusi redundansi data menjadi sangat penting. Secara tradisional, teknik seperti replikasi digunakan untuk menduplikasi data ke beberapa server atau lokasi. Namun, dengan berkembangnya kebutuhan untuk data dalam operasional, pendekatan ini memakan kapasitas penyimpanan yang semakin besar dan meningkatkan biaya operasional secara signifikan.

\textit{Erasure coding} menawarkan solusi untuk meningkatkan ketahanan dari sistem dengan penambahan penyimpanan yang lebih efisien. Dengan penerapan \textit{erasure coding}, kebutuhan akan penyimpanan data dapat dikurangi dengan tetap menjaga integritas dan ketahanan data, terutama dalam lingkungan sistem yang terdistribusi menggunakan beberapa perangkat sekaligus \parencite{balaji2018erasure}. Namun, \textit{erasure coding} membutuhkan sumber daya komputasi yang lebih tinggi dibandingkan replikasi dalam penerapannya. Hal ini menyebabkan tingginya latensi dan \textit{response time} dari layanan yang dibuat. Padahal, banyak layanan aplikasi yang menjadikan latensi rendah sebagai syarat dalam operasinya \parencite{dean2013tail}.

Akan tetapi, di samping komputasi yang diperlukan, penerapan \textit{erasure coding} pada sebuah sistem akan mengurangi ukuran data keseluruhan yang diperlukan untuk menyediakan integritas dan ketahanan data. Pengurangan ukuran data ini dapat menyebabkan turunnya juga ukuran data yang dikirim ke \textit{node} lain. Dengan demikian, \textit{erasure coding} berpotensi memiliki \textit{threshold} ketika ukuran data cukup besar, ketahanan cukup tinggi, dan jaringan cukup lambat hingga \textit{response time} yang diperlukan untuk melakukan operasi lebih rendah pada operasi tertentu dibandingkan melakukan operasi yang sama pada sistem ter-\textit{replikasi}.

Dinamika yang ada dalam \textit{erasure coding} pada sistem \textit{database} terdistribusi berpotensi meningkatkan kinerja sistem untuk operasi dalam beberapa kasus khusus. Penelitian ini berfokus untuk melakukan eksplorasi dari dinamika tersebut untuk pemanfaatannya pada sistem \textit{database} terdistribusi. Dalam penelitian ini, akan dirancang dan dianalisis \textit{response time} sistem \textit{database} terdistribusi yang memanfaatkan \textit{erasure coding} berdasarkan variabel berupa tingkat ketahanan sistem, ukuran data, kecepatan jaringan, dan kemampuan komputasi sistem. Analisis tersebut diharapkan dapat menentukan \textit{threshold} ketika response time operasi pada sistem berbasis \textit{erasure coding} lebih cepat dibandingkan sistem yang menggunakan replikasi. Selain itu, dari \textit{threshold} yang ditemukan, diharapkan dapat dipelajari dan dimanfaatkan ke dalam sebuah implementasi sistem \textit{database} terdistribusi.
