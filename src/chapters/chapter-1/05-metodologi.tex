\section{Metodologi}

Terdapat beberapa metodologi yang digunakan untuk melaksanakan tugas akhir ini, berikut adalah tahapan pelaksanaannya:

\begin{enumerate}
  \item \textbf{Identifikasi Permasalahan}

        Pada tahap ini, dilakukan identifikasi permasalahan terkait penggunaan \textit{erasure coding} pada sistem \textit{database} terdistribusi yang menjadi objek penelitian. Hasil dari identifikasi ini digunakan agar sistem yang dikembangkan untuk keperluan eksperimen dapat menggambarkan penggunaan dampak \textit{erasure coding} sesuai dengan batasan dan rumusan masalah.

  \item \textbf{Perancangan Eksperimen}

        Setelah mengidentifikasi permasalahan, dilakukan perancangan eksperimen untuk menyiapkan eksperimen yang menganalisis pengaruh \textit{erasure coding} terhadap \textit{response time} sistem \textit{database} terdistribusi. Tahap ini akan mendefinisikan sistem \textit{database} terdistribusi, lingkungan uji, dan skenario beban kerja yang disimulasikan.

  \item \textbf{Implementasi Sistem}

        Dari rancangan eksperimen tersebut, dikembangkan sebuah sistem \textit{database} terdistribusi yang digunakan untuk mengukur pengaruh penggunaan \textit{erasure coding} terhadap \textit{response time} operasi. Hasil dari tahap ini berupa implementasi sistem \textit{database} terdistribusi yang memenuhi seluruh kebutuhan untuk eksperimen.

\item \textbf{Eksperimen dan Pengujian}

        Setelah implementasi berhasil dilakukan, dilakukan serangkaian eksperimen dan pengujian sistem berdasarkan rancangan yang telah dibuat sebelumnya. Eksperimen ini diharapkan menghasilkan data pengukuran \textit{response time} pada berbagai skenario penggunaan \textit{database} terdistribusi.

\item \textbf{Analisis dan Evaluasi}

        Data hasil eksperimen dan pengujian akan dianalisis  untuk menganalisis pengaruh penggunaan \textit{erasure coding} terhadap kinerja operasi sistem \textit{database} terdistribusi. Setelah analisis dilakukan, disimpulkan dampak penerapan \textit{erasure coding} serta {threshold} ketika sebuah operasi pada \textit{database} terdistribusi dengan sistem berbasis \textit{erasure coding} memiliki \textit{response time} yang lebih rendah dibandingkan \textit{database} serupa yang menggunakan sistem replikasi.
        
\end{enumerate}

% ::TODO: fill::