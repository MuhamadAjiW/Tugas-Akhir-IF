\section{Sistematika Pembahasan}

Proses pengembangan PERISAI akan dibagi menjadi lima bagian yang terdiri dari:

\begin{enumerate}
  \item \textbf{Pendahuluan}

        Bab I menjelaskan gagasan utama dari tugas akhir ini yang berisi dari latar belakang, rumusan masalah, tujuan, batasan, metodologi hingga sistematika pembahasan mengenai proses pengembangan sistem PERISAI.

  \item \textbf{Studi Literatur}

        Bab II menjelaskan mengenai studi literatur yang telah dilakukan selama melakukan penilitian mengenai proses \textit{remote deployment} untuk aplikasi IoT. Selain itu, bab ini juga membahas mengenai teori dasar mengenai topik yang dibahas.

  \item \textbf{Analisis Persoalan dan Rancangan Solusi}

        Bab III menjelaskan analisis dari masalah yang ditemukan pada kondisi saat ini diikuti dengan kemungkinan solusi yang dianalisis kekurangan dan kelebihannya. Selain itu, bab ini juga membahas mengenai rancangan solusi yang dipilih sebagai implementasi dari sistem PERISAI.

  \item \textbf{Implementasi dan Pengujian}


        Bab IV menjelaskan hasil implementasi dari rancangan yang telah dibuat. Selain itu, bab ini juga menjelaskan mengenai pengujian yang dilakukan.

  \item \textbf{Kesimpulan dan Saran}


        Bab V menjadi penutup dari pembahasan penelitian ini. Bab ini berisi kesimpulan dan saran dari penulis setelah menjalani penelitian ini. Dengan ini, penulis bisa mengedukasi pembaca agar dapat membuat penelitian ini menjadi lebih baik lagi.

\end{enumerate}