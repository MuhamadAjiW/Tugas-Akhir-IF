\chapter{Analisis Persoalan dan Rancangan Solusi}
\label{chapter:analisis-persoalan-dan-rancangan-solusi}

Tujuan utama penulisan bab ini adalah untuk menjelaskan proses solusi atas masalah pencarian kondisi ketika \textit{erasure coding} memiliki \textit{response time} yang lebih cepat dibandingkan replikasi. Bagian ini memaparkan proses analisis masalah, rancangan solusi, serta implementasi solusi.


\section{Analisis}

\subsection{Analisis Permasalahan}
\label{subsection:analisis-permasalahan}

% Berdasarkan latar belakang yang telah diuraikan pada \ref{sec:latar-belakang}, penggunaan \textit{erasure coding} pada sebuah sistem dapat mengurangi kebutuhan penyimpanan data dengan tetap menjaga integritas dan ketahanan data. Walaupun membutuhkan sumber daya komputasi dalam penerapannya, pengurangan ukuran data keseluruhan yang diperlukan menyebabkan juga turunnya ukuran data yang perlu dikirim ke \textit{node} lainnya.Hal ini menyebabkan mungkinnya terdapat suatu kondisi ketika {response time} yang diperlukan untuk melakukan operasi pada \textit{erasure coding} lebih rendah jika dibandingkan dengan melakukan operasi yang sama pada sistem yang menggunakan \textit{replikasi} untuk mencapai ketahanan tersebut.

Berdasarkan latar belakang dan studi literatur yang telah ditentukan, \textit{erasure coding} memiliki potensi untuk menghasilkan \textit{response time} yang lebih rendah dibandingkan dengan sistem berbasis \textit{replikasi} karena walaupun membutuhkan komputasi, ukuran data yang dikirimkan untuk ketahanan lebih rendah dibandingkan replikasi. Riset ini akan menganalisis faktor-faktor untuk mencapai kondisi tersebut. Faktor yang dianalisis antara lain ukuran data yang besar dan jaringan yang lambat. Dengan demikian, diperlukan sistem yang dapat mensimulasikan operasi pada sistem \textit{erasure coding} dan replikasi sekaligus memvariasikan faktor-faktor tersebut dalam operasinya tanpa memengaruhi kinerja sistem di sisi lain.

Dari permasalahan tersebut, dirumuskan kebutuhan sebuah perangkat lunak antara lain
\begin{enumerate}

    \item Sistem harus dapat mensimulasikan kondisi \textit{database} terdistribusi yang menggunakan replikasi ataupun \textit{erasure coding}.
    \item Sistem harus dapat menyimpan data secara \textit{persistent}.
    \item Sistem harus dapat memvariasikan ukuran data dan kecepatan jaringan.
    \item Sistem harus dapat menjalankan eksperimen berulang kali untuk mendapatkan data dari eksperimen.

\end{enumerate}

Penjelasan lebih detail mengenai kebutuhan sistem terdapat pada lampiran \ref{subsection:rancangan-struktural}.

\section{Alternatif Solusi}
\label{sec:alternatif-solusi}

Berdasarkan permasalahan pada bagian \ref{sec:analisis-permasalahan}, terdapat berbagai macam alternatif solusi untuk menyelesaikan permasalahan tersebut. Pada penelitian ini, solusi yang dipilih adalah dengan membuat sistem yang mengkombinasikan \textit{in-memory key value store} dengan RocksDB sebagai \textit{persistent storage} dengan Reed-Solomon sebagai algoritma \textit{erasure coding} dan OmniPaxos sebagai protokol konsensus. Bagian ini akan menjelaskan alternatif solusi yang dipertimbangkan, perbandingan solusi tersebut, serta alasan pemilihan solusi yang digunakan dalam penelitian ini.

\input{chapters/chapter-3/02-01-alteratif-sistem.tex}
\input{chapters/chapter-3/02-02-alteratif-kakas.tex}
\input{chapters/chapter-3/02-03-alteratif-algoritma-ec.tex}
\input{chapters/chapter-3/02-04-alteratif-algoritma-konsensus.tex}


\section{Kebutuhan Sistem}
\label{sec:system-requirements}

Hasil analisis permasalahan pada Bagian \ref{sec:analisis-permasalahan} diturunkan kebutuhan fungsional dan non-fungsional sistem. Kebutuhan ini kemudian dipetakan pada komponen-komponen yang akan diimplementasikan dalam sistem eksperimen.

\subsection{Kebutuhan Fungsional}
\label{subsection:functional-requirements}

Dalam pengembangannya, sistem eksperimen ini harus memenuhi beberapa kebutuhan fungsional. Kebutuhan fungsional yang diturunkan dari analisis permasalahan pada bagian \ref{sec:analisis-permasalahan} adalah dapat dilihat pada tabel \ref{sec:alternatif-solusi}.

\begin{longtable}{|l|p{13cm}|}
\caption{Kebutuhan Fungsional}
\label{tab:functional-requirements} \\
\hline
\rowcolor{black!10} ID & Deskripsi Kebutuhan \\ \hline
\endfirsthead

\caption[]{Kebutuhan Fungsional (lanjutan)} \\
\hline
\rowcolor{black!10} ID & Deskripsi Kebutuhan \\ \hline
\endhead

F-1 & Sistem harus dapat melakukan operasi \textit{read} dan \text{write} pada sebuah \textit{key-value store database} \\ \hline
F-2 & Sistem harus dapat menyimpan data secara \textit{persistent} \\ \hline
F-3 & Sistem harus dapat mencatat waktu transaksi dari \textit{request} masuk hingga operasi selesai \\ \hline
F-4 & Sistem harus dapat meng-\textit{encode} data menggunakan \textit{erasure coding} \\ \hline
F-5 & Sistem harus dapat merekonstruksi data dari data yang disimpan menggunakan \textit{erasure coding} \\ \hline
F-6 & Sistem harus dapat mendistribusikan data atau sebagian data ke \textit{node} lain untuk keperluan ketahanan data \\ \hline
F-7 & Sistem harus dapat dikonfigurasi untuk menggunakan replikasi ataupun \textit{erasure coding} tanpa mengganti konfigurasi lainnya \\ \hline
F-8 & Sistem harus dapat dikonfigurasi untuk mencapai tingkat ketahanan tertentu tanpa mengganti konfigurasi lainnya \\ \hline
F-9 & Sistem harus dapat mensimulasikan \textit{request} dengan ukuran data yang bervariasi \\ \hline
F-10 & Sistem harus dapat menjalankan \textit{request} secara berulang kali dan bervariasi secara otomatis untuk pengumpulan data \\ \hline
F-11 & Sistem harus dapat melakukan \textit{logging} dari \textit{request} dan operasi untuk kebutuhan analisis \\ \hline
\end{longtable}

\subsection{Kebutuhan Non-fungsional}
\label{subsection:non-functional-requirements}

Dalam pengembangannya, sistem eksperimen ini harus memenuhi beberapa kebutuhan non-fungsional. Kebutuhan fungsional ini diturunkan dari bagian \ref{sec:latar-belakang} dan analisis permasalahan pada bagian \ref{sec:analisis-permasalahan}. Kebutuhan non-fungsional sistem dapat dilihat pada tabel \ref{tab:non-functional-requirements}.

\begin{longtable}{|l|p{13cm}|}
\caption{Kebutuhan Non-Fungsional}
\label{tab:non-functional-requirements} \\
\hline
\rowcolor{black!10} ID & Deskripsi Kebutuhan \\ \hline
\endfirsthead

\caption[]{Kebutuhan Non-Fungsional (lanjutan)} \\
\hline
\rowcolor{black!10} ID & Deskripsi Kebutuhan \\ \hline
\endhead

NF-1 & Sistem harus menyediakan \textit{consistency} yang tinggi dengan \textit{request} ke \textit{node} manapun harus menghasilkan hasil yang sama \\ \hline
NF-2 & Sistem harus memiliki \textit{availability} yang tinggi dengan harus dapat tetap tersedia walaupun beberapa \textit{node} ada dalam kondisi gagal \\ \hline
NF-3 & Sistem harus menggunakan penyimpanan minimal untuk skalabilitas dan efisensi biaya \\ \hline
NF-4 & Sistem harus menyediakan \textit{response time} rendah untuk operasi \textit{read} dan \textit{write} \\ \hline
\end{longtable}

\section{Rancangan}

Seperti yang telah dijelaskan pada \ref{sec:alternatif-solusi}, solusi yang dipilih adalah membuat sistem dengan mengkombinasikan Memcached sebagai \textit{in-memory key-value store} dan menggunakan RocksDB sebagai \textit{persistent database}. Replikasi dan \textit{erasure coding} hanya dilakukan pada data persisten yang disimpan pada \textit{database} tersebut.

\subsection{Modul}
\label{subsection:modules}

Kebutuhan fungsional dan non-fungsional tersebut dipetakan pada modul-modul yang diimplementasikan dalam sistem eksperimen. Setiap modul dirancang untuk memenuhi tujuan spesifik sesuai dengan kebutuhan sistem.

\subsubsection{Database Node}
\label{subsubsection:database-node}

\textit{Database Node} merupakan modul utama yang berperan sebagai \textit{key-value store} pada eksperimen ini. Berdasarkan keputusan yang diambil pada bagian \ref{subsection:modules}, modul ini akan dibuat modular dengan komponen-komponen terpisah. Memcached akan digunakan sebagai \textit{in-memory key-value store} dan RocksDB sebagai \textit{persistent database}. Untuk menggabungkan kedua komponen tersebut, akan dibuat komponen tambahan yang berperan sebagai \textit{controller}.

Komponen \textit{controller} akan berperan untuk mengambil data dari memori jika tersedia dan dari \textit{database} jika data tidak ditemukan di memori untuk operasi \textit{read}. Sementara itu, pada operasi \textit{write}, \textit{controller} akan berperan untuk menyebarkan data yang diproses menggunakan \textit{storage module} ke \textit{database node} lainnya. \textit{Controller} juga akan berperan untuk mengelola konsistensi antar-node menggunakan algoritma konsensus. Penjelasan mengenai peran \textit{database node} pada arsitektur sistem keseluruhan dapat dilihat pada bagian \ref{subsection:system-architecture}.

Merujuk analisis kebutuhan sistem pada bagian \ref{sec:analisis-kebutuhan-sistem}, modul \textit{database node} memenuhi kebutuhan 1 dan 2, yaitu:

\begin{enumerate}
    \item Sistem harus dapat mensimulasikan kondisi \textit{database} terdistribusi yang menggunakan replikasi ataupun \textit{erasure coding}.
    \item Sistem harus dapat menyimpan data secara \textit{persistent} untuk mensimulasikan kegagalan dan pemulihan.
\end{enumerate}

Sedangkan merujuk kebutuhan fungsional dan non-fungsional pada bagian \ref{subsection:system-requirements}, modul ini harus memenuhi kebutuhan fungsional F-1, F-2, F-6, F-7, F-8, dan F-11 serta non-fungsional NF-1, NF-2, NF-3, dan NF-4.

\subsubsection{Storage Module}
\label{subsubsection:storage-module}

Modul ini berfungsi untuk mengatur mekanisme \textit{database node} dalam meningkatkan ketahanan data. Salah satu peran yang dilakukan adalah \text{encode/decode} untuk \textit{erasure coding} data yang diterima dan membaginya ke dalam beberapa bagian \textit{shard} yang kemudian akan dikembalikan pada komponen \textit{controller} pada \textit{database node} untuk didistribusikan. Selain itu, modul ini juga akan menghadirkan pilihan untuk menggunakan replikasi data. Pengkhususan modul ini dari modul-modul lainnya dilakukan untuk mengisolasi perbedaan antara replikasi dan \textit{erasure coding} tanpa mengubah variabel lainnya.

Merujuk pada bagian \ref{sec:analisis-kebutuhan-sistem}, \textit{Storage Module} memenuhi kebutuhan 1, yaitu:

\begin{enumerate}
    \item Sistem harus dapat mensimulasikan kondisi \textit{database} terdistribusi yang menggunakan replikasi ataupun \textit{erasure coding}.
\end{enumerate}

Sedangkan merujuk kebutuhan fungsional dan non-fungsional pada bagian \ref{subsection:system-requirements}, modul ini harus memenuhi kebutuhan fungsional F-4, F-5, dan F-11 serta non-fungsional NF-1, NF-3, dan NF-4.

\subsubsection{Data Collection Module}
\label{subsubsection:data-collection-module}

Modul ini merupakan modul yang berperan untuk melakukan \textit{request} dan melakukan transaksi pada sistem. Hal utama yang akan dilakukan oleh modul ini adalah pengumpulan data dari eksperimen. Oleh karena itu, modul ini akan memiliki fitur yang dapat memvariasikan ukuran data, transaksi, dan \textit{timer} untuk mengukur \textit{response time} dari operasi. Selain itu, modul ini juga perlu dilengkapi dengan otomasi agar dapat menjalankan eksperimen berulang kali untuk mendapatkan data persentil dari eksperimen.

Merujuk pada bagian \ref{sec:analisis-kebutuhan-sistem}, \textit{Data Collection Module} memenuhi kebutuhan 3 dan 4, yaitu:

\begin{enumerate}
    \setcounter{enumi}{2}
    \item Sistem harus dapat memvariasikan ukuran data, tingkat ketahanan, kecepatan jaringan, dan kemampuan komputasi.
    \item Sistem harus dapat menjalankan eksperimen berulang kali untuk mendapatkan data persentil dari eksperimen.
\end{enumerate}

Sedangkan merujuk kebutuhan fungsional dan non-fungsional pada bagian \ref{subsection:system-requirements}, modul ini harus memenuhi kebutuhan fungsional F-3, F-9, F-10, dan F-11

\subsubsection{Modul Lainnya}
\label{subsubsection:other-modules}

Selain ketiga modul yang sudah dijelaskan, akan terdapat beberapa modul minor yang diperlukan untuk meningkatkan kinerja sistem dan mempermudah keberjalanan eksperimen. Contoh modul-modul tersebut antara lain adalah \textit{load balancer}, \textit{logger}, dan \textit{startup-script}. \textit{Load balancer} berfungsi untuk mendistribusikan \textit{request} dari \textit{client} ke \textit{database node} yang tersedia. Modul \textit{logger} berfungsi untuk mencatat kinerja sistem serta memudahkan analisis hasil eksperimen. Sedangkan modul \textit{startup-script} akan membantu pembangunan sistem \textit{key-value store database} secara keseluruhan untuk mengatur konfigurasi ketahanan sistem. Modul-modul ini tidak memiliki peran yang krusial dalam sistem, namun membantu dalam mempermudah pengembangan dan analisis. Dinamika eksperimen tidak menutup kemungkinan adanya modul tambahan yang diperlukan. Beberapa kebutuhan pada poin 4 bagian \ref{sec:analisis-kebutuhan-sistem} akan diimplementasikan melalui \textit{platform} tempat eksperimen dilakukan.


\section{Arsitektur Sistem}
\label{sec:system-architecture}

\begin{figure}[ht]
    \centering
    \includegraphics[width=0.95\textwidth]{resources/chapter-3/general-architecture.png}
    \caption{Gambaran Arsitektur Sistem Eksperimen}
    \label{fig:general-architecture}
\end{figure}

Arsitektur dari sistem mengasumsikan kebutuhan untuk konsistensi yang tinggi. Untuk mencapai konsistensi tersebut, operasi \textit{write} dilakukan secara \textit{synchronous} dengan distribusi replikasi dan \textit{erasure coding} dianggap selesai ketika nilai ketahanan yang diinginkan sudah tercapai.

Karena sistem bersifat terdistribusi, maka diperlukan sebuah algoritma konsensus untuk mengelola konsistensi antar \textit{Node}. Algoritma konsensus yang digunakan algoritma konsensus \textit{paxos} yang disesuaikan dengan kebutuhan. Salah satu penyesuaian yang dilakukan adalah mengadopsi pola \textit{leader-follower} untuk memudahkan sinkronisasi data dan mempercepat transaksi. Dengan adanya leader, fase 1 dari algoritma \textit{paxos} dapat dihilangkan dengan membuat proposal dari leader selalu memiliki nilai paling tinggi. Detail implementasi \textit{paxos} akan dijelaskan di bagian \ref{sec:detail-komponen}. Diagram gambaran arsitektur sistem dapat dilihat pada gambar \ref{fig:general-architecture}.

Operasi \textit{write} akan secara ekslusif disalurkan pada \textit{leader}. Kemudian untuk ketahanan, data akan didistribusikan pada \textit{follower} sesuai dengan konfigurasi \textit{node}. Sementara itu, operasi \textit{read} dapat dilakukan pada \textit{Node} manapun. Pada sistem \textit{erasure coding}, jika pada \textit{node} tersebut tidak terdapat nilai data yang dicari, maka \textit{Node} akan melakukan \textit{request} ke semua node lainnya untuk melakukan rekonstruksi data.


