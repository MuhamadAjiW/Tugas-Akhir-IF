\subsection{Erasure Coding for Distributed Storage: An Overview}

Riset yang dilakukan oleh Balaji \textit{et al} pada tahun 2018 ini menjelaskan mengenai teknik-teknik erasure coding yang digunakan pada \textit{distributed database}, teruatama untuk penyimpanan data dalam jumlah besar di \textit{data center}. Dari riset ini dijelaskan perkembangan terkait metode dan teknik untuk implementasi \textit{erasure coding} pada \textit{distributed storage}. Dari metode dan teknik tersebut, dibahas secara mendalam mengenai teknologi \textit{Regenerating code} dan \textit{locally recoverable code} untuk memulihkan data yang hilang dengan usaha sesedikit mungkin.

Beberapa kesimpulan yang dapat diambil dari riset ini adalah bahwa operasi perbaikan berbanding terbalik dengan \textit{bandwidth}. Dalam kata lain, semakin efisien penyimpanan menggunakan erasure coding, semakin banyak \textit{bandwidth} yang diperlukan untuk operasi pemulihan data \parencite{balaji2018erasure}. Di akhir riset ini juga disebutkan bahwa perkembangan \textit{erasure coding} memungkinkan sistem penyimpanan yang lebih \textit{reliable} dan efisien. Namun, pertukaran terhadap penyimpanan, \textit{bandwidth} yang dibutuhkan untuk pemulihan, dan batasan untuk \textit{deployment} pada dunia nyata membutuhkan riset berkelanjutan. 
