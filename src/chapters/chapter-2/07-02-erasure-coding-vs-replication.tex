\subsection{Erasure Coding vs. Replication: A Quantitative Comparison}

Riset yang dilakukan oleh Weatherspoon dan Kubiatowicz pada tahun 2002 ini menjelaskan perbandingan dari penggunaan \textit{erasure coding} dan \textit{replication} pada \textit{distributed database}. Perbandingan yang dilakukan adalah dalam lingkup penyimpanan yang dibutuhkan, \textit{bandwidth} yang digunakan dalam operasi, dan jumlah total \textit{seek} yang diperlukan oleh hard disk.

Beberapa kesimpulan yang dapat diambil dari riset ini adalah bahwa \textit{erasure coding} mengurangi penggunaan penyimpanan sistem secara signifikan untuk ketahanan yang sama dibandingkan dengan replikasi. Operasi \textit{seek} pada \textit{disk} juga berkurang jika dibandingkan dengan replikasi jika menggunakan lokalitas tempat dan temporal. Selain itu, \textit{erasure coding} juga dapat mencapai \textit{availability} yang lebih tinggi dibandingkan dengan replikasi untuk jumlah sumber daya yang sama. Namun, perlu diketahui juga bahwa riset ini berfokus untuk membahas mengenai penyimpanan terdistribusi yang tidak memerlukan operasi konstan seperti \textit{database} terdistribusi.