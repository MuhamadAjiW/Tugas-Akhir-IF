\subsection{When Paxos Meets Erasure Code: Reduce Network and Storage Cost in State Machine Replication}
\label{subsection:paxos-erasure}

Riset yang dilakukan oleh Mu \textit{et al} pada tahun 2014 \parencite{mu2014paxos} membahas tentang penerapan \textit{erasure coding} menggunakan algoritma \textit{Reed-Solomon} pada algoritma Paxos untuk mengurangi biaya jaringan dan penyimpanan. Hasil penelitian ini menunjukkan bahwa penerapan \textit{erasure coding} pada algoritma Paxos dapat mengurangi biaya jaringan, tingkat \textit{input-output disk}, dan biaya penyimpanan dibandingan dengan \textit{full-copy replication}. Hal lain yang ditekankan pada penelitian ini adalah bahwa kuorum yang dibutuhkan untuk mencapai konsensus dalam sistem \textit{erasure coding} bukan kuorum mayoritas, melainkan bergantung pada jumlah \textit{data shards} yang digunakan.

Penelitian ini membahas dan menunjukkan hasil evaluasi keuntungan dari penerapan \textit{erasure coding} pada algoritma Paxos bergantung pad aukuran nilai dan kecepatan jaringan serta \textit{input-output disk} dengan percoobaan dilakukan menggunakan data kecil (<64kb), data besar (>256kb), jaringan \textit{wide}, jaringan lokal, penggunaan HDD, dan penggunaan SSD. Kode yang digunakan untuk penelitian ini tidak tersedia secara publik.