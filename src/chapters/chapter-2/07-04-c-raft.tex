\subsection{CRaft: An Erasure-coding-supported Version of Raft for Reducing Storage Cost and Network Cost}
\label{subsection:craft}

Riset yang dilakukan oleh Wang \textit{et al} pada tahun 2020 memperkenalkan CRaft (\textit{Coded Raft}), sebuah protokol konsensus berbasis Raft yang menggunakan \textit{erasure coding}. Penelitian ini bertujuan untuk mengatasi masalah biaya penyimpanan dan jaringan yang disebabkan oleh replikasi data secara utuh pada konsensus standar seperti Raft dan Paxos.

Penelitian ini juga menganalisis protokol RS-Paxos yang dijelaskan sebelumnya pada \ref{subsection:paxos-erasure} dan menawarkan solusi berupa algoritma konsensus serupa berbasis Raft. Ditemukan bahwa RS-Paxos memiliki kekurangan yaitu penurunan \textit{liveness} karena kondisi \textit{commit} pada RS-Paxos yang bergantung pada jumlah \textit{data shard}. Solusi yang diusulkan dengan CRaft adalah dengan dua jenis pengiriman data. Jenis pengiriman pertama adalah dengan mengirimkan \textit{erasure coded shard} ketika sistem dalam kondisi sehat dan \textit{leader} dapat berkomunikasi dengan jumlah \textit{follower} yang dibutuhkan seperti pada RS-Paxos. Jenis pengiriman kedua dilakukan ketika jumlah \textit{node} turun di bawah jumlah yang dibutuhkan namun masih merupakan mayoritas. Pada kondisi ini, \textit{leader} akan mengirimkan data lengkap tanpa \textit{erasure coding} untuk menghindari penurunan \textit{liveness}. Kode yang digunakan untuk penelitian ini tidak tersedia secara publik.