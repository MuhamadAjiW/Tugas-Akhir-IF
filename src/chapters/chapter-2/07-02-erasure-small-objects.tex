\subsection{Erasure coding for small objects in in-memory KV storage}
\label{subsection:erasure-small-objects}

Riset yang dilakukan oleh Yiu \textit{et al} pada tahun 2017 ini memperkenalkan terkait sebuah prototipe \textit{in-memory key-value store} berbasis \textit{erasure coding} yang mencapai \textit{availability} tinggi dengan pemulihan cepat \parencite{yiu2017erasure}. Prototipe ini diberi nama MemEC. MemEC didesain secara spesifik untuk beban kerja yang didominasi oleh benda kecil. \textit{Erasure coding} yang diterapkan pada MemEC bersifat novel dengan seluruh data yang disimpan, berupa \textit{key}, \textit{value}, dan \textit{metadata} diterapkan \textit{erasure coding} pada penyimpanannya. Percobaan ini juga mensimulasikan kondisi ketika kegagalan sering terjadi dan menganalisis kinerja MemEC ketika dalam kondisi tersebut.

Dari percobaan yang dilakukan, didapat bahwa MemEC, dapat mengurangi penyimpanan sebanyak 60\%. Kinerja \textit{throughput} dan \textit{latency} juga sebanding dengan Redis atau Memcached. Hal ini mendemonstrasikan mungkinnya pengaplikasian \textit{erasure coding} di sebuah \textit{in-memory key-value story}, khususnya untuk beban kerja berupa objek kecil. Pengembangan yang dilakukan mencapai efisiensi penyimpanan tanpa mengorbankan \textit{availability}, \textit{consistency}, ataupun kinerja secara signifikan.
