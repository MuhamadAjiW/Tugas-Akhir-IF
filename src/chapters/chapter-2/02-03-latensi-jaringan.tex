\subsection{Latensi Jaringan}

Latensi jaringan adalah waktu yang dibutuhkan untuk data menempuh jarak dari satu titik ke titik lainnya melewati jaringan. Secara teori, data dapat melintasi internet dengan kecepatan yang mendekati kecepatan cahaya. Namun, pada nyatanya, data bergerak lebih lambat sebab jarak, infrastruktur internet, ukuran paket, kemacetan jaringan, dan variabel lainnya \parencite{goodwin2023latency}. Hal ini terjadi ketika data perlu dikirim dari satu perangkat ke perangkat lainnya, misalnya pada aplikasi \textit{server} dan \textit{client}.

Pada sistem terdistribusi, latensi jaringa berperan besar terhadap \textit{response time} karena sistem terpisah dari lokasi fisik yang jauh. Hal ini disebabkan mendistribusikan data pada jaringan lokal tidak memiliki banyak guna \parencite{johansson2000impact}. Sekalipun berjarak dekat, pengiriman data antar perangkat membutuhkan waktu untuk mengirimkan data atau \textit{transmit time} dan penerimaan dari data tersebut di sisi penerima.
