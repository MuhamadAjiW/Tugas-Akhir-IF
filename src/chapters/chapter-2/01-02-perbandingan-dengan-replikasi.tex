\subsection{Perbandingan dengan Replikasi}

Replikasi menyeluruh dengan menyalin data bit-per-bit adalah salah satu cara umum untuk meningkatkan ketahanan data. Untuk mencapai ketahanan pada tingkat kegagalan tertentu, perlu dilakukan penyalinan data dengan tingkat yang sama. Pada implementasi ini, ketahanan data yang tinggi akan meningkatkan keperluan \textit{bandwidth} dan penyimpanan dari sistem. Dengan demikian, teknik ini memerlukan \textit{bandwidth} dan penyimpanan yang tinggi  \parencite{weatherspoon2002erasure}.

Sebagai contoh, untuk menyimpan data dengan nilai ketahanan empat kegagalan, replikasi mengharuskan data untuk ditulis lima kali di tempat yang berbeda. Seperti yang sudah dijelaskan sebelumnya, penggunaan penyimpanan pada \textit{Erasure coding} lebih sedikit jika dibandingkan dengan replikasi. Sebagai contoh, \textit{erasure coding} berpendekatan kode Reed-Solomon dengan konfigurasi $(12, 4)$, yaitu dua belas fragmen data dan empat fragmen kode, akan mengurangi penyimpanan sebanyak 1.33x jika dibandingkan dengan replikasi untuk tingkat ketahanan yang sama.

Untuk meraih ketahanan pesan $t$, replikasi menggunakan penyimpanan $M$ sebanyak yang dapat dihitung pada Persamaan \ref{eq:replication_storage}.

\begin{equation}
	M = \text{Ukuran pesan} \times (t + 1)
	\label{eq:replication_storage}
\end{equation}

Dibandingkan \textit{erasure coding} yang menggunakan reed-solomon, dapat dihitung rasio penyimpanan \textit{erasure coding} dengan replikasi $R$, yaitu pada Persamaan \ref{eq:erasure_storage_simplified}.

\begin{equation}
	R = \frac{\text{Ukuran pesan} \times (t + 1)}{\text{Ukuran pesan} \times \frac{\text{k + t}}{k}}
	\label{eq:erasure_storage}
\end{equation}

\begin{equation}
	R = \frac{k + kt}{k + t}
	\label{eq:erasure_storage_simplified}
\end{equation}

% Setelah penyederhanaan, didapatkan

% Berdasarkan persamaan tersebut didapatkan bahwa penyimpanan replikasi akan selalu lebih besar dibandingkan \textit{erasure coding}. Perbedaan penyimpanan yang digunakan terpengaruhi oleh $t$, ketahanan dari sistem, dan $k$, jumlah persebaran yang digunakan pada kode Reed-solomon.
Berdasarkan persamaan tersebut didapatkan perbedaan penyimpanan yang digunakan terpengaruhi oleh $t$, ketahanan dari sistem, dan $k$, jumlah persebaran yang digunakan pada kode Reed-solomon. Untuk tingkat ketahanan yang sama, penyimpanan replikasi akan selalu lebih besar dibandingkan \textit{erasure coding} kecuali jika $k$ bernilai satu.

Yang dikorbankan penggunaan \textit{erasure coding} dibandingkan replikasi adalah kinerja. Penurunan kinerja terjadi ketika berurusan dengan data yang hilang atau \textit{offline} dan data penyimpanan yang sering diakses. Pada kasus $(12, 4)$ yang sebelumnya, untuk membangun ulang data perlu membaca dari dua belas fragmen terpisah, ini meningkatkan kemungkinan untuk mengenai penyimpanan yang sering diakses dan biaya jaringan dan \textit{input-output} sehingga menambahkan \textit{response time} dalam operasi membaca. Sementara itu, operasi membaca bisa dikembalikan tanpa operasi apapun di sebuah node sembarang pada replikasi \parencite{huang2012erasure}. Akan tetapi, Penyimpanan data yang kecil menyebabkan penggunaan \textit{bandwidth} yang lebih sedikit juga sehingga dalam kasus tertentu dengan variabel lainnya \textit{erasure coding} dapat menghasilkan operasi denngan \textit{response time} yang lebih cepat.
