\section{\textit{Erasure Coding}}
\label{sec:erasure-coding}

\textit{Erasure Coding} adalah sebuah metode proteksi data untuk sistem penyimpanan terdistribusi dengan membagi file data menjadi blok data dan \textit{parity} lalu mengkodekannya sehingga data primer dapat dipulihkan bahkan jika bagian dari data terkodenya tidak tersedia. Sistem penyimpanan terdistribusi yang dapat diskalakan secara horizontal mengandalkan \textit{erasure coding} untuk menyediakan proteksi data dengan menyimpan data terkode di beberapa \textit{drive} dan node. Jika sebuah \textit{drive} atau node gagal atau menjadi korup, data asli dapat dibangun ulang dari blok yang tersimpan pada \textit{drive} dan node lain \parencite{minio2022erasure}.

Skema \textit{erasure coding} memberikan ketahanan lebih tinggi di harga penyimpanan yang lebih rendah sehingga merupakan alternatif terhadap replikasi di sistem penyimpanan terdistribusi \parencite{silberstein2014lazy}. Pada skala susunan penyimpanan yang sangat besar, ruang \textit{hosting} dan daya merupakan biaya signifikan sehingga \textit{erasure code} dapat digunakan untuk meminimalkan biaya keseluruhan sistem \parencite{manasse2009reed}. \textit{Erasure coding} dapat mengurangi biaya penyimpanan dengan lebih dari 50\% dengan mengurangi pembelian perangkat keras dan daya dari menjalankan perangkat keras tersebut \parencite{huang2012erasure}. Sistem penyimpanan terdistribusi umumnya mengimplementasikan \textit{erasure coding} berdasarkan kode Reed-Solomon, namun beberapa sistem penyimpanan terdistribusi termasuk HDFS, Ceph, dan Swift menyediakan bermacam \textit{erasure coding} dan fungsi kontrol konfigurasinya \parencite{kim2021erasure}. Menyesuaikan dengan batasan, implementasi \textit{erasure coding} yang dibahas akan menggunakan kode Reed-Solomon.

\subsection{Kode Reed-Solomon}

Pada tahun 1960, I. Reed dan G. Solomon mengembangkan teknik pengkodean blok yang disebut kode Reed-Solomon. Saat ini, kode Reed-Solomon tetap populer karena sesuai dengan standar dan implementasi yang efisien di berbagai format perangkat keras dan perangkat lunak \parencite{minio2022erasure}. Algoritma Reed-Solomon menyediakan teknik sederhana yang efisien untuk mengkodekan ulang informasi sehingga kegagalan beberapa disk dalam susunan disk tidak mengganggu ketersediaan data \parencite{manasse2009reed}.

% Kode Reed-Solomon memanfaatkan matematika pada medan berhingga atau disebut juga medan Galois. Medan berhingga adalah himpunan yang memiliki operasi pertambahan, pengurangan, perkalian, dan pembagian (kecuali dengan nol) yang didefinisikan dan memenuhi aturan tertentu. Sebuah grup dalam medan berhingga memiliki sifat seperti berikut 
% \parencite{forney2005finitefields}:

% \begin{itemize}
%     \item \textit{Closure}: untuk setiap $a \in G$, $b \in G$, elemen $a \oplus b$ berada pada $G$.
%     \item \textit{Associative}: untuk setiap $a,b,c \in G$, $(a \oplus b) \oplus c = a \oplus (b \oplus c)$.
%     \item \textit{Identity}: Ada elemen identitas $0$ pada $G$ yang mana $a \oplus 0 = 0 \oplus a = a$ untuk semua $a \in G$.
%     \item \textit{Inverse}: Untuk setiap $a \in G$, ada sebuah \textit{inverse} $(-a)$ sehingga $a \oplus (-a) = 0$.
% \end{itemize}

Kode Reed-Solomon dikarakterisasi dengan tiga parameter: jumlah alfabet $q$ yang merupakan tingkatan medan berhingga, panjang blok $n$, dan panjang pesan $k$, dengan $k < n \le q$. Panjang blok biasanya adalah konstanta kelipatan dari panjang pesan. Selain itu, panjang blok adalah sama dengan atau kurang satu dari jumlah alfabet. Dengan menambahkan $t \le n - k$ simbol pengecek ke dalam data. Sebuah kode Reed-Solomon dapat membenarkan hingga $t$ kehilangan pada lokasi yang diketahui dan diberikan kepada algoritma. Pemilihan $t$ tergantung pada perancang kode dan dapat dipilih dalam batasan yang luas \parencite{riley2001introduction}.

% Pada pandangan awal \textcite{reed1960polynomial}, setiap \textit{codeword} dari kode Reed-Solomon adalah sebuah urutan nilai fungsi polinomial dengan derajat kurang dari $k$. Untuk mendapatkan \textit{codeword} dari kode Reed-Solomon, simbol pesan diperlakukan sebagai koefisien dari polinomial $p$ dengan derajat kurang dari $k$, di atas medan terhingga $F$ dengan berjumlah $q$ elemen. Selanjutnya, polinomial $q$ dievaluasi pada $n \le q$ poin yang berbeda $a, \ldots, a_n$ pada medan $F$, dan urutan nilainya adalah \textit{codeword} yang dihasilkan. Pilihan umum untuk poin yang dievaluasi termasuk $\{0, 1, 2, \ldots, n-1\}$, $\{0, 1, a, a^2, \ldots, a^{n-2}\}$, atau untuk $n < q, \{1, a, a^2, \ldots, a^{n-1}\}, \ldots$ dengan $a$ adalah elemen primitif dari $F$.

% Secara formal, set C dari \textit{codewords} kode Reed-Solomon didefinisikan sebagai berikut: \[C = \{(p(a_1),p(a_2),\ldots,p(a_n)) \mid p \text{ polynomial dalam } F \text{ dengan derajat } < k\}\]

% Meskupin jumlah dari polinomial berbeda dengan derajat kurang dari $k$ dan jumlah pesan berbeda keduanya sama dengan $q^k$ sehingga setiap pesan dapat dipetakan secara unik ke polinomial tersebut, ada beberapa cara yang berbeda untuk melakukan enkoding ini. Konstruksi asli dari \textcite{reed1960polynomial} mengartikan pesan $x$ sebagai koefisien dari polinomial $p$, sedangkan konstruksi selanjutnya mengartikan pesan sebagai nilai dari polinomial pada $k$ titik pertama $a_1,\ldots,a_k$ dan mendapatkan polinomial $p$ dengan menginterpolasikan nilai-nilai ini dengan polinomial dengan derajat kurang dari $k$.

% Pada konstruksi asli dari \textcite{reed1960polynomial}, pesan $m = (m_0, \ldots , m_{k-1}) \in F^k$ dipetakan ke sebuah polinomial $p_m$ dengan \[p_m(a) = \sum_{i=0}^{k-1}{m_i}{a^i}\]

% \textit{Codeword} didapatkan dengan mengevaluasi $p_x$ pada $n$ poin berbeda $a_1, \ldots, a_n$ pada medan $F$. Dengan demikian, fungsi pengkodean klasik $C : {F^k} -> {F^m}$ untuk kode Reed-Solomon didefinisikan sebagai berikut:

% \[
%     C(m) = \begin{bmatrix}
%         p_m(a_0)\\
%         p_m(a_1)\\
%         \ldots\\
%         p_m(a_{n-1})
%     \end{bmatrix}  
% \]

% Fungsi $C$ adalah pemetaan linear yang memenuhi $C(m) = Am$ untuk matriks $A$ berukuran ${n} \times {k}$ dengan elemen dari $F$:

% \[
%     C(m) = Am =
%     \begin{bmatrix}
%         1 & a_0 & a_0^2 & \cdots & a_0^{k-1} \\
%         1 & a_1 & a_1^2 & \cdots & a_1^{k-1} \\
%         \vdots & \vdots & \vdots & \ddots & \vdots \\
%         1 & a_{n-1} & a_{n-1}^2 & \cdots & a_{n-1}^{k-1}
%     \end{bmatrix}
%     \begin{bmatrix}
%         m_0 \\
%         m_1 \\
%         \vdots \\
%         m_{k-1}
%     \end{bmatrix}
% \]

% Matrix ini adalah matriks Vandermonde dengan $a_1,\ldots,a_{n-1} \in F$. Matriks Vandermonde adalah sebuah matriks dengan progresi geometris di setiap barisnya. Matriks ini memfasilitasi evaluasi polinomial dengan kolom ke $j$ berkorespondensi dengan $j - 1$ dari poin evaluasi. Sifat penting dari matriks ini \textit{non-singular}, yaitu determinan dari matriks Vandermonde dihitung dan tidak nol jika dan hanya jika semua $a_i$ berbeda dengan persamaan sebagai berikut

% \[
%     \det(V) = \prod_{0 \leq i < j \leq m} (x_j - x_i)
% \]

% Dengan perhitungan determinan tersebut, matriks ini selalu dapat di-\textit{inverse} ketika titik-titik evaluasinya berbeda. \textit{Inverse} dari matriks diperlukan untuk mengembalikan pesan ke bentuk semula. Selain itu, matriks Vandermode juga memiliki independensi linear yang artinya masing-masing kolom terpisah secara linear dari kolom yang lainnya.

Ada beberapa cara untuk menghasilkan kode Reed-Solomon yang sistematis. Salah satunya adalah dengan menggunakan interpolasi lagrange untuk dalam perhitungan polinomial $p_m$ sehingga $p_m(a_i) = m_i \text{ untuk semua } i \in \{0,\ldots,k-1\}$ \parencite{plank1996tutorial}. Untuk menghasilkan matriks pengkodean sistematis, matriks Vandermonde A dikalikan dengan \textit{inverse} dari submatriks kuadrat kiri A.

\begin{align}
	G = {(\text{Sub kuadrat kiri } A)}^{-1} \cdot A =
	\begin{bmatrix}
		1      & 0      & 0      & \cdots & 0      & g_{1,k+1} & \cdots & g_{1,n} \\
		0      & 1      & 0      & \cdots & 0      & g_{2,k+1} & \cdots & g_{2,n} \\
		0      & 0      & 1      & \cdots & 0      & g_{3,k+1} & \cdots & g_{3,n} \\
		\vdots & \vdots & \vdots & \ddots & \vdots & \vdots    & \ddots & \vdots  \\
		0      & \cdots & 0      & \cdots & 1      & g_{k,k+1} & \cdots & g_{k,n}
	\end{bmatrix}
	\label{eq:encoding_matrix}
\end{align}

Matrix pada Persamaan\ref{eq:encoding_matrix} adalah gabungan matriks identitas dengan matriks Vandermonde. Matriks Vandermonde adalah sebuah matriks dengan progresi geometris di setiap barisnya. Matriks ini memfasilitasi evaluasi polinomial dengan kolom ke $j$ berkorespondensi dengan $j - 1$ dari poin evaluasi. Sifat penting dari matriks ini \textit{non-singular}, yaitu determinan dari matriks Vandermonde dihitung dan tidak nol jika dan hanya jika semua $a_i$ berbeda. Sehingga didapat persamaan untuk pemetaan linear $C(m) = Gm$ untuk matriks $G$ berukuran ${n} \times {k}$ dengan elemen dari $F$

\begin{align}
	C(m) = Gm =
	\begin{bmatrix}
		1      & 0      & 0      & \cdots & 0      & g_{1,k+1} & \cdots & g_{1,n} \\
		0      & 1      & 0      & \cdots & 0      & g_{2,k+1} & \cdots & g_{2,n} \\
		0      & 0      & 1      & \cdots & 0      & g_{3,k+1} & \cdots & g_{3,n} \\
		\vdots & \vdots & \vdots & \ddots & \vdots & \vdots    & \ddots & \vdots  \\
		0      & \cdots & 0      & \cdots & 1      & g_{k,k+1} & \cdots & g_{k,n}
	\end{bmatrix}
	\begin{bmatrix}
		m_0    \\
		m_1    \\
		\vdots \\
		m_{k-1}
	\end{bmatrix}
	\label{eq:encoding_equation}
\end{align}

Pada Persamaan \ref{eq:encoding_equation}, tidak semua kolom $g_{x,y}$ pada matrix perlu digunakan. Kolom sejumlah $t \le n - k$ dapat membenarkan $t$ kehilangan pada lokasi yang diketahui dan diberikan kepada algoritma \parencite{riley2001introduction}. Pemulihan dilakukan seperti pada Persamaan \ref{eq:recovery_equation}.

\begin{align}
	G' \times m                & = C(m)' \notag                \\
	G'^{-1} \times G' \times m & = G'^{-1} \times C(m)' \notag \\
	1 \times m                 & = G'^{-1} \times C(m)' \notag \\
	m                          & = G'^{-1} \times C(m)'
	\label{eq:recovery_equation}
\end{align}

Berdasarkan pengkodean yang dilakukan, kode Reed-Solomon menyediakan ketahanan sejumlah $t$ kegagalan dengan penggunaan penyimpanan $M$ yang dapat dihitung dengan Persamaan \ref{eq:ec_storage}

\begin{equation}
	M = \text{Ukuran pesan} \times \frac{k + t}{k}
	\label{eq:ec_storage}
\end{equation}

Namun, perlu diingat bahwa untuk \textit{erasure coding} mencapai ukuran tersebut, perhitungan matriks seperti yang sudah dijelaskan sebelumnya perlu dilakukan. Perhitungan matriks Reed-Solomon memiliki kompleksitas yang bergantung pada konfigurasi serta ukuran pesan yang digunakan. Dengan demikian, kompleksitas komputasi untuk kode Reed-Solomon adalah perkalian matriks yang dapat dinyatakan dengan Notasi \ref{eq:ec_complexity} dengan $n$ adalah jumlah \textit{node} data ($k$) dan \textit{node parity} ($t$).

\begin{equation}
	O(k \cdot n \cdot \frac{\text{Ukuran Pesan}}{k})
	\label{eq:ec_complexity}
\end{equation}

Perhitungan $k$ baris adalah perkalian matriks identitas dan disederhanakan dengan menyalin data. Oleh sebab itu, kompleksitas yang membutuhkan komputasi signifikan adalah perkalian matriks yang dilakukan pada $t$ baris. Dengan demikian, kompleksitas komputasi untuk kode Reed-Solomon adalah seperti yang dinyatakan pada Persamaan \ref{eq:ec_complexity_final}.

\begin{equation}
	O(k \cdot t \cdot \frac{\text{Ukuran Pesan}}{k}) = O(t \cdot \text{Ukuran Pesan})
	\label{eq:ec_complexity_final}
\end{equation}

Dari persamaan tersebut, dapat dilihat bahwa perhitungan ini akan menambah \textit{response time} dari operasi relatif terhadap ukuran pesan yang digunakan.

\subsection{Perbandingan dengan Replikasi}

Replikasi menyeluruh dengan menyalin data bit-per-bit adalah salah satu cara umum untuk meningkatkan ketahanan data. Untuk mencapai ketahanan pada tingkat kegagalan tertentu, perlu dilakukan penyalinan data dengan tingkat yang sama. Pada implementasi ini, ketahanan data yang tinggi akan meningkatkan keperluan \textit{bandwidth} dan penyimpanan dari sistem. Dengan demikian, teknik ini memerlukan \textit{bandwidth} dan penyimpanan yang tinggi  \parencite{weatherspoon2002erasure}.

Sebagai contoh, untuk menyimpan data dengan nilai ketahanan empat kegagalan, replikasi mengharuskan data untuk ditulis lima kali di tempat yang berbeda. Seperti yang sudah dijelaskan sebelumnya, penggunaan penyimpanan pada \textit{Erasure coding} lebih sedikit jika dibandingkan dengan replikasi. Sebagai contoh, \textit{erasure coding} berpendekatan kode Reed-Solomon dengan konfigurasi $(12, 4)$, yaitu dua belas fragmen data dan empat fragmen kode, akan mengurangi penyimpanan sebanyak 1.33x jika dibandingkan dengan replikasi untuk tingkat ketahanan yang sama.

Untuk meraih ketahanan pesan $t$, replikasi menggunakan penyimpanan $M$ sebanyak yang dapat dihitung pada persamaan \ref{eq:replication_storage}.

\begin{equation}
    M = \text{Ukuran pesan} \times (t + 1)
    \label{eq:replication_storage}
\end{equation}

Dibandingkan \textit{erasure coding} yang menggunakan reed-solomon, dapat dihitung rasio penyimpanan \textit{erasure coding} dengan replikasi $R$, yaitu pada persamaan \ref{eq:erasure_storage_simplified}.

\begin{equation}
    R = \frac{\text{Ukuran pesan} \times (t + 1)}{\text{Ukuran pesan} \times \frac{\text{k + t}}{k}}
    \label{eq:erasure_storage}
\end{equation}

\begin{equation}
    R = \frac{k + kt}{k + t}
    \label{eq:erasure_storage_simplified}
\end{equation}

% Setelah penyederhanaan, didapatkan


% Berdasarkan persamaan tersebut didapatkan bahwa penyimpanan replikasi akan selalu lebih besar dibandingkan \textit{erasure coding}. Perbedaan penyimpanan yang digunakan terpengaruhi oleh $t$, ketahanan dari sistem, dan $k$, jumlah persebaran yang digunakan pada kode Reed-solomon.
Berdasarkan persamaan tersebut didapatkan perbedaan penyimpanan yang digunakan terpengaruhi oleh $t$, ketahanan dari sistem, dan $k$, jumlah persebaran yang digunakan pada kode Reed-solomon. Untuk tingkat ketahanan yang sama, penyimpanan replikasi akan selalu lebih besar dibandingkan \textit{erasure coding} kecuali jika $k$ bernilai satu.

Yang dikorbankan penggunaan \textit{erasure coding} dibandingkan replikasi adalah kinerja. Penurunan kinerja terjadi ketika berurusan dengan data yang hilang atau \textit{offline} dan data penyimpanan yang sering diakses. Pada kasus $(12, 4)$ yang sebelumnya, untuk membangun ulang data perlu membaca dari dua belas fragmen terpisah, ini meningkatkan kemungkinan untuk mengenai penyimpanan yang sering diakses dan biaya jaringan dan \textit{input-output} sehingga menambahkan \textit{response time} dalam operasi membaca. Sementara itu, operasi membaca bisa dikembalikan tanpa operasi apapun di sebuah node sembarang pada replikasi \parencite{huang2012erasure}. Akan tetapi, Penyimpanan data yang kecil menyebabkan penggunaan \textit{bandwidth} yang lebih sedikit juga sehingga dalam kasus tertentu dengan variabel lainnya \textit{erasure coding} dapat menghasilkan operasi denngan \textit{response time} yang lebih cepat.

\subsection{Perbandingan dengan RAID}

Cara lain untuk mencapai ketahanan data tinggi adalah dengan menggunakan RAID (\textit{Redundant Array of Inexpensive Disks}). RAID adalah sebuah sistem untuk menggunakan banyak \textit{disk} untuk kebutuhan redundansi data, peningkatan kinerja, atau keduanya. Taksonomi RAID memperkenalkan skema penomoran untuk membedakan cara pengenalan redundansi dan penyebaran data di antara kelompok \textit{disk} \parencite{katz2010raid}.

% Perlu ngejelasin masing masing raid ga?
% \begin{itemize}
%     \item RAID 0 
%     \item RAID 1 
%     \item RAID 2 
%     \item RAID 3 
%     \item RAID 4 
%     \item RAID 5 
%     \item RAID 6 
% \end{itemize}

\textit{Erasure code} adalah \textit{superset} dari sistem RAID dengan beberapa RAID menggunakan skema proteksi error yang sama. Namun, RAID tidak memberikan ketahanan yang cukup untuk tingkat kegagalan yang tinggi pada area yang luas \parencite{weatherspoon2002erasure}. Konfigurasi standar RAID tidak memiliki ketahanan lebih dari dua kegagalan, kegagalan lebih dari dua memerlukan konfigurasi \textit{nested} yang tidak terstandarkan.

Kelebihan dari RAID adalah tingkat \textit{input-output} per detik dan \textit{transfer rate} yang tinggi dibandingkan replikasi dan \textit{erasure coding} karena operasi pembacaan dan penulisan dapat dijalankan pada \textit{disk} yang berbeda secara konkuren \parencite{katz2010raid}. RAID tertentu juga menggunakan penyimpanan yang lebih sedikit untuk redundansi yang sama dibandingkan penggunaan replikasi. 
