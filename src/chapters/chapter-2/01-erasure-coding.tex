\section{Erasure Coding}
\label{sec:erasure-coding}

\textit{Erasure Coding} adalah sebuah metode proteksi data untuk sistem penyimpanan terdistribusi dengan membagi file data menjadi blok data dan \textit{parity} lalu mengkodekannya sehingga data primer dapat dipulihkan bahkan jika bagian dari data terkodenya tidak tersedia. Sistem penyimpanan terdistribusi yang dapat diskalakan secara horizontal mengandalkan \textit{erasure coding} untuk menyediakan proteksi data dengan menyimpan data terkode di beberapa \textit{drive} dan node. Jika sebuah \textit{drive} atau node gagal, data asli dapat dibangun ulang dari blok yang tersimpan pada \textit{drive} dan node lain \parencite{minio2022erasure}.

Skema \textit{erasure coding} memberikan ketahanan lebih tinggi di penyimpanan yang lebih rendah sehingga merupakan alternatif terhadap replikasi di sistem penyimpanan terdistribusi \parencite{silberstein2014lazy}. Sistem penyimpanan terdistribusi umumnya mengimplementasikan \textit{erasure coding} berdasarkan kode Reed-Solomon, namun beberapa sistem penyimpanan terdistribusi termasuk HDFS, Ceph, dan Swift menyediakan bermacam \textit{erasure coding} dan fungsi kontrol konfigurasinya \parencite{kim2021erasure}. Menyesuaikan dengan batasan penelitian, implementasi \textit{erasure coding} yang dibahas akan menggunakan kode Reed-Solomon.

% Pada skala susunan penyimpanan yang sangat besar, ruang \textit{hosting} dan daya merupakan biaya signifikan sehingga \textit{erasure coding} dapat digunakan untuk meminimalkan biaya keseluruhan sistem \parencite{manasse2009reed}. \textit{Erasure coding} dapat mengurangi biaya penyimpanan dengan lebih dari 50\% dengan mengurangi pembelian perangkat keras dan daya dari menjalankan perangkat keras tersebut \parencite{huang2012erasure}

\subsection{Kode Reed-Solomon}

Pada tahun 1960, I. Reed dan G. Solomon mengembangkan teknik pengkodean  "\textit{block code}" yang disebut kode Reed-Solomon. Saat ini, kode Reed-Solomon tetap populer karena sesuai dengan standar dan implementasi yang efisien di berbagai format perangkat keras dan perangkat lunak \parencite{minio2022erasure}. \textit{Erasure code} Reed-Solomon menyediakan teknik sederhana yang efisien untuk mengkodekan ulang informasi sehingga kegagalan beberapa disk dalam susunan disk tidak mengganggu ketersediaan data \parencite{manasse2009reed}.

% Cara pengkodean Reed-Solomon, jika terdapat $j$ digit berbeda

::TODO: How detailed should this section be?::
\subsection{Perbandingan dengan Replikasi}

Replikasi menyeluruh dengan menyalin data bit-per-bit adalah salah satu cara umum untuk meningkatkan ketahanan data. Untuk mencapai ketahanan pada tingkat kegagalan tertentu, perlu dilakukan penyalinan data dengan tingkat yang sama. Pada implementasi ini, ketahanan data yang tinggi akan meningkatkan keperluan \textit{bandwidth} dan penyimpanan dari sistem. Dengan demikian, teknik ini memerlukan \textit{bandwidth} dan penyimpanan yang tinggi  \parencite{weatherspoon2002erasure}.

Sebagai contoh, untuk menyimpan data dengan nilai ketahanan empat kegagalan, replikasi mengharuskan data untuk ditulis lima kali di tempat yang berbeda. Seperti yang sudah dijelaskan sebelumnya, penggunaan penyimpanan pada \textit{Erasure coding} lebih sedikit jika dibandingkan dengan replikasi. Sebagai contoh, \textit{erasure coding} berpendekatan kode Reed-Solomon dengan konfigurasi $(12, 4)$, yaitu dua belas fragmen data dan empat fragmen kode, akan mengurangi penyimpanan sebanyak 1.33x jika dibandingkan dengan replikasi untuk tingkat ketahanan yang sama.

Untuk meraih ketahanan pesan $t$, replikasi menggunakan penyimpanan $M$ sebanyak yang dapat dihitung pada Persamaan \ref{eq:replication_storage}.

\begin{equation}
    M = \text{Ukuran pesan} \times (t + 1)
    \label{eq:replication_storage}
\end{equation}

Dibandingkan \textit{erasure coding} yang menggunakan reed-solomon, dapat dihitung rasio penyimpanan \textit{erasure coding} dengan replikasi $R$, yaitu pada Persamaan \ref{eq:erasure_storage_simplified}.

\begin{equation}
    R = \frac{\text{Ukuran pesan} \times (t + 1)}{\text{Ukuran pesan} \times \frac{\text{k + t}}{k}}
    \label{eq:erasure_storage}
\end{equation}

\begin{equation}
    R = \frac{k + kt}{k + t}
    \label{eq:erasure_storage_simplified}
\end{equation}

% Setelah penyederhanaan, didapatkan


% Berdasarkan persamaan tersebut didapatkan bahwa penyimpanan replikasi akan selalu lebih besar dibandingkan \textit{erasure coding}. Perbedaan penyimpanan yang digunakan terpengaruhi oleh $t$, ketahanan dari sistem, dan $k$, jumlah persebaran yang digunakan pada kode Reed-solomon.
Berdasarkan persamaan tersebut didapatkan perbedaan penyimpanan yang digunakan terpengaruhi oleh $t$, ketahanan dari sistem, dan $k$, jumlah persebaran yang digunakan pada kode Reed-solomon. Untuk tingkat ketahanan yang sama, penyimpanan replikasi akan selalu lebih besar dibandingkan \textit{erasure coding} kecuali jika $k$ bernilai satu.

Yang dikorbankan penggunaan \textit{erasure coding} dibandingkan replikasi adalah kinerja. Penurunan kinerja terjadi ketika berurusan dengan data yang hilang atau \textit{offline} dan data penyimpanan yang sering diakses. Pada kasus $(12, 4)$ yang sebelumnya, untuk membangun ulang data perlu membaca dari dua belas fragmen terpisah, ini meningkatkan kemungkinan untuk mengenai penyimpanan yang sering diakses dan biaya jaringan dan \textit{input-output} sehingga menambahkan \textit{response time} dalam operasi membaca. Sementara itu, operasi membaca bisa dikembalikan tanpa operasi apapun di sebuah node sembarang pada replikasi \parencite{huang2012erasure}. Akan tetapi, Penyimpanan data yang kecil menyebabkan penggunaan \textit{bandwidth} yang lebih sedikit juga sehingga dalam kasus tertentu dengan variabel lainnya \textit{erasure coding} dapat menghasilkan operasi denngan \textit{response time} yang lebih cepat.

\subsection{Perbandingan dengan RAID}

Cara lain untuk mencapai ketahanan data tinggi adalah dengan menggunakan RAID (\textit{Redundant Array of Inexpensive Disks}). RAID adalah sebuah sistem untuk menggunakan banyak \textit{disk} untuk kebutuhan redundansi data, peningkatan kinerja, atau keduanya. Taksonomi RAID memperkenalkan skema penomoran untuk membedakan cara pengenalan redundansi dan penyebaran data di antara kelompok \textit{disk} \parencite{katz2010raid}.

% Perlu ngejelasin masing masing raid ga?
% \begin{itemize}
%     \item RAID 0 
%     \item RAID 1 
%     \item RAID 2 
%     \item RAID 3 
%     \item RAID 4 
%     \item RAID 5 
%     \item RAID 6 
% \end{itemize}

\textit{Erasure code} adalah \textit{superset} dari sistem RAID dengan beberapa RAID menggunakan skema proteksi error yang sama. Namun, RAID tidak memberikan ketahanan yang cukup untuk tingkat kegagalan yang tinggi pada area yang luas \parencite{weatherspoon2002erasure}. Konfigurasi standar RAID tidak memiliki ketahanan lebih dari dua kegagalan, kegagalan lebih dari dua memerlukan konfigurasi \textit{nested} yang tidak terstandarkan.

Kelebihan dari RAID adalah tingkat \textit{input-output} per detik dan \textit{transfer rate} yang tinggi dibandingkan replikasi dan \textit{erasure coding} karena operasi pembacaan dan penulisan dapat dijalankan pada \textit{disk} yang berbeda secara konkuren \parencite{katz2010raid}. RAID tertentu juga menggunakan penyimpanan yang lebih sedikit untuk redundansi yang sama dibandingkan penggunaan replikasi. 
