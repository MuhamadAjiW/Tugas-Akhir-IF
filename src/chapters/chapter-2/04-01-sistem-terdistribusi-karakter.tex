\subsection{Karakteristik Sistem Terdistribusi}
\label{subsection:karakteristik-sistem-terdistribusi}

Sistem terdistribusi sebagai sistem dengan komponen yang satu sama lain berkomunikasi dan berkoordinasi menggunakan \textit{message passing} memiliki karakter yang berbeda dengan sistem monolitik: konkurensi komponen, tidak adanya waktu global, dan kegagalan independen dari komponen \parencite{coulouris2012distributed}. Karakter ini menyebabkan sistem terdistribusi memiliki tantangan dan keterbatasan yang berbeda dengan sistem monolitik.

Sebuah model formal yang mendeskripsikan batasan sistem terdistribusi adalah Teorema CAP. Teorema ini menyatakan bahwa dalam merancang sebuah sistem terdistribusi terdapat tiga properti utama yang diinginkan: \textit{consistency}, \textit{availability}, dan \textit{partition tolerance}. Dari ketiga properti tersebut, mustahil sebuah sistem penyimpanan data terdistribusi untuk memberikan ketiganya secara bersamaan. Arsitek sistem harus melakukan kompromi dengan memilih dua properti yang diprioritaskan \parencite{gilbert2002brewer}.

\subsubsection{Consistency}

Definisi \textit{consistency} adalah kondisi sebuah layanan sebagai objek data atomik. Pada kondisi \textit{strong consistency}, harus ada urutan total pada semua operasi sehingga setiap operasi terlihat seolah-olah selesai pada suatu waktu. Hal ini setara dengan mengharuskan \textit{request} dari sebuah \textit{shared memory} terdistribusi untuk seolah-olah dieksekusi pada satu \textit{node} \parencite{gilbert2002brewer}.

\subsubsection{Availability}

Untuk sebuah sistem terdistribusi bersifat \textit{available}, setiap \textit{request} yang diterima oleh \textit{node} yang tidak rusak harus menghasilkan sebuah \textit{response}. Dengan itu, semua algoritma yang digunakan oleh layanan tersebut harus berhenti pada akhirnya \parencite{gilbert2002brewer}. Sifat \textit{availability} yang kuat adalah \textit{response time} yang rendah dan \textit{throughput} yang tinggi

\subsubsection{Partition Tolerance}

Untuk sebuah sistem bersifat \textit{partition tolerant}, jaringan harus diizinkan terpartisi dalam pesan yang dikirim dari satu \textit{node} ke yang lainnya. Ketika sebuah jaringan dipartisi, pesan yang dikirim dari \textit{node} di satu komponen partisi ke \textit{node} di komponen lain akan hilang. Sistem yang bersifat \textit{partition tolerant} dapat mempertahankan pesan tersebut walaupun terjadi partisi dalam jaringan \parencite{gilbert2002brewer}.
