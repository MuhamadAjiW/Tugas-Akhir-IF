\subsection{Perbandingan dengan RAID}
\label{sec:perbedaan-dengan-raid}

Cara lain untuk mencapai ketahanan data tinggi adalah dengan menggunakan RAID (\textit{Redundant Array of Inexpensive Disks}). RAID adalah sebuah sistem untuk menggunakan banyak \textit{disk} untuk kebutuhan redundansi data, peningkatan kinerja, atau keduanya. Taksonomi RAID memperkenalkan skema penomoran untuk membedakan cara pengenalan redundansi dan penyebaran data di antara kelompok \textit{disk} \parencite{katz2010raid}.

% Perlu ngejelasin masing masing raid ga?
% \begin{itemize}
%     \item RAID 0 
%     \item RAID 1 
%     \item RAID 2 
%     \item RAID 3 
%     \item RAID 4 
%     \item RAID 5 
%     \item RAID 6 
% \end{itemize}

Beberapa notasi RAID menggunakan skema proteksi error yang sama dengan \textit{Erasure coding}. Namun, RAID tidak memberikan ketahanan yang cukup untuk tingkat kegagalan yang tinggi pada area yang luas \parencite{weatherspoon2002erasure}. Konfigurasi standar RAID tidak memiliki ketahanan lebih dari dua kegagalan. Dengan demikian, kegagalan lebih dari dua menyebabkan perlunya konfigurasi \textit{nested} yang tidak terstandarkan.

Di sisi lain, kelebihan dari RAID adalah tingkat \textit{input-output} per detik dan \textit{transfer rate} yang tinggi dibandingkan replikasi dan \textit{erasure coding} karena operasi pembacaan dan penulisan dapat dijalankan pada \textit{disk} yang berbeda secara konkuren \parencite{katz2010raid}. RAID dengan notasi tertentu juga menggunakan penyimpanan yang lebih sedikit untuk redundansi yang sama dibandingkan penggunaan replikasi. \textit{Erasure coding} tidak dapat dibandingkan dengan RAID secara langsung karena variabilitas dari implementasi RAID.
