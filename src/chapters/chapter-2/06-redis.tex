\section{Redis}

Redis adalah salah satu \textit{key-value store} bersifat \textit{in-memory} yang populer dan berkinerja tinggi. Redis digunakan secara luas dalam sistem terdistribusi serta aplikasi yang membutuhkan akses data cepat. Walaupun Redis dapat menyediakan kinerja stabil sebagai layanan \textit{single node}, Redis masih menghadapi masalah dari sisi \textit{scalability} dan \textit{reliability} \parencite{chen2016towards}.

\subsection{Distribusi Data pada Redis}
Seperti yang sudah disebutkan sebelumnya, Redis merupakan \textit{key-value store} sehingga pendistribusian data dilakukan berdasarkan \textit{key} dari data yang disimpan. Untuk setiap data, Redis akan menghitung nilai hash dari \textit{key} yang dimilikinya. Setelah itu, semua nilai hash akan dipetakan secara merata ke setiap slot dengan jumlah slot adalah 16382 \parencite{chen2016towards}. Setiap slot kemudian akan dipetakan ke \textit{node} yang ada sehingga pembagian data dapat dilakukan secara merata.

\subsection{\textit{Persistence} pada Redis}
Pada Redis, data terletak pada memori utama sehingga bersifat \textit{non-persistent}. Oleh karena itu, data dapat hilang ketika terjadi \textit{crash} yang tidak terduga. Untuk mengatasi permasalahan ini, Redis memiliki beberapa teknik yang dapat diterapkan, yaitu dengan menyimpan data secara periodik sebagai \textit{image files} dengan RDB (\textit{Redis Database Backup}) atau dengan menyimpan pembaruan \textit{log} ke \textit{disk} dalam sebuah AOF (\textit{Append Only File}) untuk memastikan keamanan data \parencite{chen2016towards}.

\subsection{\textit{Replication} pada Redis}
Replikasi pada Redis dilakukan dengan mengirimkan \textit{snapshot} dari node lainnya berupa sebuah file RDB. Setelah itu, file RDB yang diterima akan dimuat ke dalam memory dari \textit{node} replika.
