\subsection{CassandrEAS: Highly Available and Storage-Efficient Distributed Key-Value Store with Erasure Coding}

Riset yang dilakukan oleh Viveck \textit{et al} pada tahun 2020 ini menyajikan protokol baru memanfaatkan kuorum dan \textit{erasure coding} untuk mengoptimasi ukuran penyimpanan serta menyediakan \textit{consistency} yang bersifat atomik. Protokol yang dikembangkan diintegrasikan dengan cara memodifikasi \textit{source code} dari Apache Cassandra, salah satu teknologi \textit{key-value store} yang populer.

Hal yang didemonstrasikan dari penelitian ini adalah bahwa \textit{erasure coding} dappat diimplementasikan secara efektif ke dalam \textit{key-value store} berbasis kuorum seperti \textit{Cassandra}. Implementasi ini mencapai penurunan biaya penyimpanan secara signifikan dengan tetap menyediakan \textit{consistency}, \textit{availability}, dan \textit{fault tolerance} yang kuat \parencite{yiu2017erasure}. Penelitian ini menyajikan pembuktian yang ekstensif terkait pencapaian tersebut. Akan tetapi, penelitian ini tidak membandingkan kinerja Cassandra yang dimodifikasi untuk \textit{erasure} coding dengan replikasi. Penelitian ini hanya membandingkan kinerja berdasarkan algoritma \textit{erasure coding} pada Cassandra berdasarkan konfigurasi dan algoritma \textit{erasure coding} yang digunakan.
