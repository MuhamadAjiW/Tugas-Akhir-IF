\section{Jadwal Pelaksanaan}
\label{sec:jadwal-pelaksanaan}
Perincian aktivitas yang dilakukan dapat dilihat pada  Tabel \ref{tab:rincian-aktivitas}. Dari aktivitas tersebut, jadwal rencana pelaksanaan adalah seperti pada Tabel \ref{tab:jadwal}.

\begin{table}[h!]
\centering
\caption{Jadwal Tugas Akhir}
\begin{tabular}{|p{1cm}|p{6cm}|p{6cm}|}
\hline
ID & Kegiatan & Hasil \\ \hline
K-1  & Melakukkan kajian pustaka & Bab II pada proposal tugas akhir \\ \hline
K-2  & Melakukan analisis permasalahan untuk topik tugas akhir & Bab I pada proposal tugas akhir \\ \hline
K-3  & Merancang solusi penyelesaian masalah & Bab III dan Bab IV pada proposal tugas akhir \\ \hline
K-4  & Melakukan seminar proposal tugas akhir & - \\ \hline
K-5  & Melakukan implementasi sistem \textit{key-value store database} untuk membandingkan \textit{erasure coding} dengan replikasi \textit{} & Sistem \textit{key-value store database} yang dapat mengumpulkan data untuk perbandingan \textit{erasure coding} dan replikasi \\ \hline
K-6  & Melakukan eksperimen untuk pengumpulan data \textit{response time} operasi pada sistem \textit{erasure coding} dan replikasi yang divariasikan sesuai rancangan & Data \textit{response time} operasi pada sistem \textit{erasure coding} dan replikasi \\ \hline
K-7  & Melakukan analisis data eksperimen & Hasil analisis \\ \hline
K-8  & Menyusun laporan tugas akhir secara keseluruhan & Laporan tugas akhir \\ \hline
K-9  & Melakukan sidang tugas akhir & - \\ \hline
\end{tabular}
\label{tab:rincian-aktivitas}
\end{table}

% TODO: Table jadwal
% \begin{table}[h!]
% \centering
% \caption{Jadwal Pelaksanaan Tugas Akhir}
% \begin{tabular}{|p{1cm}|p{6cm}|p{6cm}|}
% \hline
% ID & Kegiatan & Jadwal \\ \hline
% K-1  & Melakukkan kajian pustaka & Minggu 1 - Minggu 2 \\ \hline
% K-2  & Melakukan analisis permasalahan untuk topik tugas akhir & Minggu 3 - Minggu 4 \\ \hline
% K-3  & Merancang solusi penyelesaian masalah & Minggu 5 - Minggu 6 \\ \hline
% K-4  & Melakukan seminar proposal tugas akhir & Minggu 7 \\ \hline
% K-5  & Melakukan implementasi sistem \textit{key-value store database} untuk membandingkan \textit{erasure coding} dengan replikasi & Minggu 8 - Minggu 10 \\ \hline
% K-6  & Melakukan eksperimen untuk pengumpulan data \textit{response time} operasi pada sistem \textit{erasure coding} dan replikasi yang divariasikan sesuai rancangan & Minggu 11 - Minggu 12 \\ \hline
% K-7  & Melakukan analisis data eksperimen & Minggu 13 - Minggu 14 \\ \hline
% K-8  & Menyusun laporan tugas akhir secara keseluruhan & Minggu 15 - Minggu 16 \\ \hline
% \end{tabular}
% \label{tab:jadwal}
% \end{table}