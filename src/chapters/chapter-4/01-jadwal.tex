\section{Jadwal Pelaksanaan}
\label{sec:jadwal-pelaksanaan}
Perincian aktivitas yang dilakukan dapat dilihat pada  Tabel \ref{tab:rincian-aktivitas}. Dari aktivitas tersebut, jadwal rencana pelaksanaan adalah seperti pada Tabel \ref{tab:jadwal}.

\begin{table}[h!]
\centering
\caption{Daftar Aktivitas Tugas Akhir}
\begin{tabular}{|p{1cm}|p{6cm}|p{6cm}|}
\hline
ID & Kegiatan & Hasil \\ \hline
K-1  & Melakukkan kajian pustaka & Bab II pada proposal tugas akhir \\ \hline
K-2  & Melakukan analisis permasalahan untuk topik tugas akhir & Bab I pada proposal tugas akhir \\ \hline
K-3  & Merancang solusi penyelesaian masalah & Bab III dan Bab IV pada proposal tugas akhir \\ \hline
K-4  & Melakukan finalisasi proposal tugas akhir & Proposal tugas akhir \\ \hline
K-5  & Melakukan seminar proposal tugas akhir & - \\ \hline
K-6  & Melakukan implementasi sistem \textit{key-value store database} untuk membandingkan \textit{erasure coding} dengan replikasi \textit{} & Sistem \textit{key-value store database} yang dapat mengumpulkan data untuk perbandingan \textit{erasure coding} dan replikasi \\ \hline
K-7  & Melakukan eksperimen untuk pengumpulan data \textit{response time} operasi pada sistem \textit{erasure coding} dan replikasi yang divariasikan sesuai rancangan & Data \textit{response time} operasi pada sistem \textit{erasure coding} dan replikasi \\ \hline
K-8  & Melakukan analisis data eksperimen & Hasil analisis \\ \hline
K-9  & Menyusun laporan tugas akhir secara keseluruhan & Laporan tugas akhir \\ \hline
K-10  & Melakukan sidang tugas akhir & - \\ \hline
\end{tabular}
\label{tab:rincian-aktivitas}
\end{table}

\begin{table}
\centering
\caption{Jadwal Tugas Akhir Oktober 2024 hingga Februari 2025}
\resizebox{\textwidth}{!}{
\begin{ganttchart}[
    title/.append style={fill=black!10},
    time slot format=simple,
    x unit=0.9cm,
    y unit chart=0.7cm,
    hgrid,
    vgrid
    ]{1}{20}

    \gantttitle{2024}{12}
    \gantttitle{2025}{8} \\

    \gantttitle{Oktober}{4}
    \gantttitle{November}{4}
    \gantttitle{Desember}{4}
    \gantttitle{Januari}{4}
    \gantttitle{Februari}{4}\\
    
    \gantttitle{14}{1}
    \gantttitle{21}{1}
    \gantttitle{28}{1}
    \gantttitle{4}{1}
    \gantttitle{11}{1}
    \gantttitle{18}{1}
    \gantttitle{25}{1} 
    \gantttitle{2}{1}
    \gantttitle{9}{1}
    \gantttitle{16}{1}
    \gantttitle{23}{1}
    \gantttitle{30}{1}
    \gantttitle{6}{1}
    \gantttitle{13}{1}
    \gantttitle{20}{1}
    \gantttitle{27}{1}
    \gantttitle{3}{1}
    \gantttitle{10}{1}
    \gantttitle{17}{1}
    \gantttitle{24}{1}\\

    \ganttbar{K-1}{1}{4} \\
    \ganttbar{K-2}{5}{6} \\
    \ganttbar{K-3}{7}{12} \\
    \ganttbar{K-4}{12}{13} \\
    \ganttbar{K-5}{14}{16} \\

\end{ganttchart}
}
\end{table}

\begin{table}
\centering
\caption{Jadwal Tugas Akhir Maret 2025 hingga Juni 2025}
\resizebox{\textwidth}{!}{
\begin{ganttchart}[
    title/.append style={fill=black!10},
    time slot format=simple,
    x unit=0.9cm,
    y unit chart=0.7cm,
    hgrid,
    vgrid
    ]{1}{17}

    \gantttitle{2025}{17} \\

    \gantttitle{Maret}{5}
    \gantttitle{April}{4}
    \gantttitle{Mei}{4}
    \gantttitle{Juni}{4}\\
    
    \gantttitle{3}{1}
    \gantttitle{10}{1}
    \gantttitle{17}{1}
    \gantttitle{24}{1}
    \gantttitle{31}{1}
    \gantttitle{7}{1}
    \gantttitle{14}{1} 
    \gantttitle{21}{1}
    \gantttitle{28}{1}
    \gantttitle{5}{1}
    \gantttitle{12}{1}
    \gantttitle{19}{1}
    \gantttitle{26}{1}
    \gantttitle{2}{1}
    \gantttitle{9}{1}
    \gantttitle{16}{1}
    \gantttitle{23}{1}\\

    \ganttbar{K-6}{1}{4} \\
    \ganttbar{K-7}{5}{6} \\
    \ganttbar{K-8}{7}{12} \\
    \ganttbar{K-9}{12}{13} \\
    \ganttbar{K-10}{14}{16} \\

\end{ganttchart}
}
\end{table}

\label{tab:jadwal-part1}

