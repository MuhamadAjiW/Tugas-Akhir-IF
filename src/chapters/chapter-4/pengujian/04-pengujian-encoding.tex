\subsubsection{Pengujian encoding data menggunakan erasure coding}
\label{subsubsection:pengujian-encoding-data-erasure-coding}

Pengujian ini mencakup fungsional F-4, yaitu bahwa sistem harus dapat melakukan \textit{erasure coding} pada data yang disimpan. Pengujian dilakukan dengan kode pengujian P-4. Pengujian ini menggunakan \textit{script} pembantu oneshot.js yang sudah dijelaskan pada bagian \ref{subsubsection:implementasi-benchmark}.

Pengujian encoding data menggunakan \textit{erasure coding} dengan kode pengujian P-6 dilakukan dengan menuliskan \textit{key-value pair} ke dalam sistem lalu melakukan \textit{request} GET ke \textit{endpoint} /fragment untuk memastikan bahwa data yang disimpan telah di-\textit{encode} sesuai dengan konfigurasi yang telah ditentukan. Berikut langkah-langkah yang dilakukan dalam pengujian ini:

\begin{enumerate}
  \item Menunggu hingga sistem siap menerima \textit{request}. Konfirmasi dapat dilakukan dengan mengirim \textit{request} HTTP pada \textit{endpoint} /status dan memastikan bahwa sistem sudah memiliki \textit{leader}.
  \item Menjalankan \textit{script} oneshot.js dengan argumen \textit{write} untuk melakukan pengujian operasi dasar. \textit{Script} ini akan mengirimkan \textit{request} \textit{write} ke sistem.
  \item Mengirim \textit{request} HTTP pada \textit{endpoint} /fragment untuk memastikan bahwa sistem telah menerima \textit{request} \textit{write} dan telah menyimpan data yang dituliskan dalam bentuk \textit{fragment} yang telah di-\textit{encode}.
\end{enumerate}

Hasil yang diharapkan setelah tes dijalankan adalah sistem menyimpan \textit{log} yang berisi \textit{key-value pair} yang telah di-\textit{encode} sesuai dengan konfigurasi yang telah ditentukan. Untuk \textit{erasure coding}, log yang disimpan harus berisi \textit{key-value pair} yang telah di-\textit{encode} sesuai dengan konfigurasi yang telah ditentukan serta berbeda-beda untuk setiap \textit{node} sesuai dengan index \textit{fragment} yang diterima.

Selain pengujian yang dilakukan dengan kode P-6, pengujian untuk kebutuhan fungsional F-3 juga dilakukan dengan \textit{automated testing} pada \textit{test suite} repository OmniPaxos. Pengujian yang dilakukan pada \textit{test suite} tersebut mengetes implementasi \textit{Erasure Coding Service} yang dibuat terlepas dari sistem \textit{key-value store database}.