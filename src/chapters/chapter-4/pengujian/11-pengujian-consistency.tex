\subsubsection{Pengujian consistency}
\label{subsubsection:pengujian-consistency}

Pengujian ini mencakup non-fungsional NF-1, yaitu bahwa sistem harus menyediakan \textit{consistency} yang tinggi dengan \textit{request} ke \textit{node} manapun harus menghasilkan hasil yang sama. Pengujian dilakukan dengan kode pengujian P-12. Disebabkan kebutuhan ini sulit untuk diuji secara langsung, maka pengujian dilakukan dengan mendekati kebutuhan ini. 


Pengujian \textit{consistency} dengan kode pengujian P-13 dilakukan dengan menjalankan operasi \textit{write} dan \textit{read} dengan hampir bersamaan pada beberapa \textit{node} yang berbeda. Pengujian ini dilakukan dengan memanfaatkan \textit{script} oneshot.js yang sudah dijelaskan pada bagian \ref{subsubsection:implementasi-benchmark}. Berikut langkah-langkah yang dilakukan dalam pengujian ini:

\begin{enumerate}
    \item Menunggu hingga sistem siap menerima \textit{request}. Konfirmasi dapat dilakukan dengan mengirim \textit{request} HTTP pada \textit{endpoint} /status dan memastikan bahwa sistem sudah memiliki \textit{leader}.
    \item Secara bersamaan, menjalankan \textit{script} oneshot.js dengan argumen \textit{write} dan \textit{read} pada beberapa \textit{node} yang berbeda. Hal ini dilakukan untuk mensimulasikan \textit{request} yang hampir bersamaan ke beberapa \textit{node}.
    \item Mengirim \textit{request} HTTP pada \textit{endpoint} /log untuk memastikan bahwa sistem telah menerima \textit{request} \textit{write} dan telah menyimpan data yang dituliskan.
    \item Mengulangi pengujian dengan sistem \textit{erasure coding} dan juga dengan sistem replikasi.
\end{enumerate}

Hasil yang diharapkan adalah sistem tidak mengembalikan nilai hingga \textit{request} \textit{write} selesai. Selain itu, setiap \textit{node} pada sistem harus mengembalikan nilai \textit{key-value pair} yang sama pada operasi \textit{read}.
