\subsubsection{Pengujian konfigurasi sistem}
\label{subsubsection:pengujian-konfigurasi-sistem}

Pengujian ini mencakup fungsional F-7, yaitu bahwa sistem harus dapat dikonfigurasi untuk menggunakan \textit{erasure coding} atau replikasi tanpa mengganti konfigurasi lainnya. Pengujian dilakukan dengan kode pengujian P-9.

Pengujian konfigurasi sistem dengan kode pengujian P-9 mencakup proses menjalankan sistem dengan konfigurasi yang sama dengan \textit{flag} tambahan --erasure dan tanpa \textit{flag} tersebut seperti yang sudah dijelaskan pada Bagian \ref{subsection:setup-pengujian}. Berikut langkah-langkah yang dilakukan dalam pengujian ini:

\begin{enumerate}
	\item Menyalakan sistem dengan konfigurasi \textit{erasure coding} menggunakan \textit{flag} --erasure. Konfirmasi dapat dilakukan dengan mengirim \textit{request} HTTP pada \textit{endpoint} /status dan memastikan bahwa sistem sudah memiliki \textit{leader}.
	\item Mematikan sistem dengan menggunakan perintah stop\_all pada file scripts.sh.
	\item Menyalakan ulang sistem tanpa \textit{flag} --erasure. Konfirmasi dapat dilakukan dengan mengirim \textit{request} HTTP pada \textit{endpoint} /status dan memastikan bahwa sistem sudah memiliki \textit{leader}.
\end{enumerate}

Hasil yang diharapkan setelah tes dijalankan adalah sistem dapat berjalan dengan konfigurasi \textit{erasure coding} dan juga dapat berjalan tanpa \textit{flag} tersebut. Tujuannya adalah untuk memastikan bahwa sistem dapat berjalan dengan konfigurasi yang sama dan hanya berbeda pada penggunaan \textit{erasure coding} atau replikasi.