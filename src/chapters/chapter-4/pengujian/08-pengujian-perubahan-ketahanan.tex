\subsubsection{Pengujian perubahan konfigurasi ketahanan sistem}
\label{subsubsection:pengujian-perubahan-konfigurasi-ketahanan}

Pengujian ini mencakup fungsional F-8, yaitu bahwa sistem harus dapat mengubah konfigurasi ketahanan sistem tanpa mempengaruhi operasi lainnya. Pengujian dilakukan dengan kode pengujian P-10.

Pengujian perubahan konfigurasi ketahanan sistem dengan kode pengujian P-10 mencakup proses mengubah konfigurasi sistem dengan jumlah \textit{data shard} dan \textit{parity shard} yang berbeda-beda. Pengujian ini dilakukan dengan mengubah konfigurasi sistem pada file config.json yang sudah dijelaskan pada bagian \ref{subsection:setup-pengujian}. Berikut langkah-langkah yang dilakukan dalam pengujian ini:

\begin{enumerate}
  \item Mengubah konfigurasi sistem pada file config.json dengan jumlah \textit{data shard} dan \textit{parity shard} yang berbeda.
  \item Menyalakan ulang sistem untuk menerapkan perubahan konfigurasi. Konfirmasi dapat dilakukan dengan mengirim \textit{request} HTTP pada \textit{endpoint} /status dan memastikan bahwa sistem sudah memiliki \textit{leader}.
  \item Melakukan pengujian operasi \textit{write} dan \textit{read} untuk memastikan bahwa sistem masih berfungsi dengan baik setelah perubahan konfigurasi.
\end{enumerate}

Hasil yang diharapkan setelah tes dijalankan adalah sistem dapat berfungsi dengan baik setelah perubahan konfigurasi. Sistem harus dapat menyimpan dan mengambil data dengan jumlah \textit{data shard} dan \textit{parity shard} yang telah ditentukan.