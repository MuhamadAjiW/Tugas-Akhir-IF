\subsubsection{Pengujian availability}
\label{subsubsection:pengujian-availability}

Pengujian ini mencakup non-fungsional NF-2, yaitu bahwa sistem harus memiliki \textit{availability} yang tinggi dengan harus dapat tetap tersedia walaupun beberapa \textit{node} ada dalam kondisi gagal. Pengujian dilakukan dengan kode pengujian P-14. Pengujian dilakukan dengan menjalankan sistem, melakukan operasi \textit{write}, mematikan beberapa \textit{node}, dan melakukan operasi \textit{read} dengan beberapa \textit{node} dalam kondisi mati. Pengujian ini dilakukan dengan memanfaatkan \textit{script} oneshot.js yang sudah dijelaskan pada bagian \ref{subsubsection:implementasi-benchmark}.

Pengujian availability dengan kode pengujian P-14 dilakukan dengan langkah-langkah sebagai berikut:
\begin{enumerate}
    \item Menunggu hingga sistem siap menerima \textit{request}. Konfirmasi dapat dilakukan dengan mengirim \textit{request} HTTP pada \textit{endpoint} /status dan memastikan bahwa sistem sudah memiliki \textit{leader}.
    \item Menjalankan \textit{script} oneshot.js dengan argumen \textit{write} untuk melakukan pengujian operasi dasar. \textit{Script} ini akan mengirimkan \textit{request} \textit{write} ke sistem.
    \item Mengirim \textit{request} HTTP pada \textit{endpoint} /log untuk memastikan bahwa sistem telah menerima \textit{request} \textit{write} dan telah menyimpan data yang dituliskan.
    \item Mematikan beberapa \textit{node} dengan masuk ke GNU screen tempat proses tersebut berjalan dan mengirimkan SIGTERM ke proses tersebut.
    \item Menjalankan \textit{script} oneshot.js dengan argumen \textit{read} untuk melakukan pengujian operasi dasar. \textit{Script} ini akan mengirimkan \textit{request} \textit{read} ke sistem.
    \item Mengirim \textit{request} HTTP pada \textit{endpoint} /log untuk memastikan bahwa sistem mengembalikan nilai \textit{request} \textit{read} yang sesuai dengan nilai yang disimpan pada \textit{log}.
    \item Mengulangi pengujian dengan sistem \textit{erasure coding} dan juga dengan sistem replikasi.
\end{enumerate}

Hasil yang diharapkan setelah tes dijalankan adalah sistem mengembalikan nilai \textit{key-value pair} yang telah dituliskan sebelumnya meskipun beberapa \textit{node} dalam kondisi mati. Untuk sistem \textit{erasure coding}, sistem harus dapat mengembalikan nilai \textit{key-value pair} yang utuh dan bukan nilai \textit{fragment} yang di-\textit{encode}. Pengujian ini memastikan bahwa sistem tetap tersedia meskipun beberapa \textit{node} mengalami kegagalan.