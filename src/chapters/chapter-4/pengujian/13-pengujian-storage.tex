\subsubsection{Pengujian penyimpanan minimal}
\label{subsubsection:pengujian-penyimpanan-minimal}

Pengujian ini mencakup non-fungsional NF-3, yaitu bahwa sistem harus menggunakan penyimpanan minimal untuk skalabilitas dan efisiensi biaya. Pengujian dilakukan dengan kode pengujian P-15. Disebabkan kebutuhan ini sulit untuk diuji secara langsung, pengujian ini hanya mencakup memperlihatkan jumlah penyimpanan yang digunakan oleh sistem.

Pengujian penyimpanan minimal dengan kode pengujian P-15 dilakukan dengan melakukan operasi \textit{write} lalu melihat data yang disimpan pada sistem. Berikut langkah-langkah yang dilakukan dalam pengujian ini:

\begin{enumerate}
  \item Menunggu hingga sistem siap menerima \textit{request}. Konfirmasi dapat dilakukan dengan mengirim \textit{request} HTTP pada \textit{endpoint} /status dan memastikan bahwa sistem sudah memiliki \textit{leader}.
  \item Mengirim \textit{request} HTTP pada \textit{endpoint} /write untuk menyimpan data ke dalam sistem. Data yang disimpan dapat berupa \textit{key-value pair} yang berisi informasi yang relevan untuk pengujian.
  \item Mengirim \textit{request} HTTP pada \textit{endpoint} /log untuk memastikan bahwa sistem telah menerima \textit{request} \textit{write} dan telah menyimpan data yang dituliskan.
  \item Memeriksa ukuran penyimpanan yang digunakan oleh sistem.
  \item Mengulangi pengujian dengan sistem \textit{erasure coding} dan juga dengan sistem replikasi.
\end{enumerate}

Hasil yang diharapkan setelah tes dijalankan adalah sistem dapat menyimpan data dengan ukuran penyimpanan yang minimal. Untuk \textit{erasure coding}, log yang disimpan harus berisi \textit{key-value pair} yang telah di-\textit{encode} sesuai dengan konfigurasi yang telah ditentukan.