\subsubsection{Pengujian distribusi data}
\label{subsubsection:pengujian-distribusi-data}

Pengujian ini mencakup fungsional F-5, yaitu bahwa sistem harus dapat mendistribusikan data atau sebagian data ke \textit{node} lain untuk keperluan ketahanan data. Pengujian dilakukan dengan kode pengujian P-8. Pengujian ini menggunakan \textit{script} pembantu oneshot.js yang sudah dijelaskan pada bagian \ref{subsubsection:implementasi-benchmark}. Pengujian ini dilakukan dengan menuliskan \textit{key-value pair} ke dalam sistem. Setelah itu, dilakukan melakukan \textit{request} GET ke \textit{endpoint} /fragment ke beberapa node lain untuk sistem \textit{erasure coding} dan \textit{endpoint} /read untuk sistem replikasi. Hal ini untuk memastikan bahwa data yang disimpan telah didistribusikan ke \textit{node} lain sesuai dengan konfigurasi yang telah ditentukan.

Pengujian distribusi data dengan kode pengujian P-8 dilakukan dengan langkah-langkah sebagai berikut:
\begin{enumerate}
    \item Menunggu hingga sistem siap menerima \textit{request}. Konfirmasi dapat dilakukan dengan mengirim \textit{request} HTTP pada \textit{endpoint} /status dan memastikan bahwa sistem sudah memiliki \textit{leader}.
    \item Menjalankan \textit{script} oneshot.js dengan argumen \textit{write} untuk melakukan pengujian operasi dasar. \textit{Script} ini akan mengirimkan \textit{request} \textit{write} ke sistem.
    \item Mengirim \textit{request} HTTP pada \textit{endpoint} /fragment ke beberapa \textit{node} lain untuk memastikan bahwa sistem telah mendistribusikan data yang dituliskan dalam bentuk \textit{fragment} yang telah di-\textit{encode}. Untuk sistem replikasi, mengirim \textit{request} HTTP pada \textit{endpoint} /read ke beberapa \textit{node} lain untuk memastikan bahwa sistem telah mendistribusikan data yang dituliskan ke \textit{node} lain.
    \item Mengirim \textit{request} HTTP pada \textit{endpoint} /read untuk memastikan bahwa sistem dapat mengembalikan nilai \textit{key-value pair} yang telah didistribusikan ke \textit{node} lain.
\end{enumerate}

Hasil yang diharapkan setelah tes dijalankan adalah sistem mengembalikan nilai \textit{key-value pair} yang telah didistribusikan. Untuk sistem \textit{erasure coding}, sistem harus dapat mengembalikan nilai \textit{key-value pair} yang berbentuk \textit{fragment} ter-\textit{encode}. Untuk sistem replikasi, sistem harus dapat mengembalikan nilai \textit{key-value pair} yang utuh.