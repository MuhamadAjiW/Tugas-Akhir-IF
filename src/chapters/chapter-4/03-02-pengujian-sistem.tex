\subsection{Pengujian sistem}
\label{subsection:pengujian-sistem}

Pengujian sistem dilakukan untuk memastikan bahwa sistem yang telah diimplementasikan berfungsi sesuai dengan yang diharapkan. Kebutuhan dari sistem dapat dilihat pada bagian \ref{tab:functional-requirements} dan bagian \ref{tab:non-functional-requirements}. Pengujian akan merujuk pada kebutuhan fungsional dan non-fungsional untuk spesifikasi perilaku sistem yang diharapkan. Untuk memudahkan pemetaan dan memastikan semua pengujian telah mencakup semua kebutuhan, lampiran \ref{appendix:pemetaan-pengujian} menyediakan tabel yang memetakan kebutuhan fungsional dan non-fungsional ke pengujian yang dilakukan.

\subsubsection{Pengujian operasi dasar}
\label{subsubsection:pengujian-operasi-dasar}

Pengujian ini mencakup fungsional F-1, yaitu bahwa sistem harus dapat melakukan operasi \textit{read} dan \textit{write} pada sebuah \textit{key-value store database}. Pada pengujian ini, dilakukan \textit{setup} sistem dengan konfigurasi dengan menggunakan replikasi dan juga \textit{erasure coding}. Langkah-langkah pengujian adalah sebagai berikut:

\begin{enumerate}
  \item Melakukan \textit{setup} eksperimen dengan mengkonfigurasi sistem pada file ./etc/config.json. File sudah dijelaskan sebelumnya pada bagian \ref{subsubsection:implementasi-benchmark}.
  \item Menjalankan sistem eksperimen menggunakan \text{script.js} yang sudah dijelaskan pada bagian \textit{implementasi benchmark}. Pastikan sistem dijalankan dengan \textit{flag} --trace untuk mendapatkan informasi yang lebih lengkap mengenai operasi yang dilakukan oleh sistem.
  \item Menunggu hingga sistem siap menerima \textit{request}. Konfirmasi dapat dilakukan dengan mengirim \textit{request} HTTP pada \textit{endpoint} /status dan memastikan bahwa sistem sudah memiliki \textit{leader}.
  \item Menjalankan \textit{script} oneshot.js untuk melakukan pengujian operasi dasar. \textit{Script} ini akan mengirimkan \textit{request} \textit{write} dan \textit{read} ke sistem.
  \item Mengulangi pengujian dengan sistem \textit{erasure coding} dan juga dengan sistem replikasi.
\end{enumerate}

Hasil dari pengujian didapatkan dengan melihat \textit{log} yang dihasilkan oleh sistem. Log ini seharusnya berisi informasi bahwa sistem telah menerima \textit{request} \textit{write} dan \textit{read} serta hasil dari operasi tersebut. Implementasi dari oneshot.js dapat dilihat pada gambar \ref{fig:implementasi-oneshot}. 

\begin{figure}[ht]
    \centering
    \includegraphics[width=0.95\textwidth]{resources/chapter-4/oneshot.png}
    \caption{Implementasi oneshot.js}
    \label{fig:implementasi-oneshot}
\end{figure}

Untuk verifikasi lebih lanjut, \textit{endpoint} /log dapat digunakan untuk melihat log yang dihasilkan oleh sistem. Khususnya untuk sistem \textit{erasure coding}, \textit{endpoint} ini akan memperlihatkan informasi mengenai data yang telah di-\textit{encode}.
\subsubsection{Pengujian penyimpanan data persistent}
\label{subsubsection:pengujian-penyimpanan-data-persistent}

Pengujian ini mencakup fungsional F-2, yaitu bahwa sistem harus dapat menyimpan data secara persistent pada \textit{key-value store database}. Pengujian dilakukan dengan kode pengujian P-3. Pengujian ini menggunakan \textit{script} pembantu oneshot.js yang sudah dijelaskan pada bagian \ref{subsubsection:implementasi-benchmark}.

Pengujian penyimpanan data \textit{persistent} dengan kode pengujian P-3 dilakukan dengan menuliskan \textit{key-value pair} ke dalam sistem, kemudian mematikan sistem, menyalakan kembali sistem dengan flag --continue, dan memastikan bahwa data yang dituliskan sebelumnya masih dapat diterima. Pengujian dilakukan dengan langkah-langkah sebagai berikut:

\begin{enumerate}
  \item Menunggu hingga sistem siap menerima \textit{request}. Konfirmasi dapat dilakukan dengan mengirim \textit{request} HTTP pada \textit{endpoint} /status dan memastikan bahwa sistem sudah memiliki \textit{leader}.
  \item Menjalankan \textit{script} oneshot.js dengan argumen \textit{write} untuk melakukan pengujian operasi dasar. \textit{Script} ini akan mengirimkan \textit{request} \textit{write} ke sistem.
  \item Mengirim request HTTP pada \textit{endpoint} /log untuk memastikan bahwa sistem telah menerima \textit{request} \textit{write} dan telah menyimpan data yang dituliskan.
  \item Mematikan sistem dengan menggunakan perintah stop\_all pada file scripts.sh.
  \item Menyalakan ulang sistem dengan menggunakan perintah run\_all pada file scripts.sh dengan tambahan flag --continue. Flag ini akan membuat sistem tidak menghapus data persisten yang sudah ada sebelumnya.
  \item Menjalankan \textit{script} oneshot.js dengan argumen \textit{read} untuk melakukan pengujian operasi dasar. \textit{Script} ini akan mengirimkan \textit{request} \textit{read} ke sistem.
  \item Mengirim request HTTP pada \textit{endpoint} /log untuk memastikan bahwa sistem mengembalikan nilai \textit{request} \textit{read} yang sesuai dengan nilai yang disimpan pada \textit{log}.
  \item Mengulangi pengujian dengan sistem \textit{erasure coding} dan juga dengan sistem replikasi.
\end{enumerate}

Hasil yang diharapkan setelah tes dijalankan adalah sistem mengembalikan nilai \textit{key-value pair} yang telah dituliskan sebelumnya. Untuk \textit{erasure coding}, sistem harus dapat mengembalikan nilai \textit{key-value pair} yang utuh dan bukan nilai \textit{fragment} yang di-\textit{encode}. Pengujian ini memastikan bahwa data yang disimpan pada sistem tetap ada meskipun sistem dimatikan dan dinyalakan kembali.
\subsubsection{Pengujian pencatatan waktu transaksi}
\label{subsubsection:pengujian-pencatatan-waktu-transaksi}
\subsubsection{Pengujian encoding data menggunakan erasure coding}
\label{subsubsection:pengujian-encoding-data-erasure-coding}

Pengujian ini mencakup fungsional F-4, yaitu bahwa sistem harus dapat melakukan \textit{erasure coding} pada data yang disimpan. Pengujian dilakukan dengan kode pengujian P-4. Pengujian ini menggunakan \textit{script} pembantu oneshot.js yang sudah dijelaskan pada Bagian \ref{subsubsection:implementasi-benchmark}.

Pengujian encoding data menggunakan \textit{erasure coding} dengan kode pengujian P-6 dilakukan dengan menuliskan \textit{key-value pair} ke dalam sistem lalu melakukan \textit{request} GET ke \textit{endpoint} /fragment untuk memastikan bahwa data yang disimpan telah di-\textit{encode} sesuai dengan konfigurasi. Berikut langkah-langkah yang dilakukan dalam pengujian ini:

\begin{enumerate}
	\item Menunggu hingga sistem siap menerima \textit{request}. Konfirmasi dapat dilakukan dengan mengirim \textit{request} HTTP pada \textit{endpoint} /status dan memastikan bahwa sistem sudah memiliki \textit{leader}.
	\item Menjalankan \textit{script} oneshot.js dengan argumen \textit{write} untuk melakukan pengujian operasi dasar. \textit{Script} ini akan mengirimkan \textit{request} \textit{write} ke sistem.
	\item Mengirim \textit{request} HTTP pada \textit{endpoint} /fragment untuk memastikan bahwa sistem telah menerima \textit{request} \textit{write} dan telah menyimpan data yang dituliskan dalam bentuk \textit{fragment} yang telah di-\textit{encode}.
\end{enumerate}

Hasil yang diharapkan setelah tes dijalankan adalah sistem menyimpan \textit{log} yang berisi \textit{key-value pair} yang telah di-\textit{encode} sesuai dengan konfigurasi. Untuk \textit{erasure coding}, log yang disimpan harus berisi \textit{key-value pair} yang telah di-\textit{encode} sesuai dengan konfigurasi serta berbeda-beda untuk setiap \textit{node} sesuai dengan index \textit{fragment} yang diterima.

Selain pengujian yang dilakukan dengan kode P-6, pengujian untuk kebutuhan fungsional F-3 juga dilakukan dengan \textit{automated testing} pada \textit{test suite} repository OmniPaxos. Pengujian yang dilakukan pada \textit{test suite} tersebut mengetes implementasi \textit{Erasure Coding Service} yang dibuat terlepas dari sistem \textit{key-value store database}.
\subsubsection{Pengujian rekonstruksi data}
\label{subsubsection:pengujian-rekonstruksi-data}

Pengujian ini mencakup fungsional F-5, yaitu bahwa sistem harus dapat merekonstruksi data dari data yang disimpan menggunakan \textit{erasure coding}. Pengujian dilakukan dengan kode pengujian P-7. Pengujian ini menggunakan \textit{script} pembantu oneshot.js yang sudah dijelaskan pada Bagian \ref{subsubsection:implementasi-benchmark}. 

Pengujian rekonstruksi data dengan kode pengujian P-7 dilakukan dengan menuliskan \textit{key-value pair} ke dalam sistem lalu melakukan operasi \textit{read} pada \textit{key-value pair} tersebut untuk memastikan bahwa data yang disimpan telah di-\textit{encode} sesuai dengan konfigurasi dan dapat direkonstruksi. Berikut langkah-langkah yang dilakukan dalam pengujian ini:

\begin{enumerate}
  \item Menunggu hingga sistem siap menerima \textit{request}. Konfirmasi dapat dilakukan dengan mengirim \textit{request} HTTP pada \textit{endpoint} /status dan memastikan bahwa sistem sudah memiliki \textit{leader}.
  \item Menjalankan \textit{script} oneshot.js dengan argumen \textit{write} untuk melakukan pengujian operasi dasar. \textit{Script} ini akan mengirimkan \textit{request} \textit{write} ke sistem.
  \item Mengirim \textit{request} HTTP pada \textit{endpoint} /fragment untuk memastikan bahwa sistem telah menerima \textit{request} \textit{write} dan telah menyimpan data yang dituliskan dalam bentuk \textit{fragment} yang telah di-\textit{encode}.
  \item Mengirim \textit{request} HTTP pada \textit{endpoint} /read untuk memastikan bahwa sistem dapat mengembalikan nilai \textit{key-value pair} yang telah di-\textit{encode} sesuai dengan konfigurasi.
\end{enumerate}

Hasil yang diharapkan setelah tes dijalankan adalah sistem mengembalikan nilai \textit{key-value pair} yang telah dituliskan sebelumnya. Sistem harus dapat mengembalikan nilai \textit{key-value pair} yang utuh dan bukan nilai \textit{fragment} yang di-\textit{encode}.

Selain pengujian yang dilakukan dengan kode P-7, pengujian untuk kebutuhan fungsional F-4 juga dilakukan dengan \textit{automated testing} pada \textit{test suite} repository OmniPaxos. Pengujian yang dilakukan pada \textit{test suite} tersebut mengetes implementasi \textit{Erasure Coding Service} yang dibuat terlepas dari sistem \textit{key-value store database}.
\subsubsection{Pengujian distribusi data}
\label{subsubsection:pengujian-distribusi-data}
\subsubsection{Pengujian konfigurasi sistem}
\label{subsubsection:pengujian-konfigurasi-sistem}

Pengujian ini mencakup fungsional F-7, yaitu bahwa sistem harus dapat dikonfigurasi untuk menggunakan \textit{erasure coding} atau replikasi tanpa mengganti konfigurasi lainnya. Pengujian dilakukan dengan kode pengujian P-9. 

Pengujian konfigurasi sistem dengan kode pengujian P-9 mencakup proses menjalankan sistem dengan konfigurasi yang sama dengan \textit{flag} tambahan --erasure dan tanpa \textit{flag} tersebut seperti yang sudah dijelaskan pada Bagian \ref{subsection:setup-pengujian}. Berikut langkah-langkah yang dilakukan dalam pengujian ini:

\begin{enumerate}
  \item Menyalakan sistem dengan konfigurasi \textit{erasure coding} menggunakan \textit{flag} --erasure. Konfirmasi dapat dilakukan dengan mengirim \textit{request} HTTP pada \textit{endpoint} /status dan memastikan bahwa sistem sudah memiliki \textit{leader}.
  \item Mematikan sistem dengan menggunakan perintah stop\_all pada file scripts.sh.
  \item Menyalakan ulang sistem tanpa \textit{flag} --erasure. Konfirmasi dapat dilakukan dengan mengirim \textit{request} HTTP pada \textit{endpoint} /status dan memastikan bahwa sistem sudah memiliki \textit{leader}.
\end{enumerate}

Hasil yang diharapkan setelah tes dijalankan adalah sistem dapat berjalan dengan konfigurasi \textit{erasure coding} dan juga dapat berjalan tanpa \textit{flag} tersebut. Tujuannya adalah untuk memastikan bahwa sistem dapat berjalan dengan konfigurasi yang sama dan hanya berbeda pada penggunaan \textit{erasure coding} atau replikasi.
\subsubsection{Pengujian perubahan konfigurasi ketahanan sistem}
\label{subsubsection:pengujian-perubahan-konfigurasi-ketahanan}

Pengujian ini mencakup fungsional F-8, yaitu bahwa sistem harus dapat mengubah konfigurasi ketahanan sistem tanpa mempengaruhi operasi lainnya. Pengujian dilakukan dengan kode pengujian P-10.

Pengujian perubahan konfigurasi ketahanan sistem dengan kode pengujian P-10 mencakup proses mengubah konfigurasi sistem dengan jumlah \textit{data shard} dan \textit{parity shard} yang berbeda-beda. Pengujian ini dilakukan dengan mengubah konfigurasi sistem pada file config.json yang sudah dijelaskan pada bagian \ref{subsection:setup-pengujian}. Berikut langkah-langkah yang dilakukan dalam pengujian ini:

\begin{enumerate}
  \item Mengubah konfigurasi sistem pada file config.json dengan jumlah \textit{data shard} dan \textit{parity shard} yang berbeda.
  \item Menyalakan ulang sistem untuk menerapkan perubahan konfigurasi. Konfirmasi dapat dilakukan dengan mengirim \textit{request} HTTP pada \textit{endpoint} /status dan memastikan bahwa sistem sudah memiliki \textit{leader}.
  \item Melakukan pengujian operasi \textit{write} dan \textit{read} untuk memastikan bahwa sistem masih berfungsi dengan baik setelah perubahan konfigurasi.
\end{enumerate}

Hasil yang diharapkan setelah tes dijalankan adalah sistem dapat berfungsi dengan baik setelah perubahan konfigurasi. Sistem harus dapat menyimpan dan mengambil data dengan jumlah \textit{data shard} dan \textit{parity shard} yang telah ditentukan.
\subsubsection{Pengujian request dengan ukuran data bervariasi}
\label{subsubsection:pengujian-request-ukuran-data}

\subsubsection{Pengujian pengumpulan data}
\label{subsubsection:pengujian-pengumpulan-data}

Pengujian ini mencakup fungsional F-10, yaitu bahwa sistem harus dapat menjalankan \textit{request} secara berulang kali dan bervariasi secara otomatis untuk pengumpulan data. Pengujian dilakukan dengan kode pengujian P-12.

Pengujian pengumpulan data dengan kode pengujian P-12 dilakukan dengan menjalankan perintah run\_bench\_suite pada file scripts.sh yang sudah dijelaskan pada Bagian \ref{subsection:setup-pengujian}. Setelah itu dilakukan verifikasi dari durasi \textit{benchmark} yang dilakukan bahwa sistem menerima \textit{request} secara berulang kali. Berikut langkah-langkah yang dilakukan dalam pengujian ini:

\begin{enumerate}
	\item Menunggu hingga sistem siap menerima \textit{request}. Konfirmasi dapat dilakukan dengan mengirim \textit{request} HTTP pada \textit{endpoint} /status dan memastikan bahwa sistem sudah memiliki \textit{leader}.
	\item Menjalankan perintah run\_bench\_suite pada file scripts.sh.
	\item Memastikan bahwa sistem menerima \textit{request} secara berulang kali. Hal ini dapat dilakukan dengan melihat \textit{log} dari sistem dan hasil laporan yang dibuat oleh k6.
\end{enumerate}

Hasil yang diharapkan setelah tes dijalankan adalah sistem dapat menjalankan \textit{request} secara berulang kali secara otomatis, yaitu dalam satu \textit{benchmark}, sistem tidak hanya menerima satu \textit{request} saja. Perlu dipastikan juga dalam penjalanan run\_bench\_suite tidak terdapat balikan \textit{error} dari sistem.
\subsubsection{Pengujian consistency}
\label{subsubsection:pengujian-consistency}

Pengujian ini mencakup non-fungsional NF-1, yaitu bahwa sistem harus menyediakan \textit{consistency} yang tinggi dengan \textit{request} ke \textit{node} manapun harus menghasilkan hasil yang sama. Pengujian dilakukan dengan kode pengujian P-12. Disebabkan kebutuhan ini sulit untuk diuji secara langsung, maka pengujian dilakukan dengan mendekati kebutuhan ini.

Pengujian \textit{consistency} dengan kode pengujian P-13 dilakukan dengan menjalankan operasi \textit{write} dan \textit{read} dengan hampir bersamaan pada beberapa \textit{node} yang berbeda. Pengujian ini dilakukan dengan memanfaatkan \textit{script} oneshot.js yang sudah dijelaskan pada Bagian \ref{subsubsection:implementasi-benchmark}. Berikut langkah-langkah yang dilakukan dalam pengujian ini:

\begin{enumerate}
	\item Menunggu hingga sistem siap menerima \textit{request}. Konfirmasi dapat dilakukan dengan mengirim \textit{request} HTTP pada \textit{endpoint} /status dan memastikan bahwa sistem sudah memiliki \textit{leader}.
	\item Secara bersamaan, menjalankan \textit{script} oneshot.js dengan argumen \textit{write} dan \textit{read} pada beberapa \textit{node} yang berbeda. Hal ini dilakukan untuk mensimulasikan \textit{request} yang hampir bersamaan ke beberapa \textit{node}.
	\item Mengirim \textit{request} HTTP pada \textit{endpoint} /log untuk memastikan bahwa sistem telah menerima \textit{request} \textit{write} dan telah menyimpan data yang dituliskan.
	\item Mengulangi pengujian dengan sistem \textit{erasure coding} dan juga dengan sistem replikasi.
\end{enumerate}

Hasil yang diharapkan adalah sistem tidak mengembalikan nilai hingga \textit{request} \textit{write} selesai. Selain itu, setiap \textit{node} pada sistem harus mengembalikan nilai \textit{key-value pair} yang sama pada operasi \textit{read}.

\subsubsection{Pengujian availability}
\label{subsubsection:pengujian-availability}

Pengujian ini mencakup non-fungsional NF-2, yaitu bahwa sistem harus memiliki \textit{availability} yang tinggi dengan harus dapat tetap tersedia walaupun beberapa \textit{node} ada dalam kondisi gagal. Pengujian dilakukan dengan kode pengujian P-14.

Pengujian availability dengan kode pengujian P-14 dilakukan dengan menjalankan sistem, melakukan operasi \textit{write}, mematikan beberapa \textit{node}, dan melakukan operasi \textit{read} dengan beberapa \textit{node} dalam kondisi mati. Pengujian ini dilakukan dengan memanfaatkan \textit{script} oneshot.js yang sudah dijelaskan pada Bagian \ref{subsubsection:implementasi-benchmark}. Berikut langkah-langkah yang dilakukan dalam pengujian ini:

\begin{enumerate}
	\item Menunggu hingga sistem siap menerima \textit{request}. Konfirmasi dapat dilakukan dengan mengirim \textit{request} HTTP pada \textit{endpoint} /status dan memastikan bahwa sistem sudah memiliki \textit{leader}.
	\item Menjalankan \textit{script} oneshot.js dengan argumen \textit{write} untuk melakukan pengujian operasi dasar. \textit{Script} ini akan mengirimkan \textit{request} \textit{write} ke sistem.
	\item Mengirim \textit{request} HTTP pada \textit{endpoint} /log untuk memastikan bahwa sistem telah menerima \textit{request} \textit{write} dan telah menyimpan data yang dituliskan.
	\item Mematikan beberapa \textit{node} dengan masuk ke GNU screen tempat proses tersebut berjalan dan mengirimkan SIGTERM ke proses tersebut.
	\item Menjalankan \textit{script} oneshot.js dengan argumen \textit{read} untuk melakukan pengujian operasi dasar. \textit{Script} ini akan mengirimkan \textit{request} \textit{read} ke sistem.
	\item Mengirim \textit{request} HTTP pada \textit{endpoint} /log untuk memastikan bahwa sistem mengembalikan nilai \textit{request} \textit{read} yang sesuai dengan nilai yang disimpan pada \textit{log}.
	\item Mengulangi pengujian dengan sistem \textit{erasure coding} dan juga dengan sistem replikasi.
\end{enumerate}

Hasil yang diharapkan setelah tes dijalankan adalah sistem mengembalikan nilai \textit{key-value pair} yang telah dituliskan sebelumnya meskipun beberapa \textit{node} dalam kondisi mati. Untuk sistem \textit{erasure coding}, sistem harus dapat mengembalikan nilai \textit{key-value pair} yang utuh dan bukan nilai \textit{fragment} yang di-\textit{encode}. Pengujian ini memastikan bahwa sistem tetap tersedia meskipun beberapa \textit{node} mengalami kegagalan.
\subsubsection{Pengujian penyimpanan minimal}
\label{subsubsection:pengujian-penyimpanan-minimal}

\subsubsection{Pengujian response time rendah}
\label{subsubsection:pengujian-response-time-rendah}
