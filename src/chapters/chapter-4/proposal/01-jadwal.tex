\section{Jadwal Pelaksanaan}
\label{sec:jadwal-pelaksanaan}
Aktivitas-aktivitas yang akan dilakukan untuk memenuhi kebutuhan eksperiman adalah seperti yang dapat dilihat pada tabel \ref{tab:rincian-aktivitas}.

\begin{longtable}{|p{1cm}|p{6cm}|p{6cm}|}
\caption{Daftar Aktivitas Tugas Akhir} 
\label{tab:rincian-aktivitas} \\
\hline
\rowcolor{black!10} ID & Kegiatan & Hasil \\ \hline
K-1  & Melakukkan kajian pustaka & Bab II pada proposal tugas akhir \\ \hline
K-2  & Melakukan analisis permasalahan untuk topik tugas akhir & Bab I pada proposal tugas akhir \\ \hline
K-3  & Merancang solusi penyelesaian masalah & Bab III dan Bab IV pada proposal tugas akhir \\ \hline
K-4  & Melakukan finalisasi proposal tugas akhir & Proposal tugas akhir \\ \hline
K-5  & Melakukan seminar proposal tugas akhir & - \\ \hline
K-6  & Melakukan implementasi modul \textit{database node} pada sistem & Sistem \textit{key-value store database} yang dikembangkan memiliki bagian \textit{in-memory} dan \textit{persistent storage} sudah terintegrasi melalui \textit{controller} \\ \hline
K-7  & Melakukan implementasi \textit{storage module} pada sistem & Sistem \textit{key-value store database} yang dikembangkan dapat mensimulasikan operasi pada sistem \textit{erasure coding} dan replikasi \\ \hline
K-8  & Melakukan implementasi \textit{data collection module} pada sistem & Sistem yang dikembangkan dapat mengumpulkan data untuk perbandingan \textit{erasure coding} dan replikasi \\ \hline
K-9  & Melakukan eksperimen untuk pengumpulan data \textit{response time} operasi pada sistem \textit{erasure coding} dan replikasi yang divariasikan sesuai rancangan & Data \textit{response time} operasi pada sistem \textit{erasure coding} dan replikasi \\ \hline
K-10  & Melakukan analisis data eksperimen & Dokumen Hasil analisis \\ \hline
K-11  & Menyusun laporan tugas akhir secara keseluruhan & Laporan tugas akhir \\ \hline
K-12  & Melakukan sidang tugas akhir & - \\ \hline 
\end{longtable}

Dari aktivitas-aktivitas tersebut, jadwal perencanaan pelaksanaan masing-masing aktivitas dapat dilihat pada tabel \ref{tab:jadwal-part1} dan \ref{tab:jadwal-part2}.

\begin{table}[!ht]
\centering
\caption{Jadwal Tugas Akhir Oktober 2024--Februari 2025}
\resizebox{\textwidth}{!}{
\begin{ganttchart}[
    title/.append style={fill=black!10},
    time slot format=simple,
    x unit=0.9cm,
    y unit chart=0.7cm,
    hgrid,
    vgrid
    ]{1}{20}

    \gantttitle{2024}{12}
    \gantttitle{2025}{8} \\

    \gantttitle{Oktober}{4}
    \gantttitle{November}{4}
    \gantttitle{Desember}{4}
    \gantttitle{Januari}{4}
    \gantttitle{Februari}{4}\\
    
    \gantttitle{14}{1}
    \gantttitle{21}{1}
    \gantttitle{28}{1}
    \gantttitle{4}{1}
    \gantttitle{11}{1}
    \gantttitle{18}{1}
    \gantttitle{25}{1} 
    \gantttitle{2}{1}
    \gantttitle{9}{1}
    \gantttitle{16}{1}
    \gantttitle{23}{1}
    \gantttitle{30}{1}
    \gantttitle{6}{1}
    \gantttitle{13}{1}
    \gantttitle{20}{1}
    \gantttitle{27}{1}
    \gantttitle{3}{1}
    \gantttitle{10}{1}
    \gantttitle{17}{1}
    \gantttitle{24}{1}\\

    \ganttbar{K-1}{1}{4} \\
    \ganttbar{K-2}{5}{6} \\
    \ganttbar{K-3}{7}{12} \\
    \ganttbar{K-4}{12}{13} \\
    \ganttbar{K-5}{14}{16} \\
    \ganttbar{K-6}{14}{18} \\
    \ganttbar{K-7}{19}{20} \\

\end{ganttchart}
}
\label{tab:jadwal-part1}
\end{table}

\begin{table}[!ht]
\centering
\caption{Jadwal Tugas Akhir Maret 2025--Juni 2025}
\resizebox{\textwidth}{!}{
\begin{ganttchart}[
    title/.append style={fill=black!10},
    time slot format=simple,
    x unit=0.9cm,
    y unit chart=0.7cm,
    hgrid,
    vgrid
    ]{1}{17}

    \gantttitle{2025}{17} \\

    \gantttitle{Maret}{5}
    \gantttitle{April}{4}
    \gantttitle{Mei}{4}
    \gantttitle{Juni}{4}\\
    
    \gantttitle{3}{1}
    \gantttitle{10}{1}
    \gantttitle{17}{1}
    \gantttitle{24}{1}
    \gantttitle{31}{1}
    \gantttitle{7}{1}
    \gantttitle{14}{1} 
    \gantttitle{21}{1}
    \gantttitle{28}{1}
    \gantttitle{5}{1}
    \gantttitle{12}{1}
    \gantttitle{19}{1}
    \gantttitle{26}{1}
    \gantttitle{2}{1}
    \gantttitle{9}{1}
    \gantttitle{16}{1}
    \gantttitle{23}{1}\\

    \ganttbar{K-7}{1}{2} \\
    \ganttbar{K-8}{3}{6} \\
    \ganttbar{K-9}{7}{8} \\
    \ganttbar{K-10}{9}{10} \\
    \ganttbar{K-11}{11}{14} \\
    \ganttbar{K-12}{15}{17} \\

\end{ganttchart}
}
\label{tab:jadwal-part2}
\end{table}

% \begin{table}[!ht]
%     \centering
%     \caption{Jadwal Tugas Akhir Oktober 2024--Februari 2025}
%     \resizebox{\textwidth}{!}{
%     \begin{ganttchart}[
%         title/.append style={fill=black!10},
%         time slot format=simple,
%         x unit=0.9cm,
%         y unit chart=0.7cm,
%         hgrid,
%         vgrid
%         ]{1}{25}
    
%         \gantttitle{2025}{25} \\
    
%         \gantttitle{Januari}{4}
%         \gantttitle{Februari}{4}
%         \gantttitle{Maret}{5}
%         \gantttitle{April}{4}
%         \gantttitle{Mei}{4}
%         \gantttitle{Juni}{4} \\
        
%         \gantttitle{6}{1}
%         \gantttitle{13}{1}
%         \gantttitle{20}{1}
%         \gantttitle{27}{1}
%         \gantttitle{3}{1}
%         \gantttitle{10}{1}
%         \gantttitle{17}{1} 
%         \gantttitle{24}{1}
%         \gantttitle{3}{1}
%         \gantttitle{10}{1}
%         \gantttitle{17}{1}
%         \gantttitle{24}{1}
%         \gantttitle{31}{1}
%         \gantttitle{7}{1}
%         \gantttitle{14}{1}
%         \gantttitle{21}{1}
%         \gantttitle{28}{1}
%         \gantttitle{5}{1}
%         \gantttitle{12}{1}
%         \gantttitle{19}{1}
%         \gantttitle{26}{1}
%         \gantttitle{2}{1}
%         \gantttitle{9}{1}
%         \gantttitle{16}{1}
%         \gantttitle{23}{1} \\
    
%         \ganttbar{Identifikasi Permasalahan}{1}{1} \\
%         \ganttbar{Perancangan eksperimen}{2}{4} \\
%         \ganttbar{Implementasi Sistem}{2}{14} \\
%         \ganttbar{Eksperimen dan Pengujian}{15}{16} \\
%         \ganttbar{Analisis dan Evaluasi}{17}{18} \\
%         \ganttbar{Menyusun Laporan}{19}{22} \\
%         \ganttbar{Sidang Akhir}{23}{25} \\
    
% \end{ganttchart}
% }
% \label{tab:jadwal-worded}
% \end{table}

% \pagebreak
