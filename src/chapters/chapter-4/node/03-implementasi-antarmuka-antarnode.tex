\subsubsection{Implementasi Antarmuka Komunikasi Antarnode}
\label{subsubsection:implementasi-antarmuka-komunikasi-antarnode}

Antarmuka komunikasi antarnode diimplementasikan secara manual menggunakan protokol TCP. Disebabkan struktur komponen konsensus yang modular dan tidak terintegrasi dengan komponen jaringan secara langsung, diperlukan antarmuka khusus untuk menghubungkan komponen komponen konsensus agar dapat berkomunikasi dengan \textit{node} lain dalam \textit{cluster}. Peran dari komponen ini adalah sebagai perantara komunikasi tersebut, yaitu bertanggung jawab untuk mengirim dan menerima pesan dari \textit{node} lain dalam \textit{cluster}.

Implementasi antarmuka komunikasi antarnode diimplementasikan dalam tiga fungsi utama:
\begin{enumerate}
  \item send\_omnipaxos\_message
  
  Fungsi ini digunakan untuk mengirim pesan OmniPaxos ke \textit{node} lain dalam \textit{cluster}. Pesan yang dikirim akan diserialisasi menggunakan pustaka bincode dan dikirim melalui soket TCP. Fungsi ini mengirimkan banyak pesan sekaligus sebagai optimasi.

  \item send\_omnipaxos\_message\_single
  
  Fungsi \textit{wrapper} untuk send\_omnipaxos\_message tetapi hanya mengirim satu pesan saja.

  \item receive\_omnipaxos\_message
  
  Fungsi ini digunakan untuk menerima pesan OmniPaxos dari \textit{node} lain dalam \textit{cluster}. Pesan yang diterima akan deserialisasi menggunakan pustaka bincode dan dapat diteruskan ke dalam kelas OmniPaxos melalui antarmuka \textit{message passing}.

  \item run\_tcp\_loop
  
  Fungsi ini adalah \textit{event loop} yang berjalan terus-menerus untuk menerima pesan dari \textit{node} lain dalam \textit{cluster}. Fungsi ini akan memanggil receive\_omnipaxos\_message untuk menerima pesan dan kemudian meneruskan pesan tersebut ke dalam kelas OmniPaxos melalui antarmuka \textit{message passing}.
\end{enumerate}

Pesan yang dikirim dan diterima berbentuk \textit{binary} dan menggunakan pustaka bincode dan serde untuk \textit{serialization}. Pesan yang dikirimkan mengikuti struktur yang telah dijelaskan sebelumnya pada Tabel \ref{tab:omnipaxos-messages} dan Tabel \ref{tab:omnipaxos-operations}. Hampir semua pesan akan diteruskan ke dalam kelas OmniPaxos melalui antarmuka \textit{message passing} terlebih dahulu kecuali pesan dengan OperationType::RECONSTRUCT dan OperationType::GET. Pesan ini mengambil nilai dari \textit{key-value pair} yang ada pada \text{storage} dan menggunakan logika di luar OmniPaxos dan konsensus dilakukan pada kelas \textit{Node}. Hal ini disebabkan struktur sistem \textit{database} yang terpisah dari komponen konsensus. Struktur sistem tersebut dapat dilihat pada Gambar \ref{fig:node-structure}. 