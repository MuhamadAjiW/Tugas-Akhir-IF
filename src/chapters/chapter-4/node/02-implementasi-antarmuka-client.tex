\subsubsection{Implementasi Antarmuka Client}
\label{subsubsection:implementasi-antarmuka-client}

Antarmuka \textit{client} diimplementasikan menggunakan pustaka actix-web. Pustaka ini menyediakan \textit{web server} dengan antarmuka HTTP. Peran dari komponen ini adalah untuk menerima operasi dari \textit{client}. Gambaran peran dari komponen ini dapat dilihat pada Gambar \ref{fig:client-interface-component} dan pemetaan endpoint dapat dilihat pada Tabel \ref{tab:client-interface-endpoint}.

\begin{table}[h]
    \centering
    \caption{Endpoint dari Antarmuka Client}
    \resizebox{\textwidth}{!}{
        \begin{tabular}{|l|l|p{5cm}|p{5cm}|}
            \hline
            \rowcolor{black!10} Method & Endpoint & Fungsi & Request Body \\ \hline
            GET & \text{/health} & Memeriksa status kesehatan server & -- \\ \hline
            GET & \text{/status} & Mendapatkan informasi node & -- \\ \hline
            GET & \text{/cluster} & Mendapatkan status cluster & -- \\ \hline
            GET & \text{/docs} & Mendapatkan dokumentasi API & -- \\ \hline
            POST & \text{/get} & Mengambil nilai berdasarkan kunci & \{\texttt{"key": str}\} \\ \hline
            POST & \text{/put} & Menyimpan pasangan kunci-nilai & \{\texttt{"key": str, "value": str}\} \\ \hline
            POST & \text{/delete} & Menghapus pasangan kunci-nilai & \{\texttt{"key": str}\} \\ \hline
            POST & \text{/bulk} & Melakukan beberapa operasi dalam satu permintaan (maksimum 100 operasi) & \{\texttt{"key": str}\} \\ \hline
        \end{tabular}
    }
    \label{tab:client-interface-endpoint}
\end{table}

Antarmuka \textit{client} menyediakan endpoint untuk operasi \textit{bulk} untuk optimasi kinerja, namun antarmuka ini tidak digunakan dalam pengumpulan data. Hal ini karena pengumpulan data dilakukan dengan cara satu per satu lebih mencerminkan cara kerja \textit{client} yang sebenarnya. Selain itu, antarmuka ini juga menyediakan endpoint untuk mendapatkan status dari node dan cluster, serta dokumentasi API yang berguna untuk \textit{troubleshooting} dan pengembangan. Implementasi antarmuka \textit{client} terdapat dalam fungsi run\_http\_loop pada kelas Node.
