\subsubsection{Implementasi Memory Store}
\label{subsubsection:implementasi-memory-store}

Komponen \textit{Memory Store} diimplementasikan menggunakan pustaka moka sesuai dengan rancangan yang telah ditetapkan pada Bagian \ref{subsubsection:detail-in-memory-store}. Komponen ini berperan sebagai \textit{cache} dari \textit{key-value store database} yang diimplementasikan. Untuk kedua jenis penyimpanan, baik replikasi maupun \textit{erasure coding}, komponen ini menyimpan data utuh. Tujuan utamanya adalah untuk meningkatkan kinerja sistem dengan mengurangi latensi akses \textit{disk} yang lebih lambat dibandingkan akses \textit{memory}. Selain itu, untuk \textit{erasure coding}, komponen ini juga mengurangi operasi rekonstruksi data yang memerlukan waktu lama untuk operasi \textit{read}.

Pada implementasinya, komponen ini terdapat pada kelas KvMemory yang merupakan \textit{wrapper} dari pustaka moka. Kelas ini menyediakan antarmuka yang terdapat pada Tabel \ref{tab:memory-store-interface}.

\begin{table}[h]
	\centering
	\caption{Antarmuka Persistent Store}
	\resizebox{\textwidth}{!}{
		\begin{tabular}{|l|p{7cm}|p{4cm}|}
			\hline
			\rowcolor{black!10} Method & Fungsi                                                                                                 & Arguments                             \\ \hline
			get                        & Mengambil nilai berdasarkan kunci. Nilai adalah berupa \textit{binary array} untuk mempercepat operasi & \{\texttt{"key": str}\}               \\ \hline
			set                        & Menyimpan \textit{key-value pair}.                                                                     & \{\texttt{"key": str, "value": str}\} \\ \hline
			remove                     & Menghapus \textit{key-value pair}                                                                      & \{\texttt{"key": str}\}               \\ \hline
		\end{tabular}
	}
	\label{tab:memory-store-interface}
\end{table}

Berbeda dengan implementasi \textit{Persistent Store}, komponen \textit{Memory Store} tidak menggunakan kelas \textit{generic}. Hal ini karena peran komponen ini sebagai \textit{cache} tidak memerlukan banyak pengolahan data setelah diambil dari \textit{memory}. Dengan menggunakan tipe data \textit{binary array}, data yang diambil dari \textit{memory} dapat langsung dikirimkan ke jaringan tanpa perlu diserialisasi ulang.
