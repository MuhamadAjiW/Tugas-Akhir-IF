\subsubsection{Implementasi Konsensus}
\label{subsubsection:implementasi-konsensus}

Berdasarkan perbandingan alternatif yang dilakukan pada bagian \ref{subsubsection:pemilihan-algoritma-konsensus}, algoritma konsensus yang dipilih adalah OmniPaxos. Disebabkan OmniPaxos pada awalnya dikembangkan dalam bahasa pemrograman Rust dan bersifat \textit{open source}, maka implementasi algoritma konsensus ini dilakukan dengan menggunakan melakukan \textit{fork} pada pustaka OmniPaxos yang telah tersedia.

Pustaka OmniPaxos yang dimodifikasi untuk keperluan implementasi sistem ini merupakan OmniPaxos versi 0.2.2. Sesuai dengan perancangan pada bagian \ref{subsubsection:detail-komponen-konsensus}, perubahan yang dilakukan pada pustaka ini mencakup integrasi dari implementasi \textit{erasure coding}. Berikut adalah penjelasan mengenai modifikasi pada pustaka OmniPaxos:

\begin{enumerate}
  \item \textit{OmniPaxosEC}: Kelas baru yang merupakan modifikasi dari kelas OmniPaxos. Kelas ini mengimplementasikan logika konsensus dengan dukungan \textit{erasure coding} dengan distribusi data yang telah di-\textit{erasure coding} ke beberapa \textit{node}. 
  \item \textit{Erasure Coding Service}: Kelas tambahan yang bertanggung jawab untuk mengelola proses \textit{erasure coding} pada data yang disimpan. Implementasi ini menggunakan algoritma Reed-Solomon untuk melakukan \textit{erasure coding} dan rekonstruksi data seperti yang sudah dijelaskan pada bagian \ref{subsubsection:pemilihan-algoritma-erasure-coding}. Implementasi ini dilakukan dengan menggunakan pustaka Rust reed-solomon-erasure versi 6.0 dan menggunakan fitur simd-accel. Pustaka ini digunakan dalam kelas baru \textit{OmniPaxosEC}.
\end{enumerate}

Beberapa poin tambahan terkait mekanisme komponen konsensus yang diimplementasikan adalah sebagai berikut:
\begin{itemize}
  \item Penyebaran data yang telah di-\textit{erasure coding} dilakukan pada saat fase \textit{accept} pada algoritma konsensus. Data yang telah di-\textit{erasure coding} akan disebarkan ke \textit{node} lain sebagai bagian dari proses replikasi data dengan data yang disimpan masing-masing \textit{node} berbeda satu sama lain tergantung pada konfigurasi \textit{erasure coding} yang telah ditentukan.
  \item Kuorum dari konsensus yang dilakukan pada sistem \textit{erasure coding} adalah sejumlah \textit{data-shard} yang ditentukan pada konfigurasi \textit{erasure coding} service. Hal ini membuat konsensus bersifat mirip dengan konsensus pada sistem yang dijelaskan di bagian \ref{subsection:paxos-erasure}.
  \item Data \textit{erasure coding} yang disebarkan memiliki \textit{metadata} yang berisi informasi tentang \textit{index} dari \textit{shard} yang diterima untuk keperluan rekonstruksi data.
  \item Data \textit{erasure coding} yang disebarkan memiliki \textit{metadata} berisi informasi versi. Hal ini diperlukan untuk memastikan bahwa data yang diterima oleh \textit{node} lain adalah versi yang sama dengan node yang lain. Jika versi data yang diterima berbeda, maka \textit{node} tidak menggunakan data tersebut dalam rekonstruksi data.
  \item Pengujian terkait konsensus dilakukan dengan menggunakan \textit{test} yang sudah ada pada pustaka OmniPaxos ditambah dengan \textit{test} baru terkait \textit{erasure coding}. Pengujian ini dijalankan dengan lingkungan \textit{test} yang terdapat pada bahasa pemrograman Rust. Pengujian ini mencakup pengujian konsensus dasar, pengujian replikasi data, dan pengujian rekonstruksi data menggunakan \textit{erasure coding}. Hasil dari pengujian ini menunjukkan bahwa implementasi konsensus dapat berfungsi dengan baik dan memenuhi kebutuhan sistem.
  \item Disebabkan tidak ada perubahan dari pustaka OmniPaxos selain penambahan \textit{erasure coding}, maka komponen ini memiliki perilaku yang sama dengan pustaka OmniPaxos yang asli. Pustaka OmniPaxos yang asli memiliki sifat modular dengan keterhubungan dilakukan dengan \textit{message passing}, tanpa \textit{built-in interface} untuk komunikasi menggunakan jaringan. Selain itu, pustaka OmniPaxos yang asli juga memiliki \textit{interface} yang memungkinkan \textit{log} untuk disimpan dalam berbagai jenis penyimpanan. Dalam implementasi sistem, penyimpanan \textit{log} dilakukan menggunakan konektor ke \textit{rocksdb}. Tempat penyimpanan \textit{log} berbeda dengan tempat penyimpanan data yang digunakan untuk \textit{persistent store}.
\end{itemize}

Dalam pengujian, sistem yang sama persis dapat dibangun dengan hanya membedakan kelas konsensus yang digunakan. Sistem replikasi akan menggunakan kelas OmniPaxos yang asli, sedangkan sistem \textit{erasure coding} akan menggunakan kelas OmniPaxosEC yang telah dimodifikasi. Hal ini sesuai dengan kebutuhan sistem pada bagian \ref{subsection:system-requirements}, yaitu sistem dapat dikonfigurasi untuk menggunakan replikasi ataupun \textit{erasure coding} tanpa mengganti konfigurasi lainnya.

OmniPaxos termodifikasi dibangun dalam bentuk \textit{package} atau dalam bahasa pemrograman Rust, menggunakan \textit{crate}. Pustaka ini dapat digunakan sebagai pustaka independen yang dapat diimpor ke dalam proyek Rust lainnya. Implementasi OmniPaxos berada pada repositori berbeda dengan repositori induk sistem.