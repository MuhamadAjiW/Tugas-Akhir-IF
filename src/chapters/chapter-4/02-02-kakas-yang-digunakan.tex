\subsection{Kakas yang Digunakan}
\label{subsection:kakas-yang-digunakan}

Kakas yang digunakan dalam implementasi ini adalah sebagai berikut:
\begin{enumerate}
  \item Bahasa Pemrograman:
    \begin{enumerate}
      \item Rust: Digunakan untuk mengimplementasikan \textit{Node} dan sistem \textit{distributed key-value store database} secara keseluruhan.
      \item Bash: Digunakan untuk \textit{shell scripting} dalam automatisasi, melakukan pengelolaan node dan bagian dari \textit{Data Collector}.
      \item JavaScript: Digunakan untuk menulis skrip k6 dalam \textit{Data Collector} untuk melakukan pengujian beban dan pengumpulan data.
    \end{enumerate}
  \item Kakas dalam bahasa Rust:
    \begin{enumerate}
      \item bincode: Digunakan untuk serialisasi dan deserialisasi biner dalam komunikasi antarnode dan penyimpanan.
      \item tokio: \textit{runtime} async untuk \textit{Non-blocking I/O} berkinerja tinggi.
      \item serde: Digunakan sebagai kerangka kerja serialisasi dan deserialisasi biner bersamaan dengan bincode untuk transformasi data.
      \item actix-web: Digunakan sebagai \textit{framework web} untuk HTTP \textit{server} pada node.
      \item tracing, tracing-subscriber, tracing-appender: Digunakan untuk instrumentasi \textit{logging} dan \textit{tracing}.
      \item clap: Digunakan untuk parsing argumen baris perintah untuk binari node.
      \item moka: Digunakan untuk \textit{in-memory store}.
      \item rand: Digunakan sebagai generator bilangan acak untuk keperluan konsensus.
      \item rocksdb: Digunakan sebagai \textit{storage engine} untuk persistent key-value store lokal setiap node.
    \end{enumerate}
  \item Kakas dalam bahasa JavaScript:
    \begin{enumerate}
      \item k6: Digunakan untuk mengeksekusi skenario \textit{read/write} dan menghasilkan metrik JSON.
    \end{enumerate}
  \item Kakas lainnya:
    \begin{enumerate}
      \item GNU Screen: Digunakan untuk menjalankan beberapa session terminal node terpisah yang dapat dipantau.
      \item jq: Digunakan untuk memanipulasi dan memvalidasi konfigurasi JSON sebelum menjalankan sistem.
      \item iptables: Digunakan untuk menandai paket TCP dan mengontrol \textit{bandwidth}.
      \item tc: Digunakan untuk mengatur antrian paket dan pembatasan bandwidth (netem) pada loopback.
      \item sysstat: Digunakan untuk merekam statistik CPU secara \textit{real-time} saat pengujian berlangsung.
    \end{enumerate}
\end{enumerate}