\section{Lingkungan}
\label{sec:environment}

Pengembangan sistem \textit{distributed key-value store database} dilakukan pada lingkungan komputer lokal dalam \textit{virtual machine}. Berikut merupakan penjelasan implementasi dari masing-masing bagian secara terperinci yang menjadi dua bagian, yaitu perangkat lunak dan perangkat keras. Pengembangan dilakukan pada \textit{virtual machine} karena sistem memiliki pustaka yang tidak tersedia pada sistem operasi yang digunakan pada komputer lokal. Selain itu, \textit{virtual machine} juga memberikan kemudahan untuk mengatur konfigurasi sistem yang dibutuhkan untuk pengembangan sistem ini tanpa mempengaruhi sistem operasi utama yang digunakan pada komputer lokal.

Seperti yang dijelaskan pada bagian \ref{subsubsection:pilihan-bahasa-pemrograman}, implementasi \textit{distributed key-value store database} dibuat dengan menggunakan bahasa pemrograman Rust dan menggunakan pustaka bahasa tersebut. Masing-masing node dalam sistem berjalan dalam sebuah proses yang terpisah. Walaupun begitu, komunikasi antar node dilakukan dengan \textit{socket} menggunakan protokol \textit{TCP} sehingga setiap node dapat berjalan pada komputer yang berbeda jika diperlukan. Spesifikasi dari lingkungan yang digunakan untuk pengembangan sistem adalah sebagai berikut:

\begin{enumerate}
  \item \textbf{Perangkat Keras}
    \begin{enumerate}
      \item Prosesor: AMD Ryzen 7 5800H
      \item Memori: 64 GB RAM
      \item Penyimpanan: M.2 NVMe SSD 2 TB
    \end{enumerate}
  \item \textbf{Perangkat Lunak}
    \begin{enumerate}
      \item Sistem Operasi \textit{Host}: Windows 11 Home x64 Build 26100.4349
      \item Virtual Machine:
      \begin{enumerate}
        \item Sistem Operasi \textit{Guest}: Debian 12 Bookworm (Version 12.10)
        \item Virtualization: VMWare Workstation 17 Pro
        \item Alokasi Virtual CPU: 12 vCPU
        \item Alokasi Memori Virtual: 52 GB RAM
        \item Alokasi Penyimpanan Virtual: 250 GB
      \end{enumerate}
      \item Bahasa Pemrograman:
      \begin{enumerate}
        \item Rust (Versi: 1.84.1 linux)
        \item Bash (Versi: 5.2.15(1)-release)
        \item Javascript (Versi Node.js: v22.11.0)
      \end{enumerate}
      \item Dependensi Rust:
      \begin{enumerate}
        \item bincode (versi: 1.3.3)
        \item tokio (versi: 1, fitur: full)
        \item serde (versi: 1.0, fitur: derive)
        \item actix web (versi: 4.10.2)
        \item tracing (versi: 0.1)
        \item tracing-subscriber (versi: 0.3)
        \item tracing-appender (versi: 0.2)
        \item clap (versi: 4, fitur: derive)
        \item moka (versi: 0.12, fitur: future)
        \item rand (versi: 0.9.1)
        \item rocksdb (versi: 0.21.0)
      \end{enumerate}
      \item Dependensi Javascript:
      \begin{enumerate}
        \item k6 (Versi: 1.0.0)
      \end{enumerate}
      \item Dependensi lainnya:
      \begin{enumerate}
        \item GNU screen (Versi: 4.09.0)
        \item jq (Versi: 1.6)
        \item iptables (Versi: 1.8.9)
        \item tc (dari paket iproute2) (Versi: 6.1.0)
        \item sysstat (mpstat) (Versi: 12.6.1)
      \end{enumerate}
    \end{enumerate}
\end{enumerate}