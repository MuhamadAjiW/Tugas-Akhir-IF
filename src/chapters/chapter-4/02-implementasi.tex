\section{Implementasi}
\label{sec:implementation}

Bagian ini menjelaskan implementasi dari sistem \textit{distributed key-value store database} yang telah dibangun secara terperinci. Seperti yang telah dijelaskan pada bagian \ref{subsection:rancangan-struktural}, sistem dikembangkan sebagai sebuah \textit{Node} yang dibangun secara modular. Selain itu, dibangun juga \textit{Data Collector} yang berfungsi sebagai pengumpul data dari pengujian sistem. Penjelasan bagian ini dimulai dari batasan implementasi, dilanjutkan dengan kakas yang digunakan dalam pembuatan sistem, dan diakhiri dengan penjelasan mengenai implementasi dari masing-masing komponen sistem.

Mengulangi definisi yang disebutkan pada bagian \ref{sec:rancangan}, \textit{Node} adalah satuan fungsional utama yang berperan sebagai entitas dalam sistem \textit{database} terdistribusi yang dikembangkan dan \textit{Data Collector} adalah bagian dari eksperimen yang bertugas untuk mengumpulkan data eksperimen.

\subsection{Batasan Implementasi}
\label{subsection:batasan-implementasi}

Berikut adalah batasan implementasi yang ditetapkan dalam pengembangan sistem:
\begin{enumerate}
  \item Semua batasan masalah yang telah ditetapkan pada bagian \ref{sec:batasan-masalah}.
  \item Operasi yang didukung hanya pembacaan (\textit{read}) dan penulisan (\textit{write}); tidak mengakomodasi operasi penghapusan atau transaksi multi-kunci atau transaksi kompleks lainnya.
  \item Pengujian performa hanya dilakukan pada lingkungan lokal (loopback) dengan alat \texttt{k6} dan \texttt{mpstat}.
  \item Konfigurasi \textit{membership} dari sebuah \textit{cluster} bersifat statis; tidak mengakomodasi perubahan \textit{node} pada waktu berjalan.
  \item Konfigurasi \textit{erasure coding} (jumlah data shard dan parity shard) bersifat statis; tidak mengakomodasi penyesuaian dinamis pada waktu berjalan.
  \item Tidak ada pengujian toleransi kesalahan yang kompleks; kegagalan yang diuji hanya \textit{crash-stop}.
\end{enumerate}


\subsection{Kakas yang Digunakan}
\label{subsection:kakas-yang-digunakan}

Kakas yang digunakan dalam implementasi ini adalah sebagai berikut:
\begin{enumerate}
	\item Bahasa Pemrograman:
	      \begin{enumerate}
		      \item Rust: Digunakan untuk mengimplementasikan \textit{Node} dan sistem \textit{distributed key-value store database} secara keseluruhan.
		      \item Bash: Digunakan untuk \textit{shell scripting} dalam automatisasi, melakukan pengelolaan node dan bagian dari \textit{Data Collector}.
		      \item JavaScript: Digunakan untuk menulis skrip k6 dalam \textit{Data Collector} untuk melakukan pengujian beban dan pengumpulan data.
	      \end{enumerate}
	\item Kakas dalam bahasa Rust:
	      \begin{enumerate}
		      \item bincode: Digunakan untuk serialisasi dan deserialisasi biner dalam komunikasi antarnode dan penyimpanan.
		      \item tokio: \textit{runtime} async untuk \textit{Non-blocking I/O} berkinerja tinggi.
		      \item serde: Digunakan sebagai kerangka kerja serialisasi dan deserialisasi biner bersamaan dengan bincode untuk transformasi data.
		      \item actix-web: Digunakan sebagai \textit{framework web} untuk HTTP \textit{server} pada node.
		      \item tracing, tracing-subscriber, tracing-appender: Digunakan untuk instrumentasi \textit{logging} dan \textit{tracing}.
		      \item clap: Digunakan untuk parsing argumen baris perintah untuk binari node.
		      \item moka: Digunakan untuk \textit{in-memory store}.
		      \item rand: Digunakan sebagai generator bilangan acak untuk keperluan konsensus.
		      \item rocksdb: Digunakan sebagai \textit{storage engine} untuk persistent key-value store lokal setiap node.
	      \end{enumerate}
	\item Kakas dalam bahasa JavaScript:
	      \begin{enumerate}
		      \item k6: Digunakan untuk mengeksekusi skenario \textit{read/write} dan menghasilkan metrik JSON.
	      \end{enumerate}
	\item Kakas lainnya:
	      \begin{enumerate}
		      \item GNU Screen: Digunakan untuk menjalankan beberapa session terminal node terpisah yang dapat dipantau.
		      \item jq: Digunakan untuk memanipulasi dan memvalidasi konfigurasi JSON sebelum menjalankan sistem.
		      \item iptables: Digunakan untuk menandai paket TCP dan mengontrol \textit{bandwidth}.
		      \item tc: Digunakan untuk mengatur antrian paket dan pembatasan bandwidth (netem) pada loopback.
		      \item sysstat: Digunakan untuk merekam statistik CPU secara \textit{real-time} saat pengujian berlangsung.
	      \end{enumerate}
\end{enumerate}
\subsection{Implementasi Node}
\label{subsection:implementasi-node}

Implementasi \textit{Node} dilakukan dengan mengikuti rancangan yang telah ditetapkan pada bagian \ref{subsection:rancangan-struktural}. Komponen-komponen rancangan tersebut diimplementasikan dalam bahasa pemrograman Rust. Rincian implementasi dari masing-masing komponen dijelaskan pada subbab pada bagian ini.

\subsubsection{Implementasi Konsensus}
\label{subsubsection:implementasi-konsensus}
\subsubsection{Implementasi Antarmuka Client}
\label{subsubsection:implementasi-antarmuka-client}

Antarmuka \textit{client} diimplementasikan menggunakan pustaka actix-web. Pustaka ini menyediakan \textit{web server} dengan antarmuka HTTP. Peran dari komponen ini adalah untuk menerima operasi dari \textit{client}. Gambaran peran dari komponen ini dapat dilihat pada Gambar \ref{fig:client-interface-component} dan pemetaan endpoint dapat dilihat pada Tabel \ref{tab:client-interface-endpoint}.

\begin{table}[h]
	\centering
	\caption{Endpoint dari Antarmuka Client}
	\resizebox{\textwidth}{!}{
		\begin{tabular}{|l|l|p{5cm}|p{5cm}|}
			\hline
			\rowcolor{black!10} Method & Endpoint        & Fungsi                                                                  & Request Body                          \\ \hline
			GET                        & \text{/health}  & Memeriksa status kesehatan server                                       & --                                    \\ \hline
			GET                        & \text{/status}  & Mendapatkan informasi node                                              & --                                    \\ \hline
			GET                        & \text{/cluster} & Mendapatkan status cluster                                              & --                                    \\ \hline
			GET                        & \text{/docs}    & Mendapatkan dokumentasi API                                             & --                                    \\ \hline
			POST                       & \text{/get}     & Mengambil nilai berdasarkan kunci                                       & \{\texttt{"key": str}\}               \\ \hline
			POST                       & \text{/put}     & Menyimpan pasangan kunci-nilai                                          & \{\texttt{"key": str, "value": str}\} \\ \hline
			POST                       & \text{/delete}  & Menghapus pasangan kunci-nilai                                          & \{\texttt{"key": str}\}               \\ \hline
			POST                       & \text{/bulk}    & Melakukan beberapa operasi dalam satu permintaan (maksimum 100 operasi) & \{\texttt{"key": str}\}               \\ \hline
		\end{tabular}
	}
	\label{tab:client-interface-endpoint}
\end{table}

Antarmuka \textit{client} menyediakan endpoint untuk operasi \textit{bulk} untuk optimasi kinerja, namun antarmuka ini tidak digunakan dalam pengumpulan data. Hal ini karena pengumpulan data dilakukan dengan cara satu per satu lebih mencerminkan cara kerja \textit{client} yang sebenarnya. Selain itu, antarmuka ini juga menyediakan endpoint untuk mendapatkan status dari node dan cluster, serta dokumentasi API yang berguna untuk \textit{troubleshooting} dan pengembangan. Implementasi antarmuka \textit{client} terdapat dalam fungsi run\_http\_loop pada kelas Node.

\subsubsection{Implementasi Antarmuka Komunikasi Antarnode}
\label{subsubsection:implementasi-antarmuka-komunikasi-antarnode}

Antarmuka komunikasi antarnode diimplementasikan secara manual menggunakan protokol TCP. Disebabkan struktur komponen konsensus yang modular dan tidak terintegrasi dengan komponen jaringan secara langsung, diperlukan antarmuka khusus untuk menghubungkan komponen komponen konsensus agar dapat berkomunikasi dengan \textit{node} lain dalam \textit{cluster}. Peran dari komponen ini adalah sebagai perantara komunikasi tersebut, yaitu bertanggung jawab untuk mengirim dan menerima pesan dari \textit{node} lain dalam \textit{cluster}.

Implementasi antarmuka komunikasi antarnode diimplementasikan dalam tiga fungsi utama:
\begin{enumerate}
  \item send\_omnipaxos\_message
  
  Fungsi ini digunakan untuk mengirim pesan OmniPaxos ke \textit{node} lain dalam \textit{cluster}. Pesan yang dikirim akan diserialisasi menggunakan pustaka bincode dan dikirim melalui soket TCP. Fungsi ini mengirimkan banyak pesan sekaligus sebagai optimasi.

  \item send\_omnipaxos\_message\_single
  
  Fungsi \textit{wrapper} untuk send\_omnipaxos\_message tetapi hanya mengirim satu pesan saja.

  \item receive\_omnipaxos\_message
  
  Fungsi ini digunakan untuk menerima pesan OmniPaxos dari \textit{node} lain dalam \textit{cluster}. Pesan yang diterima akan deserialisasi menggunakan pustaka bincode dan dapat diteruskan ke dalam kelas OmniPaxos melalui antarmuka \textit{message passing}.

  \item run\_tcp\_loop
  
  Fungsi ini adalah \textit{event loop} yang berjalan terus-menerus untuk menerima pesan dari \textit{node} lain dalam \textit{cluster}. Fungsi ini akan memanggil receive\_omnipaxos\_message untuk menerima pesan dan kemudian meneruskan pesan tersebut ke dalam kelas OmniPaxos melalui antarmuka \textit{message passing}.
\end{enumerate}

Pesan yang dikirim dan diterima berbentuk \textit{binary} dan menggunakan pustaka bincode dan serde untuk \textit{serialization}. Pesan yang dikirimkan mengikuti struktur yang telah dijelaskan sebelumnya pada Tabel \ref{tab:omnipaxos-messages} dan Tabel \ref{tab:omnipaxos-operations}. Hampir semua pesan akan diteruskan ke dalam kelas OmniPaxos melalui antarmuka \textit{message passing} terlebih dahulu kecuali pesan dengan OperationType::RECONSTRUCT dan OperationType::GET. Pesan ini mengambil nilai dari \textit{key-value pair} yang ada pada \text{storage} dan menggunakan logika di luar OmniPaxos dan konsensus dilakukan pada kelas \textit{Node}. Hal ini disebabkan struktur sistem \textit{database} yang terpisah dari komponen konsensus. Struktur sistem tersebut dapat dilihat pada Gambar \ref{fig:node-structure}. 
\subsubsection{Implementasi Persistent Store}
\label{subsubsection:implementasi-persistent-store}

\subsubsection{Implementasi Memory Store}
\label{subsubsection:implementasi-memory-store}

Komponen \textit{Memory Store} diimplementasikan menggunakan pustaka moka sesuai dengan rancangan yang telah ditetapkan pada bagian \ref{subsubsection:detail-in-memory-store}. Komponen ini berperan sebagai \textit{cache} dari \textit{key-value store database} yang diimplementasikan. Untuk kedua jenis penyimpanan, baik replikasi maupun \textit{erasure coding}, komponen ini menyimpan data utuh. Tujuan utamanya adalah untuk meningkatkan kinerja sistem dengan mengurangi latensi akses \textit{disk} yang lebih lambat dibandingkan akses \textit{memory}. Selain itu, untuk \textit{erasure coding}, komponen ini juga mengurangi operasi rekonstruksi data yang memerlukan waktu lama untuk operasi \textit{read}.

Pada implementasinya, komponen ini terdapat pada kelas KvMemory yang merupakan \textit{wrapper} dari pustaka moka. Kelas ini menyediakan antarmuka yang terdapat pada tabel \ref{tab:memory-store-interface}.

\begin{table}[h]
    \centering
    \caption{Antarmuka Persistent Store}
    \resizebox{\textwidth}{!}{
        \begin{tabular}{|l|p{7cm}|p{4cm}|}
            \hline
            \rowcolor{black!10} Method & Fungsi & Arguments \\ \hline
            get & Mengambil nilai berdasarkan kunci. Nilai adalah berupa \textit{binary array} untuk mempercepat operasi & \{\texttt{"key": str}\} \\ \hline
            set & Menyimpan \textit{key-value pair}. & \{\texttt{"key": str, "value": str}\} \\ \hline
            remove & Menghapus \textit{key-value pair} & \{\texttt{"key": str}\} \\ \hline
        \end{tabular}
    }
    \label{tab:memory-store-interface}
\end{table}

Berbeda dengan implementasi \textit{Persistent Store}, komponen \textit{Memory Store} tidak menggunakan kelas \textit{generic}. Hal ini karena peran komponen ini sebagai \textit{cache} tidak memerlukan banyak pengolahan data setelah diambil dari \textit{memory}. Dengan menggunakan tipe data \textit{binary array}, data yang diambil dari \textit{memory} dapat langsung dikirimkan ke jaringan tanpa perlu diserialisasi ulang.

\subsection{Implementasi Data Collector}
\label{subsection:implementasi-data-collector}

Implementasi \textit{Data Collector} dilakukan dengan mengikuti rancangan yang telah ditetapkan pada bagian \ref{subsection:rancangan-struktural}. Berbeda dengan \textit{Node}, \textit{data collector} tidak berbentuk satu program yang utuh, melainkan terdiri dari beberapa komponen skrip yang saling berinteraksi. Skrip-skrip tersebut bertugas untuk mengumpulkan data dari pengujian sistem yang dilakukan pada \textit{Node} dan menyimpannya dalam format yang dapat dianalisis lebih lanjut. Komponen-komponen rancangan tersebut diimplementasikan dalam bahasa pemrograman Bash dan JavaScript. Rincian implementasi dari masing-masing komponen dijelaskan pada subbab pada bagian ini.

\subsubsection{Implementasi Komponen Testing}
\label{subsubsection:implementasi-testing}
\subsubsection{Implementasi Komponen Log Management}
\label{subsubsection:implementasi-log-management}
