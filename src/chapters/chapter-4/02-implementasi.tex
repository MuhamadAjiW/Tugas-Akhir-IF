\section{Implementasi}
\label{sec:implementation}

Bagian ini menjelaskan implementasi dari sistem \textit{distributed key-value store database} yang telah dibangun secara terperinci. Seperti yang telah dijelaskan pada bagian \ref{subsection:rancangan-struktural}, sistem dikembangkan sebagai sebuah \textit{Node} yang dibangun secara modular. Selain itu, dibangun juga \textit{Data Collector} yang berfungsi sebagai pengumpul data dari pengujian sistem. Penjelasan bagian ini dimulai dari batasan implementasi, dilanjutkan dengan kakas yang digunakan dalam pembuatan sistem, dan diakhiri dengan penjelasan mengenai implementasi dari masing-masing komponen sistem.

Mengulangi definisi yang disebutkan pada bagian \ref{sec:rancangan}, \textit{Node} adalah satuan fungsional utama yang berperan sebagai entitas dalam sistem \textit{database} terdistribusi yang dikembangkan dan \textit{Data Collector} adalah bagian dari eksperimen yang bertugas untuk mengumpulkan data eksperimen.

\subsection{Batasan Implementasi}
\label{subsection:batasan-implementasi}

Berikut adalah batasan implementasi yang ditetapkan dalam pengembangan sistem:
\begin{enumerate}
  \item Semua batasan masalah yang telah ditetapkan pada bagian \ref{sec:batasan-masalah}.
  \item Operasi yang didukung hanya pembacaan (\textit{read}) dan penulisan (\textit{write}); tidak mengakomodasi operasi penghapusan atau transaksi multi-kunci atau transaksi kompleks lainnya.
  \item Pengujian performa hanya dilakukan pada lingkungan lokal (loopback) dengan alat \texttt{k6} dan \texttt{mpstat}.
  \item Konfigurasi \textit{membership} dari sebuah \textit{cluster} bersifat statis; tidak mengakomodasi perubahan \textit{node} pada waktu berjalan.
  \item Konfigurasi \textit{erasure coding} (jumlah data shard dan parity shard) bersifat statis; tidak mengakomodasi penyesuaian dinamis pada waktu berjalan.
  \item Tidak ada pengujian toleransi kesalahan yang kompleks; kegagalan yang diuji hanya \textit{crash-stop}.
\end{enumerate}


\subsection{Kakas yang Digunakan}
\label{subsection:kakas-yang-digunakan}

Kakas yang digunakan dalam implementasi ini adalah sebagai berikut:
\begin{enumerate}
  \item Bahasa Pemrograman:
    \begin{enumerate}
      \item Rust: Digunakan untuk mengimplementasikan \textit{Node} dan sistem \textit{distributed key-value store database} secara keseluruhan.
      \item Bash: Digunakan untuk \textit{shell scripting} dalam automatisasi, melakukan pengelolaan node dan bagian dari \textit{Data Collector}.
      \item JavaScript: Digunakan untuk menulis skrip k6 dalam \textit{Data Collector} untuk melakukan pengujian beban dan pengumpulan data.
    \end{enumerate}
  \item Kakas dalam bahasa Rust:
    \begin{enumerate}
      \item bincode: Digunakan untuk serialisasi dan deserialisasi biner dalam komunikasi antarnode dan penyimpanan.
      \item tokio: \textit{runtime} async untuk \textit{Non-blocking I/O} berkinerja tinggi.
      \item serde: Digunakan sebagai kerangka kerja serialisasi dan deserialisasi biner bersamaan dengan bincode untuk transformasi data.
      \item actix-web: Digunakan sebagai \textit{framework web} untuk HTTP \textit{server} pada node.
      \item tracing, tracing-subscriber, tracing-appender: Digunakan untuk instrumentasi \textit{logging} dan \textit{tracing}.
      \item clap: Digunakan untuk parsing argumen baris perintah untuk binari node.
      \item moka: Digunakan untuk \textit{in-memory store}.
      \item rand: Digunakan sebagai generator bilangan acak untuk keperluan konsensus.
      \item rocksdb: Digunakan sebagai \textit{storage engine} untuk persistent key-value store lokal setiap node.
    \end{enumerate}
  \item Kakas dalam bahasa JavaScript:
    \begin{enumerate}
      \item k6: Digunakan untuk mengeksekusi skenario \textit{read/write} dan menghasilkan metrik JSON.
    \end{enumerate}
  \item Kakas lainnya:
    \begin{enumerate}
      \item GNU Screen: Digunakan untuk menjalankan beberapa session terminal node terpisah yang dapat dipantau.
      \item jq: Digunakan untuk memanipulasi dan memvalidasi konfigurasi JSON sebelum menjalankan sistem.
      \item iptables: Digunakan untuk menandai paket TCP dan mengontrol \textit{bandwidth}.
      \item tc: Digunakan untuk mengatur antrian paket dan pembatasan bandwidth (netem) pada loopback.
      \item sysstat: Digunakan untuk merekam statistik CPU secara \textit{real-time} saat pengujian berlangsung.
    \end{enumerate}
\end{enumerate}
\subsection{Implementasi Node}
\label{subsection:implementasi-node}

Implementasi \textit{Node} dilakukan dengan mengikuti rancangan yang telah ditetapkan pada Bagian \ref{subsection:rancangan-struktural}. Komponen-komponen rancangan tersebut diimplementasikan dalam bahasa pemrograman Rust. Rincian implementasi dari masing-masing komponen dijelaskan pada subbab pada bagian ini.

\subsubsection{Implementasi Konsensus}
\label{subsubsection:implementasi-konsensus}

Berdasarkan perbandingan alternatif yang dilakukan pada Bagian \ref{subsubsection:pemilihan-algoritma-konsensus}, algoritma konsensus yang dipilih adalah OmniPaxos. Disebabkan OmniPaxos pada awalnya dikembangkan dalam bahasa pemrograman Rust dan bersifat \textit{open source}, maka implementasi algoritma konsensus ini dilakukan dengan menggunakan melakukan \textit{fork} pada pustaka OmniPaxos yang telah tersedia.

Pustaka OmniPaxos yang dimodifikasi untuk keperluan implementasi sistem ini merupakan OmniPaxos versi 0.2.2. Dalam implementasinya, OmniPaxos memiliki beberapa jenis pesan yang digunakan untuk komunikasi antar \textit{node} dalam \textit{cluster}. Pesan-pesan ini mencakup pesan untuk fase \textit{prepare}, \textit{accept}, dan \textit{commit} yang merupakan bagian dari algoritma konsensus OmniPaxos. Tabel \ref{tab:omnipaxos-messages} menunjukkan jenis-jenis pesan yang digunakan dalam omnipaxos serta deskripsi penggunaan dari pesan tersebut. Pesan-pesan ini berbentuk \textit{binary} dan menggunakan pustaka serde untuk \textit{serialization}.
Sesuai dengan perancangan pada Bagian \ref{subsubsection:detail-komponen-konsensus}, perubahan yang dilakukan pada pustaka ini mencakup integrasi dari implementasi \textit{erasure coding}. Berikut adalah penjelasan mengenai modifikasi pada pustaka OmniPaxos:

\begin{enumerate}
  \item \textit{OmniPaxosEC}: Kelas baru yang merupakan modifikasi dari kelas OmniPaxos. Kelas ini mengimplementasikan logika konsensus dengan dukungan \textit{erasure coding} dengan distribusi data yang telah di-\textit{erasure coding} ke beberapa \textit{node}. 
  \item \textit{Erasure Coding Service}: Kelas tambahan yang bertanggung jawab untuk mengelola proses \textit{erasure coding} pada data yang disimpan. Implementasi ini menggunakan algoritma Reed-Solomon untuk melakukan \textit{erasure coding} dan rekonstruksi data seperti yang sudah dijelaskan pada Bagian \ref{subsubsection:pemilihan-algoritma-erasure-coding}. Implementasi ini dilakukan dengan menggunakan pustaka Rust reed-solomon-erasure versi 6.0 dan menggunakan fitur simd-accel. Pustaka ini digunakan dalam kelas baru \textit{OmniPaxosEC}.
\end{enumerate}

\begin{table}[h]
    \centering
    \caption{Jenis Pesan OmniPaxos}
    \resizebox{\textwidth}{!}{
        \begin{tabular}{|l|p{11cm}|}
            \hline
            \rowcolor{black!10} Tipe Pesan & Deskripsi \\ \hline
            PrepareReq & Pesan permintaan dari follower untuk meminta leader mengirim ulang pesan Prepare setelah crash-recovery atau pesan hilang. \\ \hline
            Prepare & Pesan dari leader yang baru terpilih untuk memulai fase Prepare dalam konsensus Paxos, berisi informasi ballot, indeks yang diputuskan, dan status log. \\ \hline
            Promise & Pesan balasan dari follower kepada leader sebagai respons terhadap Prepare, berisi janji untuk tidak menerima proposal dengan ballot lebih rendah dan informasi sinkronisasi log. \\ \hline
            AcceptSync & Pesan dari leader untuk mensinkronisasi log semua replika dalam fase persiapan, berisi pembaruan log untuk sinkronisasi follower dengan leader. \\ \hline
            AcceptDecide & Pesan dari leader berisi entri yang akan direplikasi dan indeks yang diputuskan terbaru dalam fase accept. \\ \hline
            Accepted & Pesan konfirmasi dari follower kepada leader bahwa entri telah diterima dan disimpan di log lokal. \\ \hline
            NotAccepted & Pesan penolakan dari follower kepada leader ketika entri tidak dapat diterima karena follower sudah berjanji pada leader dengan ballot lebih tinggi. \\ \hline
            Decide & Pesan dari leader kepada follower untuk memutuskan entri hingga indeks tertentu dalam log, menandakan bahwa entri tersebut telah mencapai konsensus. \\ \hline
            ProposalForward & Pesan untuk meneruskan proposal klien dari follower kepada leader, berisi daftar entri yang akan direplikasi. \\ \hline
            Compaction & Pesan permintaan untuk kompaksi log, dapat berupa Trim (menghapus entri hingga indeks tertentu) atau Snapshot (membuat snapshot log). \\ \hline
            AcceptStopSign & Pesan dari leader kepada follower untuk menerima StopSign yang menandakan akhir konfigurasi cluster, digunakan untuk rekonfigurasi. \\ \hline
            ForwardStopSign & Pesan untuk meneruskan StopSign dari follower kepada leader, digunakan dalam proses rekonfigurasi cluster. \\ \hline
        \end{tabular}
    }
    \label{tab:omnipaxos-messages}
\end{table}

Menyesuaikan dengan keperluan implementasi \textit{Key-value store database}, nilai yang disimpan dan disetujui dalam konsensus adalah nilai dari \textit{key-value pair} yang disimpan dalam sistem. Untuk mengoperasikan \textit{database} tersebut, masing-masing \textit{node} juga mengirimkan jenis operasi yang perlu dilakukan untuk \textit{key-value pair} tersebut dengan implementasi enum OperationType. Jenis operasi yang dikirimkan dapat dilihat pada Tabel \ref{tab:omnipaxos-operations}. Operasi-operasi ini akan diterapkan pada \textit{key-value store} yang ada pada masing-masing \textit{node} setelah nilai disepakati.

\begin{table}[h]
    \centering
    \caption{Jenis Operasi untuk Pesan Database}
    \resizebox{\textwidth}{!}{
        \begin{tabular}{|l|p{9cm}|}
            \hline
            \rowcolor{black!10} Tipe Operasi & Deskripsi \\ \hline
            OperationType::NULL & Operasi \textit{placeholder} atau fungsi kontrol yang tidak berhubungan dengan \textit{database} \\ \hline
            OperationType::GET & Operasi untuk mengambil nilai dari \textit{key-value pair} yang ditentukan. \\ \hline
            OperationType::RECONSTRUCT & Operasi untuk melakukan rekonstruksi data pada \textit{key-value pair} untuk \textit{erasure coding}. \\ \hline
            OperationType::SET & Operasi untuk menyimpan atau memperbarui nilai pada \textit{key-value pair} yang ditentukan. \\ \hline
            OperationType::DELETE & Operasi untuk menghapus \textit{key value pair} yang ditentukan dari \textit{database}. \\ \hline
          \end{tabular}
    }
    \label{tab:omnipaxos-operations}
\end{table}


Beberapa poin tambahan terkait mekanisme komponen konsensus yang diimplementasikan adalah sebagai berikut:
\begin{enumerate}
  \item Penyebaran data yang telah di-\textit{erasure coding} dilakukan pada saat fase \textit{accept} pada algoritma konsensus. Data yang telah di-\textit{erasure coding} akan disebarkan ke \textit{node} lain sebagai bagian dari proses replikasi data dengan data yang disimpan masing-masing \textit{node} berbeda satu sama lain tergantung pada konfigurasi \textit{erasure coding} yang telah ditentukan.
  \item Kuorum dari konsensus yang dilakukan pada sistem \textit{erasure coding} adalah sejumlah \textit{data-shard} yang ditentukan pada konfigurasi \textit{erasure coding} service. Hal ini membuat konsensus bersifat mirip dengan konsensus pada sistem yang dijelaskan di Bagian \ref{subsection:paxos-erasure}.
  \item Data \textit{erasure coding} yang disebarkan memiliki \textit{metadata} yang berisi informasi tentang \textit{index} dari \textit{shard} yang diterima untuk keperluan rekonstruksi data.
  \item Data \textit{erasure coding} yang disebarkan memiliki \textit{metadata} berisi informasi versi. Hal ini diperlukan untuk memastikan bahwa data yang diterima oleh \textit{node} lain adalah versi yang sama dengan node yang lain. Jika versi data yang diterima berbeda, maka \textit{node} tidak menggunakan data tersebut dalam rekonstruksi data.
  \item Pengujian terkait konsensus dilakukan dengan menggunakan \textit{test} yang sudah ada pada pustaka OmniPaxos ditambah dengan \textit{test} baru terkait \textit{erasure coding}. Pengujian ini dijalankan dengan lingkungan \textit{test} yang terdapat pada bahasa pemrograman Rust. Pengujian ini mencakup pengujian konsensus dasar, pengujian replikasi data, dan pengujian rekonstruksi data menggunakan \textit{erasure coding}. Hasil dari pengujian ini menunjukkan bahwa implementasi konsensus dapat berfungsi dengan baik dan memenuhi kebutuhan sistem.
  \item Disebabkan tidak ada perubahan dari pustaka OmniPaxos selain penambahan \textit{erasure coding}, maka komponen ini memiliki perilaku yang sama dengan pustaka OmniPaxos yang asli. Pustaka OmniPaxos yang asli memiliki sifat modular dengan keterhubungan dilakukan dengan \textit{message passing}, tanpa \textit{built-in interface} untuk komunikasi menggunakan jaringan. Selain itu, pustaka OmniPaxos yang asli juga memiliki \textit{interface} yang memungkinkan \textit{log} untuk disimpan dalam berbagai jenis penyimpanan. Dalam implementasi sistem, penyimpanan \textit{log} dilakukan menggunakan konektor ke \textit{rocksdb}. Tempat penyimpanan \textit{log} berbeda dengan tempat penyimpanan data yang digunakan untuk \textit{persistent store}.
\end{enumerate}

Dalam pengujian, sistem yang sama persis dapat dibangun dengan hanya membedakan kelas konsensus yang digunakan. Sistem replikasi akan menggunakan kelas OmniPaxos yang asli, sedangkan sistem \textit{erasure coding} akan menggunakan kelas OmniPaxosEC yang telah dimodifikasi. Hal ini sesuai dengan kebutuhan sistem pada Bagian \ref{subsection:system-requirements}, yaitu sistem dapat dikonfigurasi untuk menggunakan replikasi ataupun \textit{erasure coding} tanpa mengganti konfigurasi lainnya.

OmniPaxos termodifikasi dibangun dalam bentuk \textit{package} atau dalam bahasa pemrograman Rust, menggunakan \textit{crate}. Pustaka ini dapat digunakan sebagai pustaka independen yang dapat diimpor ke dalam proyek Rust lainnya. Implementasi OmniPaxos berada pada repositori berbeda dengan repositori induk sistem. Struktur internal implementasi konsensus dapat dilihat pada Gambar \ref{fig:consensus-component-structure}
\subsubsection{Implementasi Antarmuka Client}
\label{subsubsection:implementasi-antarmuka-client}

Antarmuka \textit{client} diimplementasikan menggunakan pustaka actix-web. Pustaka ini menyediakan \textit{web server} dengan antarmuka HTTP. Peran dari komponen ini adalah untuk menerima operasi dari \textit{client}. Gambaran peran dari komponen ini dapat dilihat pada Gambar \ref{fig:client-interface-component} dan pemetaan endpoint dapat dilihat pada Tabel \ref{tab:client-interface-endpoint}.

\begin{table}[h]
	\centering
	\caption{Endpoint dari Antarmuka Client}
	\resizebox{\textwidth}{!}{
		\begin{tabular}{|l|l|p{5cm}|p{5cm}|}
			\hline
			\rowcolor{black!10} Method & Endpoint        & Fungsi                                                                  & Request Body                          \\ \hline
			GET                        & \text{/health}  & Memeriksa status kesehatan server                                       & --                                    \\ \hline
			GET                        & \text{/status}  & Mendapatkan informasi node                                              & --                                    \\ \hline
			GET                        & \text{/cluster} & Mendapatkan status cluster                                              & --                                    \\ \hline
			GET                        & \text{/docs}    & Mendapatkan dokumentasi API                                             & --                                    \\ \hline
			POST                       & \text{/get}     & Mengambil nilai berdasarkan kunci                                       & \{\texttt{"key": str}\}               \\ \hline
			POST                       & \text{/put}     & Menyimpan pasangan kunci-nilai                                          & \{\texttt{"key": str, "value": str}\} \\ \hline
			POST                       & \text{/delete}  & Menghapus pasangan kunci-nilai                                          & \{\texttt{"key": str}\}               \\ \hline
			POST                       & \text{/bulk}    & Melakukan beberapa operasi dalam satu permintaan (maksimum 100 operasi) & \{\texttt{"key": str}\}               \\ \hline
		\end{tabular}
	}
	\label{tab:client-interface-endpoint}
\end{table}

Antarmuka \textit{client} menyediakan endpoint untuk operasi \textit{bulk} untuk optimasi kinerja, namun antarmuka ini tidak digunakan dalam pengumpulan data. Hal ini karena pengumpulan data dilakukan dengan cara satu per satu lebih mencerminkan cara kerja \textit{client} yang sebenarnya. Selain itu, antarmuka ini juga menyediakan endpoint untuk mendapatkan status dari node dan cluster, serta dokumentasi API yang berguna untuk \textit{troubleshooting} dan pengembangan. Implementasi antarmuka \textit{client} terdapat dalam fungsi run\_http\_loop pada kelas Node.

\subsubsection{Implementasi Antarmuka Komunikasi Antarnode}
\label{subsubsection:implementasi-antarmuka-komunikasi-antarnode}

\subsubsection{Implementasi Persistent Store}
\label{subsubsection:implementasi-persistent-store}

Komponen \textit{Persistent Store} diimplementasikan menggunakan pustaka rocksdb sesuai dengan rancangan yang telah ditetapkan pada bagian \ref{subsubsection:detail-persistent-store}. Komponen ini berperan sebagai tempat penyimpanan utama dari \textit{key-value store database} yang diimplementasikan. Data yang di-\textit{encode} atau direplikasi disimpan pada komponen ini.

Pada implementasinya, komponen ini terdapat pada kelas KvPersistent yang merupakan \textit{wrapper} dari pustaka rocksdb. Kelas ini menyediakan antarmuka yang terdapat pada tabel \ref{tab:persistent-store-interface}.

\begin{table}[h]
    \centering
    \caption{Antarmuka Persistent Store}
    \resizebox{\textwidth}{!}{
        \begin{tabular}{|l|p{7cm}|p{4cm}|}
            \hline
            \rowcolor{black!10} Method & Fungsi & Arguments \\ \hline
            get & Mengambil nilai berdasarkan kunci. Nilai adalah berupa tipe \textit{generic} yang diberikan ketika membuat \textit{object} & \{\texttt{"key": str}\} \\ \hline
            set & Menyimpan \textit{key-value pair}. Pada \textit{erasure coding} hanya menyimpan \textit{fragment} & \{\texttt{"key": str, "value": str}\} \\ \hline
            remove & Menghapus \textit{key-value pair} & \{\texttt{"key": str}\} \\ \hline
        \end{tabular}
    }
    \label{tab:persistent-store-interface}
\end{table}

Kelas ini dibuat sebagai \textit{generic} dengan data yang disimpan dapat diatur menjadi sebuah kelas tertentu. Hal ini memungkinkan komponen \textit{Persistent Store} untuk digunakan pada berbagai jenis data yang berbeda. Penggunaan \textit{erasure coding} memerlukan metadata tambahan pada data berupa \textit{version} dan \textit{index} seperti yang telah dijelaskan pada bagian \ref{subsubsection:implementasi-konsensus}. Data ini berbeda dengan replikasi yang tidak membutuhkan metadata apapun.
\subsubsection{Implementasi Memory Store}
\label{subsubsection:implementasi-memory-store}

Komponen \textit{Memory Store} diimplementasikan menggunakan pustaka moka sesuai dengan rancangan yang telah ditetapkan pada Bagian \ref{subsubsection:detail-in-memory-store}. Komponen ini berperan sebagai \textit{cache} dari \textit{key-value store database} yang diimplementasikan. Untuk kedua jenis penyimpanan, baik replikasi maupun \textit{erasure coding}, komponen ini menyimpan data utuh. Tujuan utamanya adalah untuk meningkatkan kinerja sistem dengan mengurangi latensi akses \textit{disk} yang lebih lambat dibandingkan akses \textit{memory}. Selain itu, untuk \textit{erasure coding}, komponen ini juga mengurangi operasi rekonstruksi data yang memerlukan waktu lama untuk operasi \textit{read}.

Pada implementasinya, komponen ini terdapat pada kelas KvMemory yang merupakan \textit{wrapper} dari pustaka moka. Kelas ini menyediakan antarmuka yang terdapat pada Tabel \ref{tab:memory-store-interface}.

\begin{table}[h]
	\centering
	\caption{Antarmuka Persistent Store}
	\resizebox{\textwidth}{!}{
		\begin{tabular}{|l|p{7cm}|p{4cm}|}
			\hline
			\rowcolor{black!10} Method & Fungsi                                                                                                 & Arguments                             \\ \hline
			get                        & Mengambil nilai berdasarkan kunci. Nilai adalah berupa \textit{binary array} untuk mempercepat operasi & \{\texttt{"key": str}\}               \\ \hline
			set                        & Menyimpan \textit{key-value pair}.                                                                     & \{\texttt{"key": str, "value": str}\} \\ \hline
			remove                     & Menghapus \textit{key-value pair}                                                                      & \{\texttt{"key": str}\}               \\ \hline
		\end{tabular}
	}
	\label{tab:memory-store-interface}
\end{table}

Berbeda dengan implementasi \textit{Persistent Store}, komponen \textit{Memory Store} tidak menggunakan kelas \textit{generic}. Hal ini karena peran komponen ini sebagai \textit{cache} tidak memerlukan banyak pengolahan data setelah diambil dari \textit{memory}. Dengan menggunakan tipe data \textit{binary array}, data yang diambil dari \textit{memory} dapat langsung dikirimkan ke jaringan tanpa perlu diserialisasi ulang.

\subsection{Implementasi Data Collector}
\label{subsection:implementasi-data-collector}

Implementasi \textit{Data Collector} dilakukan dengan mengikuti rancangan yang telah ditetapkan pada bagian \ref{subsection:rancangan-struktural}. Berbeda dengan \textit{Node}, \textit{data collector} tidak berbentuk satu program yang utuh, melainkan terdiri dari beberapa komponen skrip yang saling berinteraksi. Skrip-skrip tersebut bertugas untuk mengumpulkan data dari \textit{benchmark} sistem yang dilakukan pada \textit{Node} dan menyimpannya dalam format yang dapat dianalisis lebih lanjut. Selain itu, komponen ini juga berinteraksi dengan pengguna untuk mengatur konfigurasi eksperimen dan mengelola sistem yang akan diuji. Implementasi dari \textit{Data Collector} banyaknya terdapat pada file scripts.sh. Berikut adalah perintah yang dapat digunakan oleh pengguna dalam mengoperasikan \textit{Data Collector} melalui file tersebut:

\begin{enumerate}
  \item clean
  
  Menghapus semua data \textit{persistent} dari sistem dan juga \textit{log} dari Node.
  
  \item run\_node
  
  Menjalankan fungsi run\_node pada komponen \textit{benchmark}.
  
  \item run\_all
  
  Menjalankan fungsi run\_all pada komponen \textit{benchmark}.
  
  \item run\_benchmark
  
  Menjalankan fungsi run\_benchmark pada komponen \textit{benchmark}.
  
  \item stop\_all: Menjalankan fungsi stop\_all pada komponen \textit{benchmark}.
  
  \item bench\_system
  
  Menjalankan fungsi bench\_system pada komponen \textit{benchmark}.
  
  
  \item bench\_system\_with\_reset
  
  Menjalankan fungsi bench\_system\_with\_reset pada komponen \textit{benchmark}.
  
  \item bench\_baseline
  
  Menjalankan fungsi bench\_baseline pada komponen \textit{benchmark}.
  
  \item run\_bench\_suite
  
  Menjalankan fungsi run\_bench\_suite pada komponen \textit{benchmark}.
  
  \item add\_netem\_limits
  
  Menjalankan fungsi add\_netem\_limits pada komponen \textit{benchmark}.
  
  \item remove\_netem\_limits
  
  Menjalankan fungsi remove\_netem\_limits pada komponen \textit{benchmark}.

  \item organize\_files
  
  Menjalankan fungsi organize\_files pada komponen \textit{log management}.
  
  \item help
  
  Menampilkan daftar perintah yang tersedia beserta penjelasannya. Perintah ini akan menampilkan informasi tentang perintah-perintah yang dapat digunakan dalam \textit{Data Collector} dan cara penggunaannya.
\end{enumerate}

Komponen-komponen \textit{Data Collector} diimplementasikan dalam bahasa pemrograman Bash dan JavaScript. Rincian implementasi dari masing-masing komponen dijelaskan pada subbab pada bagian ini.

\subsubsection{Implementasi Komponen Benchmark}
\label{subsubsection:implementasi-benchmark}

Seperti yang dijelaskan pada rancangan di bagian \ref{subsubsection:detail-data-benchmark}, komponen \textit{benchmark} bertanggung jawab untuk melakukan pengujian kinerja dari \textit{key-value store database} yang sudah dibangun. Dalam melakukan pengujian kinerja tersebut, komponen ini juga memiliki kemampuan untuk mengelola sistem yang sedang diuji, termasuk memulai, menghentikan, dan menghapus data dari masing-masing \textit{node} dari \textit sistem {key-value store database}.

Implementasi komponen \textit{benchmark} dilakukan dengan menggunakan bahasa bash untuk mengelola proses sistem dan menggunakan Javascript sebagai \textit{client}. Implementasi menggunakan bahasa bash terdapat pada file scripts.sh dengan bentuk fungsi. Beberapa fungsi dapat dipanggil secara manual melalui terminal oleh pengguna seperti yang sudah disebutkan sebelumnya pada bagian \ref{subsubsection:data-collector}.

Seperti yang dapat dilihat pada gambar \ref{fig:benchmark-structure}, komponen \textit{benchmark} memiliki beberapa fungsi yang dapat dikelompokkan menjadi konfigurasi sistem atau utilitas, manajemen \textit{node}, dan eksekutor \textit{benchmark} utama.

Fungsi-fungsi yang dikelompokkan menjadi konfigurasi sistem adalah sebagai berikut:

\begin{enumerate}
  \item create\_screen\_session

  Membuat sesi \textit{screen} baru untuk menjalankan proses sistem \textit{key-value store database}. Sesi ini akan digunakan untuk menjalankan setiap \textit{node} secara terpisah.

  \item start\_terminal

  Memulai terminal baru untuk menjalankan perintah secara interaktif. Terminal ini akan digunakan untuk menjalankan perintah-perintah yang diperlukan selama pengujian kinerja.

  \item validate\_config

  Memeriksa apakah konfigurasi yang diberikan valid dan sesuai dengan yang diharapkan. Jika konfigurasi tidak valid, fungsi ini akan memberikan pesan kesalahan yang sesuai. Konfigurasi ditentukan pada file ./etc/config.json yang berisi tentang informasi \textit{node}. Untuk masing-masing node, konfigurasi yang ada adalah \textit{ip}, \textit{port}, \textit{port http}, \textit{path} untuk \textit{transaction log}, dan \textit{path} untuk data persisten \textit{key-value store database}. Konfigurasi global yang ada selain konfigurasi masing-masing node adalah \textit{storage}, yaitu jumlah \textit{data shard} dan \textit{parity shard} untuk erasure coding serta ukuran \textit{payload} maksimal yang akan diterima sistem.

  \item clean

  Membersihkan data persisten dan log dari semua \textit{node} yang ada. Fungsi ini akan menghapus direktori ./db/node* dan ./logs/* untuk memastikan bahwa sistem dimulai dalam kondisi bersih sebelum pengujian kinerja dilakukan.

  \item add\_netem\_limits

  Menambahkan pembatasan \textit{bandwidth} pada \textit{interface} jaringan untuk mensimulasikan kondisi jaringan. Fungsi ini menggunakan kakas iptables untuk menandai paket tempat berjalannya sistem \textit{key-value store database} dan menerapkan \textit{traffic control} (tc) dengan netem \textit{bandwidth} pada \textit{interface} jaringan yang digunakan oleh sistem tersebut.

  \item remove\_netem\_limits
  
  Menghapus pembatasan \textit{bandwidth} yang telah diterapkan sebelumnya. Fungsi ini akan menghapus aturan yang telah ditambahkan pada fungsi add\_netem\_limits untuk mengembalikan kondisi jaringan ke keadaan normal.
  
\end{enumerate}

Fungsi-fungsi yang dikelompokkan menjadi manajemen \textit{node} adalah sebagai berikut:

\begin{enumerate}
  \item run\_node

  Menjalankan satu \textit{node} dari sistem \textit{key-value store database} sesuai dengan konfigurasi yang telah ditentukan. Konfigurasi ditentukan pada file ./etc/config.json yang berisi tentang informasi \textit{node}. Untuk masing-masing node, konfigurasi yang ada adalah \textit{ip}, \textit{port}, \textit{port http}, \textit{path} untuk \textit{transaction log}, dan \textit{path} untuk data persisten \textit{key-value store database}.

  \item run\_all
  
  Menjalankan semua \textit{node} dari sistem \textit{key-value store database} sesuai dengan konfigurasi yang telah ditentukan. Konfigurasi ditentukan pada file ./etc/config.json yang berisi tentang informasi \textit{node}. Konfigurasi global yang ada selain konfigurasi masing-masing node adalah \textit{storage}, yaitu jumlah \textit{data shard} dan \textit{parity shard} untuk erasure coding serta ukuran \textit{payload} maksimal yang akan diterima sistem.
  
  \item stop\_all
  
  Menghentikan semua sesi \textit{screen} terkait sistem \textit{key-value store database} yang sedang berjalan. Fungsi ini akan mencari sesi \textit{screen} dengan nama penanda proses sistem dan menghentikan semua sesi tersebut. Fungsi ini digunakan untuk membersihkan proses yang berjalan sebelum memulai benchmark baru. Pengguna dapat memanggil fungsi ini secara manual melalui terminal.
\end{enumerate}


Fungsi-fungsi yang dikelompokkan menjadi eksekutor \textit{benchmark} utama adalah sebagai berikut:

\begin{enumerate}  
  \item bench\_system
  
  Menjalankan benchmark k6 terhadap sistem yang sedang berjalan. Melakukan iterasi melalui parameter konfigurasi testing, yaitu bandwidth, virtual\_users, dan size. Untuk setiap kombinasi parameter, fungsi ini akan menerapkan pembatasan bandwidth menggunakan fungsi \textit{add\_netem\_limits}, menjalankan monitoring CPU dengan \textit{mpstat}, menjalankan benchmark k6 dengan skrip JavaScript yang sesuai (read/write), dan menyimpan hasil benchmark dalam format JSON serta log CPU. Setelah selesai, fungsi ini akan menghapus pembatasan bandwidth yang telah diterapkan.

  \item bench\_system\_with\_reset

  Menjalankan benchmark k6 terhadap sistem yang sedang berjalan dengan mereset kondisi sistem sebelum pengujian. Fungsi ini akan menghentikan semua proses yang berjalan, menghapus data yang ada, dan memulai ulang sistem sebelum menjalankan benchmark. Hal ini bertujuan untuk memastikan bahwa pengujian dilakukan dalam kondisi yang bersih dan konsisten.
  
  \item run\_bench\_suite
  
  Menjalankan seluruh rangkaian pengujian \textit{benchmark} dengan semua kombinasi variabel yang ditentukan. Fungsi ini menjalankan benchmark komprehensif dengan replikasi dan \textit{erasure coding}, read dan write, variasi \textit{bandwidth}, \textit{virtual users}, dan \textit{payload size} yang ditentukan.
\end{enumerate}

Pada scripts.js, selain fungsi-fungsi yang disebutkan sebelumnya, terdapat juga variabel-variabel untuk konfigurasi \textit{benchmark} yang akan dilakukan menggunakan run\_bench\_suite. Variabel-variabel tersebut adalah:

\begin{enumerate}
  \item virtual\_users: \textit{number array} untuk jumlah pengguna virtual yang akan digunakan dalam pengujian. Setiap pengguna virtual akan melakukan permintaan ke sistem secara bersamaan.
  \item size: \textit{number array} untuk ukuran \textit{payload} yang akan digunakan dalam pengujian. Ukuran ini menentukan seberapa besar data yang disimpan.
  \item bandwidth: \textit{number array} untuk pembatasan \textit{bandwidth} yang akan diterapkan pada sistem selama pengujian. Pembatasan ini digunakan untuk mensimulasikan kondisi jaringan yang berbeda-beda.
\end{enumerate}

Selain komponen yang terdapat pada scripts.js, bagian dari komponen \textit{benchmark} adalah \textit{script} Javascript yang digunakan untuk menjalankan benchmark k6. File-file terkait yang digunakan untuk menjalankan \textit{client} adalah:

\begin{itemize}
  \item ./benchmark/script-write.js

  File ini berisi skrip JavaScript yang digunakan untuk melakukan pengujian tulis (write) pada sistem \textit{key-value store database} menggunakan k6. Skrip ini akan mengirimkan permintaan \textit{write} ke sistem dan mengukur kinerja respons. Nilai yang dikirim di-\textit{encode} menjadi \textit{base64} untuk memastikan bahwa data yang dikirim dapat diterima oleh sistem. Oleh karena itu, ukuran \textit{payload} yang dikirim akan lebih besar dari ukuran yang ditentukan dalam konfigurasi. Akan tetapi, hal ini membuat gambaran kinerja sistem menjadi lebih fleksibel karena dapat menyimpan data dalam bentuk apapun.

  \item ./benchmark/script-read.js
  
  File ini berisi skrip JavaScript yang digunakan untuk melakukan pengujian baca (read) pada sistem \textit{key-value store database} menggunakan k6. Skrip ini akan mengirimkan permintaan \textit{read} ke sistem dan mengukur kinerja respons. Sebelum menjalankan permintaan \textit{read}, \textit{script} ini akan mengirimkan permintaan \textit{write} untuk memastikan bahwa data yang akan dibaca sudah tersedia di sistem. Sama seperti pada \textit{script} write, nilai yang dikirim di-\textit{encode} menjadi \textit{base64} untuk memastikan bahwa data yang dikirim dapat diterima oleh sistem.

  \item ./benchmark/oneshot.js
  
  File ini berisi skrip JavaScript yang digunakan untuk melakukan pengujian \textit{oneshot} pada sistem \textit{key-value store database} menggunakan k6. Skrip ini akan mengirimkan permintaan \textit{write} atau \textit{read} satu kali saja. Tujuannya adalah untuk memverifikasi bahwa sistem dapat menerima permintaan dengan benar. Selain itu, \textit{script} ini juga membantu dalam proses \textit{debugging}.
\end{itemize}
\subsubsection{Implementasi Komponen Log Management}
\label{subsubsection:implementasi-log-management}
