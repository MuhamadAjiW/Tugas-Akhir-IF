\section{Implementasi}
\label{sec:implementation}

Bagian ini menjelaskan implementasi dari sistem \textit{distributed key-value store database} yang telah dibangun secara terperinci. Seperti yang telah dijelaskan pada bagian \ref{subsection:rancangan-struktural}, sistem dikembangkan sebagai sebuah \textit{Node} yang dibangun secara modular. Selain itu, dibangun juga \textit{Data Collector} yang berfungsi sebagai pengumpul data dari pengujian sistem. Penjelasan bagian ini dimulai dari batasan implementasi, dilanjutkan dengan kakas yang digunakan dalam pembuatan sistem, dan diakhiri dengan penjelasan mengenai implementasi dari masing-masing komponen sistem.

Mengulangi definisi yang disebutkan pada bagian \ref{sec:rancangan}, \textit{Node} adalah satuan fungsional utama yang berperan sebagai entitas dalam sistem \textit{database} terdistribusi yang dikembangkan dan \textit{Data Collector} adalah bagian dari eksperimen yang bertugas untuk mengumpulkan data eksperimen.

\subsection{Batasan Implementasi}
\label{subsection:batasan-implementasi}

Berikut adalah batasan implementasi yang ditetapkan dalam pengembangan sistem:
\begin{enumerate}
  \item Semua batasan masalah yang telah ditetapkan pada bagian \ref{sec:batasan-masalah}.
  \item Operasi yang didukung hanya pembacaan (\textit{read}) dan penulisan (\textit{write}); tidak mengakomodasi operasi penghapusan atau transaksi multi-kunci atau transaksi kompleks lainnya.
  \item Pengujian performa hanya dilakukan pada lingkungan lokal (loopback) dengan alat \texttt{k6} dan \texttt{mpstat}.
  \item Konfigurasi \textit{membership} dari sebuah \textit{cluster} bersifat statis; tidak mengakomodasi perubahan \textit{node} pada waktu berjalan.
  \item Konfigurasi \textit{erasure coding} (jumlah data shard dan parity shard) bersifat statis; tidak mengakomodasi penyesuaian dinamis pada waktu berjalan.
  \item Tidak ada pengujian toleransi kesalahan yang kompleks; kegagalan yang diuji hanya \textit{crash-stop}.
\end{enumerate}


\subsection{Kakas yang Digunakan}
\label{subsection:kakas-yang-digunakan}

Kakas yang digunakan dalam implementasi ini adalah sebagai berikut:
\begin{enumerate}
  \item Bahasa Pemrograman:
    \begin{enumerate}
      \item Rust: Digunakan untuk mengimplementasikan \textit{Node} dan sistem \textit{distributed key-value store database} secara keseluruhan.
      \item Bash: Digunakan untuk \textit{shell scripting} dalam automatisasi, melakukan pengelolaan node dan bagian dari \textit{Data Collector}.
      \item JavaScript: Digunakan untuk menulis skrip k6 dalam \textit{Data Collector} untuk melakukan pengujian beban dan pengumpulan data.
    \end{enumerate}
  \item Kakas dalam bahasa Rust:
    \begin{enumerate}
      \item bincode: Digunakan untuk serialisasi dan deserialisasi biner dalam komunikasi antarnode dan penyimpanan.
      \item tokio: \textit{runtime} async untuk \textit{Non-blocking I/O} berkinerja tinggi.
      \item serde: Digunakan sebagai kerangka kerja serialisasi dan deserialisasi biner bersamaan dengan bincode untuk transformasi data.
      \item actix-web: Digunakan sebagai \textit{framework web} untuk HTTP \textit{server} pada node.
      \item tracing, tracing-subscriber, tracing-appender: Digunakan untuk instrumentasi \textit{logging} dan \textit{tracing}.
      \item clap: Digunakan untuk parsing argumen baris perintah untuk binari node.
      \item moka: Digunakan untuk \textit{in-memory store}.
      \item rand: Digunakan sebagai generator bilangan acak untuk keperluan konsensus.
      \item rocksdb: Digunakan sebagai \textit{storage engine} untuk persistent key-value store lokal setiap node.
    \end{enumerate}
  \item Kakas dalam bahasa JavaScript:
    \begin{enumerate}
      \item k6: Digunakan untuk mengeksekusi skenario \textit{read/write} dan menghasilkan metrik JSON.
    \end{enumerate}
  \item Kakas lainnya:
    \begin{enumerate}
      \item GNU Screen: Digunakan untuk menjalankan beberapa session terminal node terpisah yang dapat dipantau.
      \item jq: Digunakan untuk memanipulasi dan memvalidasi konfigurasi JSON sebelum menjalankan sistem.
      \item iptables: Digunakan untuk menandai paket TCP dan mengontrol \textit{bandwidth}.
      \item tc: Digunakan untuk mengatur antrian paket dan pembatasan bandwidth (netem) pada loopback.
      \item sysstat: Digunakan untuk merekam statistik CPU secara \textit{real-time} saat pengujian berlangsung.
    \end{enumerate}
\end{enumerate}
\input{chapters/chapter-4/02-03-setup.tex}
\input{chapters/chapter-4/02-04-implementasi-node.tex}
\input{chapters/chapter-4/02-05-implementasi-data-collector.tex}
