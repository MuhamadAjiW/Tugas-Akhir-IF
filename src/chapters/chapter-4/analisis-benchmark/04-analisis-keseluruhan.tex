\subsubsection{Analisis Keseluruhan}
\label{subsubsection:analisis-keseluruhan}

Setelah melakukan analisis terhadap dua operasi \textit{write} dan \textit{read}, bagian analisis ini menyatukan temuan dari evaluasi kinerja operasi tersebut. Tujuannya menyajikan gambaran keseluruhan mengenai \textit{trade-off} fundamental antara \textit{erasure coding} dan replikasi.

Hasil penelitian ini menunjukkan bahwa untuk kedua sistem:
\begin{enumerate}
	\item \textit{Erasure Coding}

	      Sistem berbasis \textit{erasure coding} selain menawarkan efisiensi penyimpanan yang superior dibandingkan replikasi, juga memiliki keunggulan dalam hal latensi operasi \textit{write} pada beban kerja tertentu. Beban kerja yang dimaksud adalah ketika ukuran data cukup besar dan \textit{bandwidth} jaringan yang tersedia terbatas.

	\item Replikasi

	      Sistem berbasis replikasi menawarkan kinerja operasi \textit{write} yang lebih baik pada beban kerja dengan ukuran data kecil dan \textit{bandwidth} jaringan yang tinggi. Selain itu, replikasi juga unggul dalam operasi \textit{read}. Perlu diperhatikan bahwa replikasi membutuhkan biaya penyimpanan yang lebih tinggi dibandingkan \textit{erasure coding}.

\end{enumerate}

Dalam penggunaannya \textit{erasure coding} dan replikasi memiliki kelebihan dan kekurangan masing-masing. Penggunaan \textit{erasure coding} lebih cocok untuk sistem yang memprioritaskan efisiensi penyimpanan dan toleransi kegagalan, sedangkan replikasi lebih sesuai untuk sistem yang mengutamakan kinerja operasi \textit{read} dan \textit{write} pada beban kerja dengan ukuran data kecil.

Pemilihan penggunaan kedua sistem perlu mempertimbangkan karakteristik beban kerja, rasio baca dan tulis, ukuran data, dan infrastruktur jaringan yang tersedia. Dalam \textit{key-value store} yang lazim beroperasi dengan data kecil dan menggunakan infrastruktur \textit{data center} ber-\textit{bandwidth} tinggi, replikasi akan menjadi pilihan yang lebih baik. Namun, untuk sistem yang menangani data besar dan memerlukan efisiensi penyimpanan, dan berinfrastruktur terbatas, \textit{erasure coding} akan lebih diuntungkan.
