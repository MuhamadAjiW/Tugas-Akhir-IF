\chapter*{Kata Pengantar}
\addcontentsline{toc}{chapter}{KATA PENGANTAR}

Puji dan syukur penulis panjatkan kepada Tuhan Yang Maha Esa atas berkat dan rahmat-Nya, laporan tugas akhir yang berjudul "\thetitle" dapat diselesaikan dalam rangka memenuhi syarat kelulusan tingkat sarjana. Pengerjaan tugas ini tentu tidak lepas dari bantuan berbagai pihak. Oleh karena itu, penulis ingin mengucapkan terima kasih kepada:

\begin{enumerate}
	\item Bapak Achmad Imam Kistijantoro, S.T, M.Sc., Ph.D. selaku dosen pembimbing atas segala bentuk jasa, dukungan, dan kesabarannya dalam membimbing penulis serta memberikan saran dalam pengerjaan tugas akhir.
	\item Bapak Dr.techn. Saiful Akbar, S.T., M.T. dan Bapak Dion Tanjung,. S.Kom., M.sc., Ph.D. selaku dosen penguji atas segala masukan serta kritik yang telah diberikan terhadap tugas akhir penulis.
	\item Ibu Robithoh Annur, S.T., M.Eng., Ph.D. dan Tricya Esterina Widagdo, ST., M.Sc. selaku dosen koordinator tim tugas akhir atas arahan dan dorongan kepada mahasiswa program studi Teknik Informatika untuk mengerjakan tugas akhirnya.
	\item Seluruh dosen program studi Teknik Informatika ITB yang telah memberikan ilmu pengetahuan yang sangat berharga bagi penulis.
	\item Kak Fadhil I. Kurnia dan Kak Muhammad Akram Al Bari yang telah memberikan ide serta berperan sebagai mentor penulis dalam proses pengerjaan tugas akhir.
	\item Semua pengembang AI Gemini 2.5 pro, Claude Sonnet 4, dan GPT 4.1 yang telah memberikan penulis rekan setia dalam proses implementasi, pengujian, hingga penulisan penelitian ini.
	\item Kedua orang tua penulis yang telah memberikan dukungan, doa, dan motivasi kepada penulis selama menempuh pendidikan di program studi Teknik Informatika.
	\item Teman-teman penulis, khususnya anggota dari grup "lesgo wahoo!" dan Laboratorium Sistem Terdistribusi yang telah memberikan kenangan berharga.
	\item Teman-teman SUDO 2021 dan para anggota HMIF ITB lainnya yang kerap membantu sesama dalam menempuh kuliah pada program studi Teknik Informatika.
	\item Semua member discord "Juniority Forever" yang sering memberikan hiburan dan asupan drama kepada penulis ketika penat mengerjakan tugas akhir.
	\item Hatsune Miku sebagai penyedia \textit{soundtrack} selama proses pengerjaan tugas akhir.
	\item Seluruh pihak lain yang telah membantu dalam proses pengerjaan tugas akhir.
\end{enumerate}

Akhir kata, penulis mengucapkan terima kasih kepada semua pihak yang telah terlibat dalam pengerjaan tugas akhir ini. Penulis juga ingin menyampaikan mohon maaf apabila terdapat kesalahan maupun kekurangan dalam laporan tugas akhir ini. Penulis berharap semoga tugas akhir ini dapat bermanfaat bagi pembaca dan riset-riset kedepannya.

\begin{flushright}
	\vspace{0.5cm}
	Bandung, \tanggalpengesahan

	\vspace{1.5cm}

	\authorname
\end{flushright}