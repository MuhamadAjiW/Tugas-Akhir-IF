\subsubsection{Pemilihan Bahasa Pemrograman}
\label{subsubsection:pilihan-bahasa-pemrograman}

Bahasa Rust dipilih sebagai bahasa pemrograman yang digunakan untuk mengembangkan sistem dalam eksperimen ini. Alasan pemilihan bahasa Rust didasarkan pada keunggulan bahasa ini dalam hal kinerja, keamanan, dan kemudahan pengembangan sistem terdistribusi. Khususnya dalam hal \textit{memory safety} dan \textit{concurrency safety} yang menjadi fokus utama dalam pengembangan sistem terdistribusi. Selain itu, Rust juga memiliki ekosistem yang berkembang pesat dengan banyak library yang mendukung pengembangan sistem.

Sementara itu, untuk pengembangan sistem testing dan pengumpulan data eksperimen, digunakan bahasa lain yang menyesuaikan dengan alat dan framework yang tersedia. Berikut adalah bahasa lain yang digunakan dalam eksperimen ini:
\begin{itemize}
    \item \textbf{Python}: Digunakan untuk mengembangkan sistem testing dan pengumpulan data eksperimen. Python dipilih karena kemudahan dalam pengembangan dan ketersediaan library yang mendukung pengumpulan data eksperimen. Python juga memiliki ekosistem yang luas untuk analisis data, sehingga memudahkan dalam pengolahan dan visualisasi hasil eksperimen.
    \item \textbf{Bash}: Digunakan untuk mengotomatisasi proses pengumpulan data eksperimen. Bash dipilih karena kemudahan dalam mengelola skrip dan menjalankan perintah sistem. Bash juga memungkinkan integrasi dengan alat lain yang digunakan dalam eksperimen ini, seperti Python dan Rust.
\end{itemize}