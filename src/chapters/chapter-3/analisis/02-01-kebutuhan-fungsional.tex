
\subsubsection{Kebutuhan Fungsional}
\label{subsubsection:functional-requirements}

Dalam pengembangannya, sistem eksperimen ini harus memenuhi beberapa kebutuhan fungsional. Kebutuhan fungsional yang diturunkan dari analisis permasalahan pada bagian \ref{subsection:analisis-permasalahan} adalah dapat dilihat pada tabel \ref{subsection:alternatif-solusi}.

\begin{longtable}{|l|p{13cm}|}
\caption{Kebutuhan Fungsional}
\label{tab:functional-requirements} \\
\hline
\rowcolor{black!10} ID & Deskripsi Kebutuhan \\ \hline
\endfirsthead

\caption[]{Kebutuhan Fungsional (lanjutan)} \\
\hline
\rowcolor{black!10} ID & Deskripsi Kebutuhan \\ \hline
\endhead

F-1 & Sistem harus dapat melakukan operasi \textit{read} dan \text{write} pada sebuah \textit{key-value store database} \\ \hline
F-2 & Sistem harus dapat menyimpan data secara \textit{persistent} \\ \hline
F-3 & Sistem harus dapat mencatat waktu transaksi dari \textit{request} masuk hingga operasi selesai \\ \hline
F-4 & Sistem harus dapat meng-\textit{encode} data menggunakan \textit{erasure coding} \\ \hline
F-5 & Sistem harus dapat merekonstruksi data dari data yang disimpan menggunakan \textit{erasure coding} \\ \hline
F-6 & Sistem harus dapat mendistribusikan data atau sebagian data ke \textit{node} lain untuk keperluan ketahanan data \\ \hline
F-7 & Sistem harus dapat dikonfigurasi untuk menggunakan replikasi ataupun \textit{erasure coding} tanpa mengganti konfigurasi lainnya \\ \hline
F-8 & Sistem harus dapat dikonfigurasi untuk mencapai tingkat ketahanan tertentu tanpa mengganti konfigurasi lainnya \\ \hline
F-9 & Sistem harus dapat mensimulasikan \textit{request} dengan ukuran data yang bervariasi \\ \hline
F-10 & Sistem harus dapat menjalankan \textit{request} secara berulang kali dan bervariasi secara otomatis untuk pengumpulan data \\ \hline
\end{longtable}
