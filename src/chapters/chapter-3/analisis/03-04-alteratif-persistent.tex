\subsubsection{Pemilihan Persistent Database}
\label{subsubsection:persistent-database}

Komponen \textit{Persistent Database} berperan sebagai penyimpanan data \textit{persistent} untuk memenuhi kebutuhan yang sudah disebutkan di bagian \ref{subsection:analisis-permasalahan}. Tabel \ref{tab:tools-db} menunjukkan perbandingan beberapa kakas \textit{persistent database} yang dipertimbangkan untuk digunakan dalam eksperimen ini.

\begin{table}[h]
    \centering
    \caption{Perbandingan kakas persistent database}
    \resizebox{\textwidth}{!}{
        \begin{tabular}{|l|p{5cm}|p{5cm}|p{3cm}|}
            \hline
            \rowcolor{black!10} Kakas & Kelebihan & Kekurangan & Notes \\ \hline
            Cassandra & +Horizontal Scalability baik \newline +Sudah ada referensi implementasi erasure coding pada paper lain & -Eventual Consistency \newline -Latensi tinggi untuk operasi write \newline -Sangat Kompleks & Utamanya Wide-Column Store\\ \hline
            ScyllaDB & +Lebih efisien dan performant dibandingkan Cassandra \newline +Memiliki pilihan strong consistency & -Lebih kompleks dibandingkan Cassandra \newline -Ekosistem terbatas & Mirip dan kompatibel dengan Cassandra. \\ \hline
            RocksDB & +Kinerja tinggi untuk operasi write \newline +Recovery menggunakan write ahead log & -Single Node \newline -Keterbatasan pada concurrent writes & Pengembangan LevelDB\\ \hline
            LevelDB & +Simpel \newline +Penggunaan memori kecil \newline +Kinerja tinggi untuk operasi write & -Fitur terbatas seperti compression dan data terbatas pada string \newline -Operasi write skala besar masih lebih lambat dibandingkan RocksDB \newline -Keterbatasan pada concurrent writes & Dasar dari RocksDB\\ \hline
            MongoDB & +Support query kompleks \newline +Penggunaan fleksibel & -Operasi kurang optimal \newline -Konsumsi memori besar \newline -Storage overhead besar & Utamanya Dokumen database\\ \hline
        \end{tabular}
    }
    \label{tab:tools-db}
\end{table}

Dari kakas-kakas tersebut, dipilih RocksDB sebagai kakas untuk eksperimen ini. Selain RocksDB, LevelDB juga merupakan kandidat yang kuat untuk eksperimen ini karena sifatnya yang simpel. Namun, pada akhirnya, RocksDB dipilih karena pengembangannya yang masih aktif, kinerja \textit{write} yang lebih tinggi, serta fitur-fitur tambahan seperti \textit{compression} dan penyimpanan data dalam \textit{byte}.