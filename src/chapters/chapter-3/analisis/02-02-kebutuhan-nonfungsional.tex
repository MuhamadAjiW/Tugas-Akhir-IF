\subsubsection{Kebutuhan Non-fungsional}
\label{subsubsection:non-functional-requirements}

Dalam pengembangannya, sistem eksperimen ini harus memenuhi beberapa kebutuhan non-fungsional. Kebutuhan fungsional ini diturunkan dari Bagian \ref{sec:latar-belakang} dan analisis permasalahan pada Bagian \ref{subsection:analisis-permasalahan}. Kebutuhan non-fungsional sistem dapat dilihat pada Tabel \ref{tab:non-functional-requirements}.

\begin{longtable}{|l|p{13cm}|}
\caption{Kebutuhan Non-Fungsional}
\label{tab:non-functional-requirements} \\
\hline
\rowcolor{black!10} ID & Deskripsi Kebutuhan \\ \hline
\endfirsthead

\caption[]{Kebutuhan Non-Fungsional (lanjutan)} \\
\hline
\rowcolor{black!10} ID & Deskripsi Kebutuhan \\ \hline
\endhead

NF-1 & Sistem harus menyediakan \textit{consistency} yang tinggi dengan \textit{request} ke \textit{node} manapun harus menghasilkan hasil yang sama \\ \hline
NF-2 & Sistem harus memiliki \textit{availability} yang tinggi dengan harus dapat tetap tersedia walaupun beberapa \textit{node} ada dalam kondisi gagal \\ \hline
NF-3 & Sistem harus menggunakan penyimpanan minimal untuk skalabilitas dan efisensi biaya \\ \hline
NF-4 & Sistem harus menyediakan \textit{response time} rendah untuk operasi \textit{read} dan \textit{write} \\ \hline
\end{longtable}