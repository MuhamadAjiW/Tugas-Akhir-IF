\subsection{Alternatif Solusi}
\label{sec:alternatif-solusi}

Berdasarkan analisis kebutuhan sistem pada bagian \ref{sec:analisis-kebutuhan-sistem}, terdapat berbagai macam alternatif solusi untuk memenuhi kebutuhan tersebut. Pada penelitian ini, solusi yang dipilih adalah dengan membuat sistem yang mengkombinasikan Memcached sebagai \textit{in-memory key value store} dengan RocksDB sebagai \textit{persistent storage} dengan Reed-Solomon sebagai algoritma \textit{erasure coding}. Pertimbangan dari pemilihan solusi dibandingkan alternatif yang lain adalah sebagai berikut:

\subsubsection{Menggunakan sistem yang sudah memiliki \textit{persistent storage}}
Salah satu kebutuhan sistem adalah sistem harus dapat mensimulasikan kondisi database terdistribusi yang menggunakan replikasi ataupun erasure coding. Penggunaan sistem yang sudah memiliki \textit{persistent storage} akan membatasi sistem dengan implementasi \textit{persistent storage} yang sudah ada. Selain itu, jika implementasi tidak sesuai, memodifikasi \textit{source code} untuk penyesuaian kebutuhan eksperimen adalah aksi yang kompleks. Hal ini mempersulit pengembangan sistem untuk memenuhi kebutuhan tersebut. Dengan demikian, pengembangan akan lebih fleksibel jika menggunakan \textit{in-memory key value store} dan \textit{persistent storage} yang terpisah.

\subsubsection{Menggunakan database selain RocksDB}
Seperti yang dijelaskan pada bagian \ref{sec:erasure-coding}, penerapan \textit{erasure coding} akan memerlukan komputasi yang tinggi, khususnya pada pembangunan ulang data. Hal ini membuat erasure coding lebih layak jika digunakan untuk operasi \textit{write}. Oleh karena itu, RocksDB dipilih sebagai \textit{persistent storage} karena berfokus pada kinerja \textit{write} yang tinggi. Penggunaan RocksDB juga dilakukan seperti \textit{library} sehingga memudahkan pengembangan sistem yang modular. Perbandingan RocksDB dengan \textit{database} lain dapat dilihat pada lampiran \ref{appendix:tools}. 

\subsubsection{Menggunakan algoritma erasure coding selain Reed-Solomon}
Algoritma Reed-Solomon dipilih sebagai algoritma untuk \textit{erasure coding} karena algoritma ini juga memiliki kemampuan untuk ketahanan yang tidak terbatas. Selain itu, algoritma ini juga berbasis \textit{block} sehingga lebih sesuai untuk penyimpanan data. Algoritma ini juga umum digunakan pada sistem penyimpanan data terdistribusi. Perbandingan Reed-Solomon dengan algoritma erasure coding lain dapat dilihat pada lampiran \ref{appendix:ec-algorithms}.

% Pendekatan ini mengkombinasikan antara sistem \textit{in-memory key-value store} milik Memcached dan menggunakan \textit{database} terpisah sebagai tempat penyimpanan data \textit{persistence}. Replikasi dan \textit{erasure coding} dapat dilakukan pada data persisten yang disimpan pada \textit{database} tersebut. Kelebihan dari pendekatan ini adalah kemudahan dari implementasi dibandingkan pengembangan sendiri ataupun modifikasi \textit{database} yang sudah lebih kompleks. Namun, kekurangannya adalah \textit{overhead} penggunaan memori dari menjalankan Memcached dan \textit{database} terpisah. Dari ketiga alternatif solusi, kemudahan implementasi dan kekurangan yang dapat ditoleransi menjadikan alternatif ini solusi yang dipilih dalam penelitian ini.

% \subsubsection{Menggunakan sistem Memcached dengan fitur-fitur yang dibutuhkan dikembangkan sendiri}
% Pendekatan ini membangun fitur-fitur yang dibutuhkan untuk penelitian di atas Memcached secara mandiri. Kelebihan dari pendekatan ini kebebasan yang tinggi sehingga sistem dapat disesuaikan untuk eksperimen. Namun, pendekatan ini sulit mencerminkan kondisi nyata. Selain itu, kinerja sistem yang dikembangkan juga bergantung erat pada waktu serta kemampuan pengembangan peneliti.

% \subsubsection{Mengembangkan sistem yang sudah memiliki fitur lengkap dengan mengganti \textit{source-code database} untuk fitur yang dibutuhkan}
% Pengembangan ini membangun mengganti fitur yang dibutuhkan untuk penelitian di atas \textit{database} lebih kompleks dari Memcached. Pendekatan ini memiliki kelebihan kedekatan kondisi eksperimen dengan dunia nyata. Namun, pendekatan ini sulit dilakukan karena membutuhkan waktu yang lama untuk memahami dan mengubah \textit{source-code database} yang kompleks.
