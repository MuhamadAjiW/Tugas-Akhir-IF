\section{Rancangan Struktural}
\label{sec:rancangan-struktural}

Kebutuhan fungsional dan non-fungsional dipetakan pada struktur sistem yang diimplementasikan sebagai kumpulan komponen. Komponen didefinisikan sebagai sebuah unit fungsional yang berdiri sendiri dan dapat disusun. Beberapa komponen disusun secara hierarkis dan membentuk subsistem yang mewakili domain tanggung jawab tertentu dalam sistem. Setiap komponen dirancang untuk memenuhi tujuan spesifik berdasarkan kebutuhan sistem. Detail terkait setiap komponen ini dapat dilihat pada bagian \ref{sec:detail-komponen}.

\subsection{Node}
\label{subsection:node}

\textit{Node} adalah satuan fungsional utama yang berperan sebagai entitas dalam sistem \textit{database} terdistribusi yang dikembangkan. Dalam eksperimen ini, \textit{Node} berfungsi sebagai \textit{key-value store}. Sesuai dengan perancangan pada bagian \ref{sec:rancangan-struktural}, \textit{Node} dibangun secara modular dengan beberapa komponen yang tergabung dalam beberapa subsistem.

% Secara umum, struktur \textit{Node} terdiri dari:

% \begin{enumerate}
%     \item Subsistem penyimpanan: Subsistem ini bertanggung jawab untuk fungsionalitas \textit{key-value store} dalam sebuah \textit{Node}. Subsistem ini akan terdiri atas komponen \textit{in-memory store}, \textit{persistent store}, dan \textit{transaction log}. Subsistem ini juga mengkonfigurasi \textit{Node} untuk menggunakan replikasi atau \textit{erasure coding}.
%     \item Subsistem kontrol: Subsistem ini bertanggung jawab mengelola transaksi dan konsistensi antar-\textit{Node}. Pengelolaan tersebut dilakukan dengan mengaplikasikan algoritma konsensus untuk menjaga konsistensi data antar-\textit{Node}. Subsistem ini juga bertanggung jawab untuk melakukan \textit{recovery} data dari \textit{transaction log} jika terjadi kegagalan pada \textit{Node}.
%     \item Komponen HTTP \textit{server}: Komponen ini berperan sebagai antarmuka komunikasi antara \textit{client} dan \textit{Node}.
%     \item Komponen komunikasi antar-\textit{Node}: Komponen ini berfungsi untuk mengelola komunikasi antar-\textit{Node} dalam sistem terdistribusi. Komponen ini juga bertanggung jawab untuk mendistribusikan data ke \textit{Node} lain.
% \end{enumerate}

Merujuk analisis kebutuhan sistem pada bagian \ref{sec:analisis-permasalahan}, \textit{Node} memenuhi kebutuhan 1 dan 2, yaitu:

\begin{enumerate}
    \item Sistem harus dapat mensimulasikan kondisi \textit{database} terdistribusi yang menggunakan replikasi ataupun \textit{erasure coding}.
    \item Sistem harus dapat menyimpan data secara \textit{persistent} untuk mensimulasikan kegagalan dan pemulihan.
\end{enumerate}

Sedangkan merujuk kebutuhan fungsional dan non-fungsional pada bagian \ref{sec:system-requirements}, \textit{Node} memenuhi kebutuhan fungsional F-1, F-2, F-4, F-5, F-6, F-7, F-8, dan F-11 serta non-fungsional NF-1, NF-2, NF-3, dan NF-4.

\subsubsection{Data Collector}
\label{subsubsection:data-collector}

\textit{Data Collector} adalah satuan fungsional yang bertugas untuk melakukan \textit{request} dan transaksi pada sistem untuk mengumpulkan data eksperimen. \textit{Data Collector} ini memiliki fitur yang memungkinkan variasi dalam ukuran data, jumlah transaksi, dan pengukuran \textit{response time} untuk operasi. Selain itu, komponen ini juga dilengkapi dengan otomatisasi untuk menjalankan eksperimen berulang kali untuk mendapatkan data yang representatif dari eksperimen.

Secara umum, struktur \textit{Data Collector} terdiri dari:

% \begin{enumerate}
%     \item Komponen testing: Komponen ini bertanggung jawab untuk melakukan \textit{request} dan transaksi pada sistem.
%     \item Komponen \textit{logging} dan \textit{tracing}: Komponen ini mengelola pencatatan dan pelacakan operasi yang dilakukan oleh sistem secara keseluruhan.
%     \item Komponen \textit{reporting}: Komponen ini bertanggung jawab untuk mengumpulkan dan menyajikan hasil eksperimen dalam bentuk laporan.
% \end{enumerate}

Merujuk pada bagian \ref{sec:analisis-permasalahan}, \textit{Data Collector} memenuhi kebutuhan 3 dan 4, yaitu:

\begin{enumerate}
    \setcounter{enumi}{2}
    \item Sistem harus dapat memvariasikan ukuran data, tingkat ketahanan, kecepatan jaringan, dan kemampuan komputasi.
    \item Sistem harus dapat menjalankan eksperimen berulang kali untuk mendapatkan data persentil dari eksperimen.
\end{enumerate}

Sedangkan merujuk kebutuhan fungsional dan non-fungsional pada bagian \ref{sec:system-requirements}, \textit{Data Collector} memenuhi kebutuhan fungsional F-3, F-9, F-10, dan F-11
