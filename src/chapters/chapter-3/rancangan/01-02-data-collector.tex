\subsubsection{Data Collector}
\label{subsubsection:data-collector}

\textit{Data Collector} adalah bagian dari eksperimen yang bertugas untuk mengumpulkan data eksperimen. \textit{Data Collector} ini memiliki fitur yang memungkinkan variasi dalam ukuran data, jumlah transaksi, dan pengukuran \textit{response time} untuk operasi. Selain itu, komponen ini juga dilengkapi dengan otomatisasi untuk menjalankan eksperimen berulang kali untuk mendapatkan data yang representatif dari eksperimen.

Secara umum, struktur \textit{Data Collector} terdiri dari:

\begin{enumerate}
    \item Komponen benchmark: Komponen ini bertanggung jawab untuk menjalankan \textit{benchmark} dengan melakukan \textit{request} dan transaksi pada sistem. Variasi ukuran data, jumlah transaksi, dan pengukuran \textit{response time} dilakukan pada komponen ini. Komponen ini juga mengelola pengulangan eksperimen untuk mendapatkan data yang representatif.
    \item Komponen \textit{log management}: Komponen ini mengelola pencatatan dan pelacakan operasi yang dilakukan oleh sistem secara keseluruhan. \textit{Log} dari operasi dihasilkan oleh sistem, komponen ini hanya mengelola penyimpanan data eksperimen yang telah dikumpulkan menjadi bentuk yang dapat dianalisis.
\end{enumerate}

Merujuk pada bagian \ref{subsection:analisis-permasalahan}, \textit{Data Collector} memenuhi kebutuhan 3 dan 4, yaitu:

\begin{enumerate}
    \setcounter{enumi}{2}
    \item Sistem harus dapat memvariasikan ukuran data, tingkat ketahanan, kecepatan jaringan, dan kemampuan komputasi.
    \item Sistem harus dapat menjalankan eksperimen berulang kali untuk mendapatkan data persentil dari eksperimen.
\end{enumerate}

Sedangkan merujuk kebutuhan fungsional dan non-fungsional pada bagian \ref{subsection:system-requirements}, \textit{Data Collector} memenuhi kebutuhan fungsional F-3, F-9, dan F-10
