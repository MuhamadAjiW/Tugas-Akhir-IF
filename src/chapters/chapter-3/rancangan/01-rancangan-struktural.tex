\subsection{Rancangan Struktural}
\label{subsection:rancangan-struktural}

Seperti yang telah dijelaskan pada Bagian \ref{subsection:alternatif-solusi}, solusi yang dipilih adalah membuat sistem dengan mengkombinasikan \textit{in-memory key-value store} dan menggunakan RocksDB sebagai \textit{persistent database}. Replikasi dan \textit{erasure coding} hanya dilakukan pada data persisten yang disimpan pada \textit{database} tersebut sedangkan \textit{in-memory key-value store} berperan seperti \textit{cache} untuk meningkatkan kinerja sistem, khususnya untuk mengurangi rekonstruksi data dari \textit{persistent database} untuk \textit{erasure coding} pada operasi \textit{read}.

Mengikuti pendekatan pada Bagian \ref{subsubsection:pemilihan-pengembangan-sistem}, solusi dirancang untuk dibuat dalam bentuk komponen-komponen. Beberapa komponen disusun secara hierarkis dan membentuk subsistem yang mewakili domain tanggung jawab tertentu dalam sistem. Setiap komponen dirancang untuk memenuhi tujuan spesifik berdasarkan kebutuhan sistem.

\subsubsection{Node}
\label{subsubsection:node}

\textit{Node} adalah satuan fungsional utama yang berperan sebagai entitas dalam sistem \textit{database} terdistribusi yang dikembangkan. Sesuai dengan perancangan pada Bagian \ref{subsection:rancangan-struktural}, \textit{Node} dibangun secara modular dengan beberapa komponen yang tergabung dalam beberapa subkomponen. Subkomponen dapat berbentuk kelas atau kumpulan fungsi yang memiliki tanggung jawab spesifik dalam sistem.

\begin{enumerate}
    \item Komponen konsensus: Komponen ini bertanggung jawab mengelola \textit{state} dari \textit{Node} dan memastikan konsistensi data antar-\textit{Node} dalam sistem. Komponen ini menggunakan algoritma konsensus OmniPaxos yang telah dipilih pada Bagian \ref{subsubsection:pemilihan-algoritma-konsensus}.
    \item Komponen antarmuka client: Komponen ini menyediakan antarmuka bagi klien untuk berinteraksi dengan \textit{Node}. Antarmuka ini mendukung operasi \textit{read} dan \textit{write} data.
    \item Komponen komunikasi antar-\textit{Node}: Komponen ini mengelola komunikasi antar-\text{Node} dalam sistem. Komunikasi diperlukan untuk melakukan konsensus, replikasi data, dan pemulihan data.
    \item Komponen \textit{persistent store}: Komponen ini bertanggung jawab untuk menyimpan data secara \textit{persistent}. Data yang disimpan pada komponen ini akan direplikasi atau di-\textit{erasure coding} untuk memastikan ketersediaan dan keandalan data.
    \item Komponen \textit{in-memory store}: Komponen ini berfungsi sebagai \textit{cache} untuk meningkatkan kinerja sistem. Penyimpanan data pada komponen ini tidak menerapkan \textit{erasure coding} untuk mengurangi latensi pada operasi \textit{read}.
\end{enumerate}

Merujuk analisis kebutuhan sistem pada Bagian \ref{subsection:analisis-permasalahan}, \textit{Node} memenuhi kebutuhan 1 dan 2, yaitu:

\begin{enumerate}
    \item Sistem harus dapat mensimulasikan kondisi \textit{database} terdistribusi yang menggunakan replikasi ataupun \textit{erasure coding}.
    \item Sistem harus dapat menyimpan data secara \textit{persistent} untuk mensimulasikan kegagalan dan pemulihan.
\end{enumerate}

Sedangkan merujuk kebutuhan fungsional dan non-fungsional pada Bagian \ref{subsection:system-requirements}, \textit{Node} memenuhi kebutuhan fungsional F-1, F-2, F-4, F-5, F-6, F-7, dan F-8 serta non-fungsional NF-1, NF-2, NF-3, dan NF-4.

\subsubsection{Data Collector}
\label{subsubsection:data-collector}

\textit{Data Collector} adalah bagian dari eksperimen yang bertugas untuk mengumpulkan data eksperimen. \textit{Data Collector} ini memiliki fitur yang memungkinkan variasi dalam ukuran data, jumlah transaksi, dan pengukuran \textit{response time} untuk operasi. Selain itu, komponen ini juga dilengkapi dengan otomatisasi untuk menjalankan eksperimen berulang kali untuk mendapatkan data yang representatif dari eksperimen.

Secara umum, struktur \textit{Data Collector} terdiri dari:

\begin{enumerate}
    \item Komponen benchmark: Komponen ini bertanggung jawab untuk menjalankan \textit{benchmark} dengan melakukan \textit{request} dan transaksi pada sistem. Variasi ukuran data, jumlah transaksi, dan pengukuran \textit{response time} dilakukan pada komponen ini. Komponen ini juga mengelola pengulangan eksperimen untuk mendapatkan data yang representatif.
    \item Komponen \textit{log management}: Komponen ini mengelola pencatatan dan pelacakan operasi yang dilakukan oleh sistem secara keseluruhan. \textit{Log} dari operasi dihasilkan oleh sistem, komponen ini hanya mengelola penyimpanan data eksperimen yang telah dikumpulkan menjadi bentuk yang dapat dianalisis.
\end{enumerate}

Merujuk pada Bagian \ref{subsection:analisis-permasalahan}, \textit{Data Collector} memenuhi kebutuhan 3 dan 4, yaitu:

\begin{enumerate}
    \setcounter{enumi}{2}
    \item Sistem harus dapat memvariasikan ukuran data, tingkat ketahanan, kecepatan jaringan, dan kemampuan komputasi.
    \item Sistem harus dapat menjalankan eksperimen berulang kali untuk mendapatkan data persentil dari eksperimen.
\end{enumerate}

Sedangkan merujuk kebutuhan fungsional dan non-fungsional pada Bagian \ref{subsection:system-requirements}, \textit{Data Collector} memenuhi kebutuhan fungsional F-3, F-9, dan F-10
