\subsection{Alur Transaksi}
\label{subsection:system-flow}

Alur untuk transaksi \textit{read} dapat dilihat pada gambar \ref{fig:flow-read-mermaidjs} dengan \textit{request} masuk ke \textit{load balancer} kemudian disalurkan ke \textit{database node} yang tersedia. \textit{Database node} akan melakukan operasi \textit{read} pada \textit{key-value store} dan mengembalikan hasil operasi ke \textit{load balancer} untuk dikirimkan ke \textit{client} jika tersedia. Jika tidak, maka \textit{database node} akan melakukan rekonstruksi data dari \textit{erasure-coded persistent data} yang tersebar pada \textit{node-node} lainnya. Pada replikasi, data akan diambil dari \textit{node} lain yang memiliki data yang sama.

\begin{figure}[!ht]
    \centering
    \includegraphics[width=0.95\textwidth]{resources/chapter-3/flow-read-mermaidjs.png}
    \caption{Flow operasi \textit{read} dalam rancangan implementasi}
    \label{fig:flow-read-mermaidjs}
\end{figure}

Alur untuk transaksi \textit{read} dapat dilihat pada gambar \ref{fig:flow-write-mermaidjs} dengan \textit{request} masuk ke \textit{load balancer} kemudian disalurkan ke \textit{database node} yang merupakan leader. Leader kemudian akan melakukan operasi \textit{erasure coding} lalu menyebarkan \textit{shard} ke \textit{follower} yang tersedia. Setelah semua \textit{follower} menerima \textit{shard}, maka operasi \textit{write} dianggap selesai. Operasi \textit{write} pada replikasi juga akan menunggu semua \textit{follower} menerima data sebelum dianggap selesai. 

\begin{figure}[!ht]
    \centering
    \includegraphics[width=0.95\textwidth]{resources/chapter-3/flow-write-mermaidjs.png}
    \caption{Flow operasi \textit{write} dalam rancangan implementasi}
    \label{fig:flow-write-mermaidjs}
\end{figure}
