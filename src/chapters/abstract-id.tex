\clearpage
\chapter*{ABSTRAK}
\addcontentsline{toc}{chapter}{ABSTRAK}
\begin{center}
  \center
  \begin{singlespace}
    \large\bfseries\MakeUppercase{\thetitle}
    
    \normalfont\normalsize
    Oleh:
    
    \bfseries \theauthor
  \end{singlespace}
\end{center}

\begin{singlespace}
  \small
  Seiring bertambahnya jumlah pengguna perangkat \textit{Internet of Things (IoT)}, proses manajemen perangkat menjadi semakin lama dan kompleks. Salah satu proses yang penting dalam pengelolaan perangkat IoT yaitu proses \textit{deployment}. Proses ini biasanya dilakukan secara serial dari satu perangkat ke perangkat lainnya. Hal ini sangat memakan waktu dan tenaga yang sebenarnya dapat dimanfaatkan untuk proses lainnya untuk manajemen perangkat. Pengelolaan proses \textit{deployment} pada IoT adalah kunci untuk memastikan keandalan dan kinerja optimal dari jaringan yang semakin berkembang. Seiring dengan perkembangan teknologi IoT, metode \textit{Over The Air} (OTA) mulai diperkenalkan, metode ini memungkinkan \textit{deployment} dilakukan secara \textit{remote}. Meskipun metode OTA telah membawa banyak perbaikan, tantangan dalam manajemen perangkat yang skalabel tetap menjadi masalah.
  
  Penelitian ini bertujuan untuk merancang sistem \textit{remote deployment} untuk menyelesaikan masalah yang ditemukan pada metode OTA. Sistem harus bersifat \textit{platform agnostic}, sistem dapat berjalan pada platform apa pun, sistem harus mampu beroperasi pada perangkat yang memiliki sumber daya terbatas untuk mengelola aplikasi IoT dalam jumlah besar, dan sistem harus dapat melakukan \textit{deployment} ke perangkat tertentu. Dengan memenuhi \textit{requirement} tersebut, sistem dapat memberikan solusi yang efisien dan dapat digunakan untuk manajemen perangkat IoT skala besar.
  
  Penelitian ini menghasilkan PERISAI, sebuah implementasi sistem \textit{remote deployment} berbasis kubernetes sebagai alat orkestrasi. Kubernetes, yang telah terbukti efektif dalam manajemen aplikasi container di lingkungan \textit{cloud} dapat diimplementasikan pada lingkungan IoT dengan menggunakan distribusi K3s untuk mengatasi masalah \textit{resource constrained}. Dengan bantuan kubernetes, proses \textit{deployment} dapat dijalankan pada seluruh perangkat yang memiliki fitur kontainerisasi didalamnya. Dengan PERISAI, proses \textit{remote deployment} dapat dilakukan pada seluruh platform mulai dari \textit{Raspberry Pi}, \textit{virtual machine}, hingga kluster lokal yang dibuat dengan kontainer. Dilakukan proses \textit{deployment} pada dua buah \textit{Raspberry Pi} yang berjalan sebagai kubernetes kluster untuk menyalakan dua buah lampu yang terhubung dengan masing masing \textit{Raspberry Pi}. Alhasil proses \textit{remote deployment} dapat dilakukan dengan baik walaupun dijalankan pada Raspberry pi yang memiliki sumber daya yang terbatas.
  
  \textbf{\textit{Kata kunci: IoT, Raspberry Pi, Remote Deployment, Kubernetes }}
  
\end{singlespace}
\clearpage