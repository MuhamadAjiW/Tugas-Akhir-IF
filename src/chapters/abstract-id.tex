\clearpage
\chapter*{ABSTRAK}
\addcontentsline{toc}{chapter}{ABSTRAK}

\begin{center}
  \center
  \begin{singlespace}
    \large\bfseries\MakeUppercase{\thetitle}
    
    \normalfont\normalsize
    Oleh:
    
    \bfseries \theauthor
  \end{singlespace}
\end{center}

\begin{singlespace}
  \small
  Algoritma \textit{erasure coding} dapat mengurangi jumlah data yang dioperasikan dalam sebuah sistem. Dengan seringnya operasi \textit{I/O} menjadi faktor utama dalam kinerja sistem terdistribusi, \textit{erasure coding} menjadi solusi yang menarik untuk meningkatkan efisiensi penyimpanan dan kinerja sistem. Penelitian ini bertujuan menganalisis kinerja \textit{erasure coding} dibandingkan replikasi pada sistem \textit{key-value store database} terdistribusi, khususnya pada \textit{response time} sistem dalam operasi \textit{write} dan \textit{read}.
  
  Dalam memenuhi tujuan tersebut, penelitian ini mengimplementasikan sistem database terdistribusi yang mendukung kedua mekanisme redundansi tersebut serta membuat sistem \textit{benchmark} yang dapat memvariasikan parameter \textit{bandwidth} jaringan dan ukuran \textit{payload}. Hasil penelitian menunjukkan bahwa \textit{erasure coding} memiliki \textit{threshold} kondisi ketika \textit{response time} lebih rendah dibandingkan replikasi pada operasi write. Kondisi tersebut adalah ketika bandwidth jaringan terbatas dan ukuran payload cukup besar. Namun, replikasi konsisten mengungguli \textit{erasure coding} dalam operasi \textit{read} karena kompleksitas rekonstruksi data. Penggunaan \textit{erasure coding} tidak sesuai untuk \textit{distributed key-value store database} yang beroperasi dengan data kecil di \textit{data center} berkapasitas tinggi. Akan tetapi, \textit{erasure coding} masih dapat dipertimbangkan untuk sistem yang menangani data besar dengan infrastruktur jaringan terbatas dan sistem yang perlu efisiensi penyimpanan tinggi.
  
  \textbf{\textit{Kata kunci: Sistem Terdistribusi, Erasure Coding, Replikasi, Database, Key-Value Store }}
  
\end{singlespace}
\clearpage