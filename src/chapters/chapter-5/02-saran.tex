\section{Saran}
\label{sec:saran}

Adapun banyak kekurangan dari penelitian ini yang dapat diperbaiki pada penelitian selanjutnya. Beberapa saran yang dapat diberikan untuk penelitian selanjutnya adalah sebagai berikut:

\begin{enumerate}
  \item Penilitian ini tidak mencakup analisis proses \textit{recovery} dari \textit{erasure coding} yang dilakukan ketika terjadi kehilangan data. Perubahan konsensus yang dilakukan dengan adanya proses \textit{erasure coding} dapat mempengaruhi \textit{consistency} dan \textit{availability} dari \textit{database} terdistribusi.
  \item Penilitian ini tidak mencakup keseluruhan variabel yang dapat mempengaruhi kinerja sistem \textit{erasure coding} dan replikasi. Dapat dilakukan penelitian lebih lanjut untuk variasi dari konfigurasi sistem atau percobaan menggunakan beberapa sistem berbeda untuk mensimulasikan perbedaan kinerja. 
  \item Pengembangan sistem \textit{hybrid} yang menggabungkan \textit{erasure coding} dan replikasi seperti disarankan untuk melakukan penelitian lanjutan yang menganalisis kinerja sistem. Penggabungan ini berpotensi memberikan solusi dengan keunggulan kedua sistem \textit{erasure coding} dan replikasi. Sistem \textit{hybrid} dapat digunakan misalnya dengan menggunakan replikasi untuk data yang sering dibaca atau berukuran kecil dan menggunakan \textit{erasure coding} untuk data yang bersifat arsip, jarang diakses, berukuran besar, atau pada beban kerja yang didominasi operasi \textit{write}.
  \item Optimasi dapat dilakukan untuk meningkatkan kinerja \textit{erasure coding}. Implementasi \textit{erasure coding} yang digunakan pada penelitian ini adalah implementasi sederhana. Salah satu optimasi adalah penambahan berupa \textit{worker} yang melakukan algoritma \textit{encoding} atau \textit{decoding} secara paralel dapat meningkatkan kinerja \textit{erasure coding}.
  \item Penelitian ini tidak mencakup percobaan sistem nyata dengan jarak geografis serta variasi dari sistem yang digunakan. Hal ini berpotensi memberikan hasil yang berbeda dengan percobaan yang dilakukan pada penelitian ini.
\end{enumerate}