%--------------------------------------------------------------------%
%
% Template Credits goes to Petra Barus, Peb Ruswono Aryan, Faris Rizki Ekananda, Muhammad Garebaldhie ER Rahman
%
%--------------------------------------------------------------------%
%
% Berkas ini berisi struktur utama dokumen LaTeX yang akan dibuat.
%
%--------------------------------------------------------------------%
\documentclass[bahasa, 12pt, a4paper, onecolumn, oneside, final]{report}

\hyphenpenalty=100000
\tolerance=10000

\input{config/if-itb-thesis.sty}
\input{config/hypenation-id.tex}

\makeatletter

\makeatother

\addbibresource{references.bib}

\begin{document}

\title{Analisis Kinerja Erasure Coding terhadap Replikasi pada Key-Value Store Database Terdistribusi}
\date{}
\author{
  \authorname{} \\
  NIM: \authornim{}
}

\pagenumbering{roman}
\setcounter{page}{1}

\input{chapters/cover}
\clearpage
\pagestyle{empty}

\begin{center}
  \smallskip
  
  \Large \bfseries \MakeUppercase{\thetitle}
  \vfill
  
  \Large Laporan Tugas Akhir
  \vfill
  
  \large Oleh
  
  \Large \theauthor
  
  \large Program Studi Teknik Informatika \\
  
  \normalsize \normalfont
  Sekolah Teknik Elektro dan Informatika \\
  Institut Teknologi Bandung \\
  
  \vfill
  \normalsize \normalfont
  Bandung, \tanggalpengesahan \\
  Mengetahui

  \vspace{0.5cm}
  % Telah disetujui dan disahkan sebagai Laporan Tugas Akhir \\
  % Telah disetujui dan disahkan sebagai Draft Laporan Tugas Akhir \\  
  Pembimbing,
  
  \vfill
  \underline{Achmad Imam Kistijantoro, S.T, M.Sc., Ph.D.
  } \\
  NIP. 19730809 200604 1 001
  
\end{center}
\clearpage

\input{chapters/statement}

\pagestyle{plain}

\clearpage
\chapter*{ABSTRAK}
\addcontentsline{toc}{chapter}{ABSTRAK}

\begin{center}
  \center
  \begin{singlespace}
    \large\bfseries\MakeUppercase{\thetitle}
    
    \normalfont\normalsize
    Oleh:
    
    \bfseries \theauthor
  \end{singlespace}
\end{center}

\begin{singlespace}
  \small
  ::TODO: fill::

  \textbf{\textit{Kata kunci: ::TODO: fill:: }}
  
\end{singlespace}
\clearpage
\clearpage
\chapter*{ABSTRACT}
\addcontentsline{toc}{chapter}{ABSTRACT}

\begin{center}
  \center
  \begin{singlespace}
    \large\bfseries\MakeUppercase{Erasure Coding Performance Analysis Against Replication on Distributed Key-Value Store Database}
    
    \normalfont\normalsize
    By:
    
    \bfseries \theauthor
  \end{singlespace}
\end{center}


\begin{singlespace}
  \small
  Erasure coding algorithms can reduce the amount of data transferred in a distributed system. With I/O operations involving data transfer being a major factor in distributed system performance, erasure coding becomes an attractive solution to improve storage efficiency and system performance. This research aims to analyze the performance of erasure coding compared to replication in distributed key-value store databases, specifically focusing on the system's response time in write and read operations.

  To achieve this goal, a distributed database system supporting both redundancy mechanisms is implemented, along with a benchmarking system that can vary network bandwidth and payload size parameters. The research results show that erasure coding has a threshold condition when the response time is lower than replication in write operation. This condition occurs when the network bandwidth is limited and the payload size is sufficiently large. However, replication consistently outperforms erasure coding in read operations due to the complexity of data reconstruction. The use of erasure coding is not suitable for distributed key-value store databases that operate with small data at high-capacity data centers. However, erasure coding may still be considered for systems handling large data with limited network infrastructure and systems that require high storage efficiency.

  \textbf{\textit{Keywords: Distributed System, Erasure Coding, Replication, Database, Key-Value Store }}
\end{singlespace}
\clearpage

\clearpage
\chapter*{Kata Pengantar}
\addcontentsline{toc}{chapter}{KATA PENGANTAR}

::TODO: fill::

\begin{flushright}
  \vspace{0.5cm}
  Bandung, \tanggalpengesahan
  
  
  \vspace{1.5cm}
  
  \authorname
\end{flushright}

\titlespacing*{\chapter}{0pt}{0pt}{4pt}

% Setting judul toc, lot, lof, bib
\renewcommand{\contentsname}{DAFTAR ISI}
\renewcommand{\listfigurename}{DAFTAR GAMBAR}
\renewcommand{\listtablename}{DAFTAR TABEL}
\renewcommand{\bibname}{DAFTAR PUSTAKA}

% daftar isi, lampiran, gambar, table
\tableofcontents
\listofappendices
\listoffigures
\listoftables

\newpage

\pagenumbering{arabic}

%----------------------------------------------------------------%
% Konfigurasi Bab
%----------------------------------------------------------------%
\setcounter{page}{1}
\renewcommand{\chaptername}{BAB}
\renewcommand{\thechapter}{\Roman{chapter}}
%----------------------------------------------------------------%

%----------------------------------------------------------------%
% Daftar Bab
%----------------------------------------------------------------%

% Adjust spacing around section titles
\titlespacing*{\section}{0pt}{12pt}{8pt}
\titlespacing*{\section}{0pt}{8pt}{6pt}
\titlespacing*{\section}{0pt}{6pt}{4pt}

\input{chapters/chapter-1}
\chapter{Studi Literatur}
\label{chapter:studi-literatur}

Bab ini akan diisi oleh studi literatur yang berkaitan dengan topik persoalan tugas akhir untuk memberikan informasi mengenai dasar teori dan studi yang dipakai. Bab ini diharapkan dapat membantu pembaca untuk lebih mengerti tentang penelitian tugas akhir ini.

\section{\textit{Operating System}}
Secara informal, \textit{Operating System} adalah.

\section{\textit{Filesystem}}
Secara informal, \textit{filesystem} adalah.

\input{chapters/chapter-2/03-snapshot.tex}
\section{\textit{Version Control System}}
Secara informal, \textit{version control system} adalah.

\subsection{Centralized Version Control System}

Centralized Version Control System
\subsection{Decentralized Version Control System}

Decentralized Version Control System

\section{\textit{Backup}}
Secara informal, \textit{backup} adalah.


\chapter{Analisis Persoalan dan Rancangan Solusi}
\label{chapter:analisis-persoalan-dan-rancangan-solusi}

Tujuan utama penulisan bab ini adalah untuk menjelaskan proses solusi atas masalah pencarian kondisi ketika \textit{erasure coding} memiliki \textit{response time} yang lebih cepat dibandingkan replikasi. Bagian ini memaparkan proses analisis masalah, rancangan solusi, serta implementasi solusi.

\section{Analisis}

Bagian ini menganalisis permasalahan lalu menurunkan kebutuhan sistem yang akan dibangun. Setelah itu, bagian ini juga membandingkan beberapa alternatif solusi yang dapat digunakan untuk menyelesaikan permasalahan tersebut.

\subsection{Analisis Permasalahan}
\label{subsection:analisis-permasalahan}

% Berdasarkan latar belakang yang telah diuraikan pada \ref{sec:latar-belakang}, penggunaan \textit{erasure coding} pada sebuah sistem dapat mengurangi kebutuhan penyimpanan data dengan tetap menjaga integritas dan ketahanan data. Walaupun membutuhkan sumber daya komputasi dalam penerapannya, pengurangan ukuran data keseluruhan yang diperlukan menyebabkan juga turunnya ukuran data yang perlu dikirim ke \textit{node} lainnya.Hal ini menyebabkan mungkinnya terdapat suatu kondisi ketika {response time} yang diperlukan untuk melakukan operasi pada \textit{erasure coding} lebih rendah jika dibandingkan dengan melakukan operasi yang sama pada sistem yang menggunakan \textit{replikasi} untuk mencapai ketahanan tersebut.

Berdasarkan latar belakang dan studi literatur yang telah ditentukan, \textit{erasure coding} memiliki potensi untuk menghasilkan \textit{response time} yang lebih rendah dibandingkan dengan sistem berbasis \textit{replikasi} karena walaupun membutuhkan komputasi, ukuran data yang dikirimkan untuk ketahanan lebih rendah dibandingkan replikasi. Riset ini akan menganalisis faktor-faktor untuk mencapai kondisi tersebut. Faktor yang dianalisis antara lain ukuran data yang besar dan jaringan yang lambat. Dengan demikian, diperlukan sistem yang dapat mensimulasikan operasi pada sistem \textit{erasure coding} dan replikasi sekaligus memvariasikan faktor-faktor tersebut dalam operasinya tanpa memengaruhi kinerja sistem di sisi lain.

Dari permasalahan tersebut, dirumuskan kebutuhan sebuah perangkat lunak antara lain
\begin{enumerate}

    \item Sistem harus dapat mensimulasikan kondisi \textit{database} terdistribusi yang menggunakan replikasi ataupun \textit{erasure coding}.
    \item Sistem harus dapat menyimpan data secara \textit{persistent}.
    \item Sistem harus dapat memvariasikan ukuran data dan kecepatan jaringan.
    \item Sistem harus dapat menjalankan eksperimen berulang kali untuk mendapatkan data dari eksperimen.

\end{enumerate}

Penjelasan lebih detail mengenai kebutuhan sistem terdapat pada lampiran \ref{subsection:rancangan-struktural}.
\subsection{Analisis Kebutuhan Sistem}
\label{sec:analisis-kebutuhan-sistem}

Berdasarkan bagian \ref{sec:analisis-permasalahan}, dirumuskan kebutuhan sebuah perangkat lunak antara lain
\begin{enumerate}

    \item Sistem harus dapat mensimulasikan kondisi \textit{database} terdistribusi yang menggunakan replikasi ataupun \textit{erasure coding}.
    \item Sistem harus dapat menyimpan data secara \textit{persistent} untuk mensimulasikan kegagalan dan pemulihan.
    \item Sistem harus dapat memvariasikan ukuran data, tingkat ketahanan, kecepatan jaringan, dan kemampuan komputasi.
    \item Sistem harus dapat menjalankan eksperimen berulang kali untuk mendapatkan median, dan data persentil dari eksperimen.

\end{enumerate}
\subsection{Alternatif Solusi}
\label{subsection:alternatif-solusi}

Berdasarkan permasalahan pada Bagian \ref{subsection:analisis-permasalahan} dan kebutuhan pada Bagian \ref{subsection:system-requirements}, terdapat berbagai macam alternatif solusi untuk menyelesaikan permasalahan tersebut. Pada penelitian ini, solusi yang dipilih adalah dengan membuat sistem yang mengkombinasikan \textit{in-memory key value store} dengan RocksDB sebagai \textit{persistent storage} dengan Reed-Solomon sebagai algoritma \textit{erasure coding} dan OmniPaxos sebagai protokol konsensus. Bagian ini akan menjelaskan alternatif solusi yang dipertimbangkan, perbandingan solusi tersebut, serta alasan pemilihan solusi yang digunakan dalam penelitian ini.

\input{chapters/chapter-3/analisis/03-01-alteratif-sistem.tex}
\input{chapters/chapter-3/analisis/03-02-alteratif-bahasa.tex}
\input{chapters/chapter-3/analisis/03-03-alteratif-memory.tex}
\input{chapters/chapter-3/analisis/03-04-alteratif-persistent.tex}
\input{chapters/chapter-3/analisis/03-05-alteratif-algoritma-ec.tex}
\input{chapters/chapter-3/analisis/03-06-alteratif-algoritma-konsensus.tex}


\section{Rancangan}
\label{sec:rancangan}

Bagian ini menjelaskan rancangan sistem yang akan diimplementasikan. Rancangan sistem merujuk pada kebutuhan sistem yang dihasilkan pada Bagian \ref{subsection:system-requirements} dan alternatif solusi yang dipilih pada Bagian \ref{subsection:alternatif-solusi}. Rancangan sistem ini mencakup struktur sistem, arsitektur sistem, alur transaksi, dan detail komponen yang akan diimplementasikan. Komponen didefinisikan sebagai sebuah unit fungsional yang berdiri sendiri dan dapat disusun.

\subsection{Rancangan Struktural}
\label{subsection:rancangan-struktural}

Seperti yang telah dijelaskan pada \ref{subsection:alternatif-solusi}, solusi yang dipilih adalah membuat sistem dengan mengkombinasikan \textit{in-memory key-value store} dan menggunakan RocksDB sebagai \textit{persistent database}. Replikasi dan \textit{erasure coding} hanya dilakukan pada data persisten yang disimpan pada \textit{database} tersebut sedangkan \textit{in-memory key-value store} berperan seperti \textit{cache} untuk meningkatkan kinerja sistem, khususnya untuk mengurangi rekonstruksi data dari \textit{persistent database} untuk \textit{erasure coding} pada operasi \textit{read}.

Mengikuti pendekatan pada \ref{subsubsection:pemilihan-pengembangan-sistem}, solusi dirancang untuk dibuat dalam bentuk komponen-komponen. Beberapa komponen disusun secara hierarkis dan membentuk subsistem yang mewakili domain tanggung jawab tertentu dalam sistem. Setiap komponen dirancang untuk memenuhi tujuan spesifik berdasarkan kebutuhan sistem.

\input{chapters/chapter-3/rancangan/01-01-node.tex}
\input{chapters/chapter-3/rancangan/01-02-data-collector.tex}
\subsection{Arsitektur Sistem}
\label{subsection:system-architecture}

Rancangan struktural yang telah dijelaskan pada bagian \ref{subsection:rancangan-struktural} diimplementasikan sebagai sistem terdistribusi yang terdiri dari beberapa \textit{Node}. Komponen-komponen tersebut disusun dalam arsitektur sistem seperti yang ditunjukkan pada gambar \ref{fig:general-architecture}.

\begin{figure}[ht]
    \centering
    \includegraphics[width=0.75\textwidth]{resources/chapter-3/general-architecture.png}
    \caption{Gambaran Arsitektur Sistem Eksperimen}
    \label{fig:general-architecture}
\end{figure}

Dalam gambaran umum arsitektur sistem, sistem \textit{Key-Value Store} terdistribusi yang dikembangkan adalah \textit{Node}. Dalam \textit{Node} terdapat komponen penyimpanan data yang dapat diakses dan dikelola secara terdistribusi dengan dapat replikasi atau \textit{erasure coding}. Setiap \textit{Node} dapat berkomunikasi dengan \textit{Node} lainnya untuk melakukan konsensus, replikasi data, dan pemulihan data.

Sementara itu, \textit{Data collector} merupakan \textit{script} eksternal yang tidak memiliki peran dalam sistem terdistribusi, namun berfungsi untuk mengumpulkan data eksperimen. Keterhubungan \textit{Data Collector} terbatas pada komponen \textit{testing} yang dapat mengatur konfigurasi dari sistem yang akan diujikan. Detail dari masing-masing komponen akan dijelaskan pada bagian \ref{subsection:detail-komponen}. 

% Arsitektur dari sistem mengasumsikan kebutuhan untuk konsistensi yang tinggi. Untuk mencapai konsistensi tersebut, operasi \textit{write} dilakukan secara \textit{synchronous} dengan distribusi replikasi dan \textit{erasure coding} dianggap selesai ketika nilai ketahanan yang diinginkan sudah tercapai.

% Karena sistem bersifat terdistribusi, maka diperlukan sebuah algoritma konsensus untuk mengelola konsistensi antar \textit{Node}. Algoritma konsensus yang digunakan algoritma konsensus \textit{paxos} yang disesuaikan dengan kebutuhan. Salah satu penyesuaian yang dilakukan adalah mengadopsi pola \textit{leader-follower} untuk memudahkan sinkronisasi data dan mempercepat transaksi. Dengan adanya leader, fase 1 dari algoritma \textit{paxos} dapat dihilangkan dengan membuat proposal dari leader selalu memiliki nilai paling tinggi. Detail implementasi \textit{paxos} akan dijelaskan di bagian \ref{subsection:detail-komponen}. Diagram gambaran arsitektur sistem dapat dilihat pada gambar \ref{fig:general-architecture}.

% Operasi \textit{write} akan secara ekslusif disalurkan pada \textit{leader}. Kemudian untuk ketahanan, data akan didistribusikan pada \textit{follower} sesuai dengan konfigurasi \textit{node}. Sementara itu, operasi \textit{read} dapat dilakukan pada \textit{Node} manapun. Pada sistem \textit{erasure coding}, jika pada \textit{node} tersebut tidak terdapat nilai data yang dicari, maka \textit{Node} akan melakukan \textit{request} ke semua node lainnya untuk melakukan rekonstruksi data.

\subsection{Alur Transaksi}
\label{subsection:system-flow}

Mengikuti batasan pada Bagian \ref{sec:batasan-masalah}, transaksi yang akan diimplementasikan dalam sistem ini adalah transaksi \textit{write} dan \textit{read}. Sistem akan mendukung dua skema penyimpanan, yaitu replikasi dan \textit{erasure coding}, yang akan mempengaruhi alur transaksi.

\begin{figure}[!ht]
    \centering
    \includegraphics[width=0.95\textwidth]{resources/chapter-3/flow-write.png}
    \caption{Flow operasi \textit{write} dalam rancangan implementasi}
    \label{fig:flow-write-mermaidjs}
\end{figure}

Alur untuk transaksi \textit{write} dapat dilihat pada Gambar \ref{fig:flow-write-mermaidjs} dengan \textit{request} masuk ke \textit{Node}. Node kemudian akan melakukan operasi \textit{erasure coding} lalu menyebarkan \textit{shard} ke \textit{Node} lain sesuai dengan algoritma konsensus yang digunakan. Operasi \textit{write} dianggap selesai setelah konsensus mencapai kuorum.

\begin{figure}[!ht]
    \centering
    \includegraphics[width=0.95\textwidth]{resources/chapter-3/flow-read.png}
    \caption{Flow operasi \textit{read} dalam rancangan implementasi}
    \label{fig:flow-read-mermaidjs}
\end{figure}

Alur untuk transaksi \textit{read} dapat dilihat pada Gambar \ref{fig:flow-read-mermaidjs} dengan \textit{request} masuk ke \textit{Node} yang tersedia. \textit{node} melakukan operasi \textit{read} pada \textit{key-value store} dan mengembalikan hasil operasi ke \textit{client} jika tersedia. Jika tidak, maka \textit{Node} melakukan rekonstruksi data dari \textit{erasure-coded persistent data} yang tersebar pada \textit{node-node} lainnya. Pada replikasi, data diambil dari \textit{Node} lain yang memiliki data yang sama.
\subsection{Rancangan Detail Komponen}
\label{subsection:detail-komponen}

Berdasarkan rancangan struktural yang sudah dijelaskan pada bagian \ref{subsection:rancangan-struktural}, sistem akan diimplementasikan sebagai kumpulan komponen. Masing-masing komponen tersebut memiliki peran dan tanggung jawab yang berbeda dalam sistem eksperimen.

\input{chapters/chapter-3/rancangan/04-01-detail-node.tex}
\input{chapters/chapter-3/rancangan/04-02-detail-konsensus.tex}
\input{chapters/chapter-3/rancangan/04-03-detail-client-interface.tex}
\input{chapters/chapter-3/rancangan/04-04-detail-internode-interface.tex}
\input{chapters/chapter-3/rancangan/04-05-detail-persistent-store.tex}
\input{chapters/chapter-3/rancangan/04-06-detail-memory-store.tex}
\input{chapters/chapter-3/rancangan/04-07-detail-data-collector.tex}
\input{chapters/chapter-3/rancangan/04-08-detail-testing.tex}
\input{chapters/chapter-3/rancangan/04-09-detail-logging.tex}


\chapter{Rencana pelaksanaan}
\label{chapter:rencana-pelaksanaan}
Bab ini akan memaparkan berupa jadwal dan risiko-risiko yang mungkin dihadapi beserta rencana mitigasinya. Tujuannya sebagai gambaran umum mengenai rencana pelaksanaan penelitian tugas akhir ini.

\section{Jadwal Pelaksanaan}
\label{sec:jadwal-pelaksanaan}
Aktivitas-aktivitas yang akan dilakukan untuk memenuhi kebutuhan eksperiman adalah seperti yang dapat dilihat pada tabel \ref{tab:rincian-aktivitas}.

\begin{table}[ht]
\centering
\caption{Daftar Aktivitas Tugas Akhir}
\begin{tabular}{|p{1cm}|p{6cm}|p{6cm}|}
\hline
\rowcolor{black!10} ID & Kegiatan & Hasil \\ \hline
K-1  & Melakukkan kajian pustaka & Bab II pada proposal tugas akhir \\ \hline
K-2  & Melakukan analisis permasalahan untuk topik tugas akhir & Bab I pada proposal tugas akhir \\ \hline
K-3  & Merancang solusi penyelesaian masalah & Bab III dan Bab IV pada proposal tugas akhir \\ \hline
K-4  & Melakukan finalisasi proposal tugas akhir & Proposal tugas akhir \\ \hline
K-5  & Melakukan seminar proposal tugas akhir & - \\ \hline
K-6  & Melakukan implementasi sistem \textit{key-value store database} untuk membandingkan \textit{erasure coding} dengan replikasi \textit{} & Sistem \textit{key-value store database} yang dapat mengumpulkan data untuk perbandingan \textit{erasure coding} dan replikasi \\ \hline
K-7  & Melakukan eksperimen untuk pengumpulan data \textit{response time} operasi pada sistem \textit{erasure coding} dan replikasi yang divariasikan sesuai rancangan & Data \textit{response time} operasi pada sistem \textit{erasure coding} dan replikasi \\ \hline
K-8  & Melakukan analisis data eksperimen & Hasil analisis \\ \hline
K-9  & Menyusun laporan tugas akhir secara keseluruhan & Laporan tugas akhir \\ \hline
K-10  & Melakukan sidang tugas akhir & - \\ \hline
\end{tabular}
\label{tab:rincian-aktivitas}
\end{table}

Dari aktivitas-aktivitas tersebut, jadwal perencanaan pelaksanaan masing-masing aktivitas dapat dilihat pada tabel \ref{tab:jadwal-part1} dan \ref{tab:jadwal-part2}.

\begin{table}[ht]
\centering
\caption{Jadwal Tugas Akhir Periode Oktober 2024 hingga Februari 2025}
\resizebox{\textwidth}{!}{
\begin{ganttchart}[
    title/.append style={fill=black!10},
    time slot format=simple,
    x unit=0.9cm,
    y unit chart=0.7cm,
    hgrid,
    vgrid
    ]{1}{20}

    \gantttitle{2024}{12}
    \gantttitle{2025}{8} \\

    \gantttitle{Oktober}{4}
    \gantttitle{November}{4}
    \gantttitle{Desember}{4}
    \gantttitle{Januari}{4}
    \gantttitle{Februari}{4}\\
    
    \gantttitle{14}{1}
    \gantttitle{21}{1}
    \gantttitle{28}{1}
    \gantttitle{4}{1}
    \gantttitle{11}{1}
    \gantttitle{18}{1}
    \gantttitle{25}{1} 
    \gantttitle{2}{1}
    \gantttitle{9}{1}
    \gantttitle{16}{1}
    \gantttitle{23}{1}
    \gantttitle{30}{1}
    \gantttitle{6}{1}
    \gantttitle{13}{1}
    \gantttitle{20}{1}
    \gantttitle{27}{1}
    \gantttitle{3}{1}
    \gantttitle{10}{1}
    \gantttitle{17}{1}
    \gantttitle{24}{1}\\

    \ganttbar{K-1}{1}{4} \\
    \ganttbar{K-2}{5}{6} \\
    \ganttbar{K-3}{7}{12} \\
    \ganttbar{K-4}{12}{13} \\
    \ganttbar{K-5}{14}{16} \\

\end{ganttchart}
}
\label{tab:jadwal-part1}
\centering
\caption{Jadwal Tugas Akhir Periode Maret 2025 hingga Juni 2025}
\resizebox{\textwidth}{!}{
\begin{ganttchart}[
    title/.append style={fill=black!10},
    time slot format=simple,
    x unit=0.9cm,
    y unit chart=0.7cm,
    hgrid,
    vgrid
    ]{1}{17}

    \gantttitle{2025}{17} \\

    \gantttitle{Maret}{5}
    \gantttitle{April}{4}
    \gantttitle{Mei}{4}
    \gantttitle{Juni}{4}\\
    
    \gantttitle{3}{1}
    \gantttitle{10}{1}
    \gantttitle{17}{1}
    \gantttitle{24}{1}
    \gantttitle{31}{1}
    \gantttitle{7}{1}
    \gantttitle{14}{1} 
    \gantttitle{21}{1}
    \gantttitle{28}{1}
    \gantttitle{5}{1}
    \gantttitle{12}{1}
    \gantttitle{19}{1}
    \gantttitle{26}{1}
    \gantttitle{2}{1}
    \gantttitle{9}{1}
    \gantttitle{16}{1}
    \gantttitle{23}{1}\\

    \ganttbar{K-6}{1}{4} \\
    \ganttbar{K-7}{5}{6} \\
    \ganttbar{K-8}{7}{12} \\
    \ganttbar{K-9}{12}{13} \\
    \ganttbar{K-10}{14}{16} \\

\end{ganttchart}
}
\label{tab:jadwal-part2}
\end{table}




\section{Risiko}
\label{sec:risiko}
Bagian ini menjelaskan tentang implementasi ::TODO: fill::



\chapter{Penutup}

Bab Kesimpulan dan Saran akan menjadi bagian akhir dan penutup dari penelitian tugas akhir ini. Bab ini akan membahas kesimpulan yang berisi ketercapaian tujuan penelitian tugas akhir dengan permasalahan yang diselesaikan dalam penelitian tugas akhir. Selain itu, bab ini akan membahas saran yang dapat dilakukan untuk pengembangan atau penelitian selanjutnya.

\section{Kesimpulan}
Penelitian tugas akhir ini ::TODO: fill::

\section{Saran}
Adapun banyak kekurangan dan kelemahan yang ditemukan dalam penelitian tugas akhir ini. Berikut adalah beberapa saran yang dapat dilakukan untuk pengembangan atau penelitian selanjutnya ::TODO: fill::
\begin{enumerate}
  \item Dapat diimplementasikan ::TODO: fill::
\end{enumerate}
%---------------------------------------------------------------%

% Daftar pustaka
\printbibliography

% Setting judul lampiran
\titlespacing*{\chapter}{0pt}{0pt}{0pt}
\titlespacing*{\section}{0pt}{0pt}{*1}

% Setting judul anak lampiran
\titleformat*{\section}{\bfseries}

\appendix

\chapter{Detail Arsitektur Sistem}
\label{appendix:detailed-architecture}

Gambar \ref{fig:detailed-architecture} menunjukkan gambaran detail arsitektur sistem eksperimen yang digunakan dalam penelitian ini. Arsitektur ini mencakup komponen-komponen utama yang berperan dalam pengumpulan data, pengelolaan konsistensi, dan replikasi data antar \textit{Node} seperti yang telah dijelaskan pada bagian \ref{sec:rancangan}

\begin{figure}[ht]
    \centering
    \includegraphics[width=\textwidth]{resources/appendix/detailed-architecture.png}
    
    \caption{Gambaran Detail Arsitektur Sistem Eksperimen}
    \label{fig:detailed-architecture}
\end{figure}
\chapter{Pemetaan Pengujian}
\label{appendix:pemetaan-pengujian}

Lampiran ini menyediakan tabel yang memetakan kebutuhan fungsional dan non-fungsional ke pengujian yang dilakukan. Tujuannya adalah memudahkan pemetaan antara kebutuhan sistem dan pengujian yang dilakukan serta memastikan bahwa semua kebutuhan telah diuji.

\begin{table}
    \centering
    \caption{Perbandingan kakas in-memory key-value store}
    \resizebox{\textwidth}{!}{
        \begin{tabular}{|p{3cm}|p{3cm}|p{3cm}|p{3cm}|p{3cm}|}
            \hline
            \rowcolor{black!10} ID Kebutuhan & ID Pengujian & Skenario & Ekspektasi & Realita \\ \hline

        \end{tabular}
    }
    \label{tab:pemetaan-pengujian}
\end{table}
\chapter{Kode Analisis Benchmark Write}
\label{appendix:write-benchmark-code}

Lampiran ini menunjukkan kode yang digunakan untuk melakukan visualisasi dari hasil \textit{benchmark} pada skenario \textit{write} dengan \textit{payload} besar dan \textit{bandwidth} internet menengah. Kode ini bertujuan untuk memperlihatkan proses yang dilakukan untuk menghasilkan grafik dan diagram dalam analisis. Analisis dilakukan dengan menggunakan kakas Jupyter Notebook dan bahasa pemrograman Python, dengan pustaka seperti NumPy, Pandas, Matplotlib, dan Seaborn.

\includepdf[pages=-]{resources/appendix/benchmark_k6_write.pdf}
\chapter{Kode Analisis Regresi Write}
\label{appendix:write-regression-code}

Lampiran ini menunjukkan kode yang digunakan untuk melakukan analisis regresi pada skenario \textit{write} dengan \textit{payload} besar dan \textit{bandwidth} internet menengah. Kode ini bertujuan untuk memperkirakan titik impas antara sistem berbasis \textit{erasure coding} dan sistem berbasis replikasi. Analisis dilakukan dengan menggunakan kakas Jupyter Notebook dan bahasa pemrograman Python, dengan pustaka seperti NumPy, Pandas, Matplotlib, dan Scikit-learn.

\includepdf[pages=-]{resources/appendix/benchmark_regression_write.pdf}
\chapter{Kode Analisis Benchmark Read}
\label{appendix:read-benchmark-code}

Lampiran ini menunjukkan kode yang digunakan untuk melakukan visualisasi dari hasil \textit{benchmark} pada skenario \textit{read} dengan \textit{payload} besar dan \textit{bandwidth} internet menengah. Kode ini bertujuan untuk memperlihatkan proses yang dilakukan untuk menghasilkan grafik dan diagram dalam analisis. Analisis dilakukan dengan menggunakan kakas Jupyter Notebook dan bahasa pemrograman Python, dengan pustaka seperti NumPy, Pandas, Matplotlib, dan Seaborn.

\includepdf[pages=-]{resources/appendix/benchmark_k6_read.pdf}
\chapter{Kode Analisis Regresi Read}
\label{appendix:read-regression-code}

Lampiran ini menunjukkan kode yang digunakan untuk melakukan analisis regresi pada skenario \textit{read} dengan \textit{payload} besar dan \textit{bandwidth} internet menengah. Kode ini bertujuan untuk memperkirakan titik impas antara sistem berbasis \textit{erasure coding} dan sistem berbasis replikasi. Analisis dilakukan dengan menggunakan kakas Jupyter Notebook dan bahasa pemrograman Python, dengan pustaka seperti NumPy, Pandas, Matplotlib, dan Scikit-learn.

\includepdf[pages=-]{resources/appendix/benchmark_regression_read.pdf}

\end{document}